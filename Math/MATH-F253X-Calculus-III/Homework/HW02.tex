\documentclass[10pt, letterpaper]{article}

% Files using this must be two subfolders
% deep. Adjust the number of ../ for the
% depth of the file.
% Imports
\usepackage{fancyhdr}
\usepackage{geometry}
\usepackage{icomma}
\usepackage{amsmath}
\usepackage{multicol}
\usepackage{mathptmx}
\usepackage{anyfontsize}
\usepackage{t1enc}
\usepackage{tabto}
\usepackage{listings}
\usepackage{filecontents}
\usepackage{subcaption}
\usepackage{tikz}
\usepackage[parfill]{parskip}
\usepackage{graphicx}
\usepackage[]{mdframed}
\usepackage{amsmath}
\usepackage[makeroom]{cancel}
\usepackage{pgfplots}
\usepackage{pgfplotstable}
\usepackage{xfrac}
\usepackage{amssymb}
\usepackage{mathtools}
\pgfplotsset{compat=1.18}
\usetikzlibrary{patterns}
\usepgfplotslibrary{polar}
\usepgfplotslibrary{fillbetween}

\geometry{margin=2.5cm}

\newcommand{\name}{Kaleb Burris}
\newcommand{\classname}{MATH F253, Elizabeth S. Allman, University of Alaska Fairbanks}
\newcommand{\assignment}{FILL IN ASSIGNMENT NAME}

\pagestyle{fancy}

\fancyhead[L]{
    \name 
    \newline
    \classname
    \newline
    \assignment
}

\newcommand{\horizontal}{\noindent\rule{\hsize}{0.4pt}}

\setlength{\headheight}{42pt}
\setlength{\headsep}{0.25in}
\setlength{\columnsep}{0.35cm}
\setlength{\columnseprule}{1pt}

\usepackage[T1]{fontenc}
\usepackage{lmodern}
\usepackage{tkz-fct}

% Put class number, class name, and professor 
% name.
% Use only in case of emergency, this
% should be covered by the preamble.
% \renewcommand\classname{}

% Put the assignment name with \S if 
% necessary for the section and the question 
% numbers.
\renewcommand\assignment{Homework 2}

\begin{document}
    % Templates
    \iffalse
    % Use these for equations.
    \begin{equation*}
        \begin{gathered}
            Equations go here.
        \end{gathered}
    \end{equation*}

    % Use this if a line of math is too long.
    \resizebox{\hsize}{!}{$Long equation goes here$}

    % Use these for multiple columns.
    \begin{multicol*}{# of columns}
        % Remove the * if you want the columns to be balanced.
    \end{multicol*}

    % Use this to add a horizontal line.
    \horizontal

    \fi

    % Begin homework here.
    %%%%%%%%%%%%%%%%%%%%%%

    \paragraph{2.4:   184, 188, 190, 195, 204, 205, 207, 210, 212, 214, 219, 228, 235}

    \begin{enumerate}
        \item[184.]\mbox{}
        \begin{mdframed}
            \begin{equation*}
                \begin{gathered}
                    \vec{u} \cross \vec{v} =
                    \begin{vmatrix}
                        \ihat & \jhat & \khat   \\
                        3 & 2 & -1              \\
                        1 & 1 & 0               \\
                    \end{vmatrix}
                    = \ihat(0-(-1)) - \jhat(0-(-1)) + \khat(3-2) 
                    = \ihat + \jhat - \khat
                    = \boxed{\expval{1, -1, 1}}
                \end{gathered}
            \end{equation*}
        \end{mdframed}

        \item[188.]\mbox{}
        \begin{mdframed}
            \begin{equation*}
                \begin{gathered}
                    \jhat(\khat \cross \jhat + 2\jhat \cross \ihat 
                          - 3 \jhat \cross \jhat + 5\ihat \cross \khat)
                          = \jhat(-\ihat - 2\khat - 5\jhat)
                          = \khat - 2\ihat = \boxed{-2, 0, 1}
                \end{gathered}
            \end{equation*}
        \end{mdframed}

        \item[190.]\mbox{}
        \begin{mdframed}
            \begin{equation*}
                \begin{gathered}
                    \vec{u} \cross \vec{v} =
                    \begin{vmatrix}
                        \ihat & \jhat & \khat   \\
                        2 & 6 & 1               \\
                        3 & 0 & 1               \\
                    \end{vmatrix}
                    = \ihat(6) - \jhat(2 - 3) + \khat(0 - 18)
                    = 5\ihat + \jhat -18\khat   \\
                    \frac{6\ihat + \jhat -18\khat}
                         {\sqrt{36 + 1 + 324}}
                    = \boxed{\frac{6\ihat}{19} + \frac{\jhat}{19} - \frac{18\khat}{19}}
                \end{gathered}
            \end{equation*}
        \end{mdframed}

        \item[195.]\mbox{}
        \begin{mdframed}
            \begin{equation*}
                \begin{gathered}
                    \text{Given: } \vec{u} \neq \vec{0} \quad \vec{v} \neq \vec{0}  \\
                    2
                \end{gathered}
            \end{equation*}
        \end{mdframed}
    \end{enumerate}


\end{document}