\documentclass[10pt, letterpaper]{article}

% Files using this must be two subfolders
% deep. Adjust the number of ../ for the
% depth of the file.
\providecommand\pointsize{10pt}

\documentclass[\pointsize, letterpaper]{article}

% Imports
\usepackage{fancyhdr}
\usepackage{pgfplots}
\usepackage{geometry}
\usepackage{icomma}
\usepackage{amsmath}
\usepackage{multicol}
\usepackage{mathptmx}
\usepackage{anyfontsize}
\usepackage{t1enc}
\usepackage{tabto}
\usepackage{listings}
\usepackage{filecontents}
\usepackage{subcaption}
\usepackage{tikz}
\usepackage[parfill]{parskip}
\usepackage{graphicx}
\usepackage[]{mdframed}
\usepackage{amsmath}
\usepackage[makeroom]{cancel}
\pgfplotsset{compat=1.18}

\geometry{margin=2.5cm}

\newcommand{\name}{Kaleb Burris}
\newcommand{\classname}{MATH F252, Dr. J. Gimbel}
\newcommand{\assignment}{FILL IN ASSIGNMENT NAME}

\pagestyle{fancy}

\fancyhead[L]{
    \name 
    \newline
    \classname
    \newline
    \assignment
}

\newcommand{\horizontal}{\noindent\rule{\hsize}{0.4pt}}

\setlength{\headheight}{42pt}
\setlength{\headsep}{0.25in}
\setlength{\columnsep}{0.35cm}
\setlength{\columnseprule}{1pt}
\usepackage{tkz-fct}

% Put class number, class name, and professor 
% name.
% Use only in case of emergency, this
% should be covered by the preamble.
% \renewcommand\classname{}

% Put the assignment name with \S if 
% necessary for the section and the question 
% numbers.
\renewcommand\assignment{Homework 1}

\begin{document}
    % Templates
    \iffalse
    % Use these for equations.
    \begin{equation*}
        \begin{gathered}
            Equations go here.
        \end{gathered}
    \end{equation*}

    % Use this if a line of math is too long.
    \resizebox{\hsize}{!}{$Long equation goes here$}

    % Use these for multiple columns.
    \begin{multicol*}{# of columns}
        % Remove the * if you want the columns to be balanced.
    \end{multicol*}

    % Use this to add a horizontal line.
    \horizontal

    \fi

    % Begin homework here.
    %%%%%%%%%%%%%%%%%%%%%%

    \paragraph*{2.1: 4, 7, 9, 12, 14, 17, 26, 30, 32, 38, 40, 46}

    \begin{itemize}
        \item [4.] $\vec{RP}$ \\
        
        \begin{mdframed}
            \begin{equation*}
                \begin{gathered}
                    \text{Given: } R(-3,7) \text{ and } P(-1, 3)                \\
                    \vec{RP} =  \expval{-1 - (-3), 3 - 7} = \boxed{\text{a.} \expval{2, -4}} =
                                \boxed{\text{b.}\, 2 \ihat - 4 \jhat}
                \end{gathered}
            \end{equation*}
        \end{mdframed}
        
        \item [7.] $2\vec{PQ} - 2\vec{PR}$ \\

        \begin{mdframed}
            \begin{equation*}
                \begin{gathered}
                    \text{Given: } R(-3,7), P(-1, 3), \text{ and } Q(1, 5)  \\
                    2\vec{PQ} - 2\vec{PR} = 
                    2\expval{1 - (-1), 5 - 3} - 2\expval{-3 - (-1), 7 - 3}  \\
                    = 2\expval{2, 2} - 2\expval{-2, 4} = 
                    \expval{4, 4} - \expval{-4, 8}  =                       \\
                    = \boxed{\text{a.} \expval{8, -4}} = \boxed{\text{b.}\, 8\ihat - 4\jhat}
                \end{gathered}
            \end{equation*}
        \end{mdframed}

        \item [9.] The unit vector in the direction of $\vec{PQ}$ \\
        
        \begin{mdframed}
            \begin{equation*}
                \begin{gathered}
                    \text{As found: } \vec{PQ} = \expval{2, 2}                  \\
                    \text{UV of } \vec{PQ} = \frac{\expval{2, 2}}{||\vec{PQ}||} \\
                    \frac{\expval{2, 2}}{\sqrt{2^2 + 2^2}} 
                    = \frac{\expval{2, 2}}{2\sqrt{2}}
                    = \expval{\frac{1}{\sqrt{2}}, \frac{1}{\sqrt{2}}}
                    = \expval{\frac{\sqrt{2}}{2}, \frac{\sqrt{2}}{2}}
                    = \boxed{\frac{\sqrt{2}}{2}\ihat + \frac{\sqrt{2}}{2}\jhat}
                \end{gathered}
            \end{equation*}
        \end{mdframed}

        \item [12.] A vector $\mathbf{v}$ has initial point $(-2, 5)$ and terminal point $(3, -1)$. Find the unit vector in the direction of $\mathbf{v}$. Express the answer in component form. \\
        
        \begin{mdframed}
            \begin{equation*}
                \begin{gathered}
                    \text{Given: } \vec{v} = \expval{3 - (-2), -1 - 5} = \expval{5, -6}    \\
                    \text{UV of } \vec{v} = \frac{\vec{v}}{||\vec{v}||}                     \\
                    \frac{\expval{5, -6}}{\sqrt{5^2 + (-6)^2}} 
                    = \frac{\expval{5, -6}}{\sqrt{25 + 36}}
                    = \frac{\expval{5, -6}}{\sqrt{61}} 
                    = \boxed{\expval{\frac{5}{\sqrt{61}}, -\frac{6}{\sqrt{61}}}}
                \end{gathered}
            \end{equation*}
        \end{mdframed}

        \item [14.] The vector $\mathbf{v}$ has initial point $P(1, 1)$ and terminal point $Q$ that is on the x-axis and left of the initial point. Find the coordinates of terminal point $Q$ such that the magnitude of the vector $\mathbf{v}$ is 10. \\
        
        \begin{mdframed}
            \begin{equation*}
                \begin{gathered}
                    \text{Given: } P(1, 1),\; Q(q_{x}, 0), 
                    \text{ and } ||\vec{P, Q}|| = 10        \\
                    \text{Note that: } q_{x} < 1            \\
                    (q_{x} - 1)^2 + (0 - 1)^2 = 10          \\
                    (q_{x} - 1)^2 + 1 = 10                  \\
                    (q_{x} - 1)^2 = 9                       \\
                    q_{x} = \pm \sqrt{9} + 1                \\
                    q_{x} = 1 - 3 = -2                      \\
                    \boxed{Q = (-2, 0)}
                \end{gathered}
            \end{equation*}
        \end{mdframed}

        \pagebreak

        \item [17.] Let a be a standard-position vector with terminal point $(-2, -4)$. Let b be a vector with initial point $(1, 2)$ and terminal point $(-1, 4)$. Find the magnitude of vector $-3a + b - 4\ihat + \jhat$ \\
        
        \begin{mdframed}
            \begin{equation*}
                \begin{gathered}
                    \text{Given: } \vec{a} = \expval{-2, -4}; \; 
                    \vec{b} = \expval{-1 - 1, 4 - 2} = \expval{-2, 2}                   \\
                    -3\expval{-2, -4} + \expval{-2, 2} - 4\expval{1, 0} + \expval{0, 1}
                    = \expval{6, 12} + \expval{-2, 2} - \expval{4, 0} + \expval{0, 1}   \\
                    = \expval{0, 15}; \quad ||\expval{0, 15}|| = \boxed{15}
                \end{gathered}
            \end{equation*}
        \end{mdframed}

        \item [26.]
        \item [30.]
        \item [32.]
        \item [38.]
        \item [40.]
        \item [46.]
    \end{itemize}

    \paragraph*{2.2: 61, 64, 66, 68, 72, 74, 76, 78, 84, 89, 92, 100, 103}
    
    \begin{itemize}
        \item [61.]
        \item [64.]
        \item [66.]
        \item [72.]
        \item [74.]
        \item [76.]
        \item [78.]
        \item [84.]
        \item [89.]
        \item [92.]
        \item [100.]
        \item [103.]
    \end{itemize}

    \paragraph*{2.3: 126, 130, 132, 136, 140, 142, 146, 148, 150, 152, 154, 168, 170, 172}

    \begin{itemize}
        \item [126.]
        \item [130.]
        \item [132.]
        \item [136.]
        \item [140.]
        \item [142.]
        \item [146.]
        \item [148.]
        \item [150.]
        \item [152.]
        \item [154.]
        \item [168.]
        \item [170.]
        \item [172.]
    \end{itemize}
    
\end{document}
