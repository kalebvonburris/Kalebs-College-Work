\documentclass[10pt, letterpaper]{article}

% Files using this must be two subfolders
% deep. Adjust the number of ../ for the
% depth of the file.
% Imports
\usepackage{fancyhdr}
\usepackage{geometry}
\usepackage{icomma}
\usepackage{amsmath}
\usepackage{multicol}
\usepackage{mathptmx}
\usepackage{anyfontsize}
\usepackage{t1enc}
\usepackage{tabto}
\usepackage{listings}
\usepackage{filecontents}
\usepackage{subcaption}
\usepackage{tikz}
\usepackage[parfill]{parskip}
\usepackage{graphicx}
\usepackage[]{mdframed}
\usepackage{amsmath}
\usepackage[makeroom]{cancel}
\usepackage{pgfplots}
\usepackage{pgfplotstable}
\usepackage{xfrac}
\usepackage{amssymb}
\usepackage{mathtools}
\pgfplotsset{compat=1.18}
\usetikzlibrary{patterns}
\usepgfplotslibrary{polar}
\usepgfplotslibrary{fillbetween}

\geometry{margin=2.5cm}

\newcommand{\name}{Kaleb Burris}
\newcommand{\classname}{MATH F253, Elizabeth S. Allman, University of Alaska Fairbanks}
\newcommand{\assignment}{FILL IN ASSIGNMENT NAME}

\pagestyle{fancy}

\fancyhead[L]{
    \name 
    \newline
    \classname
    \newline
    \assignment
}

\newcommand{\horizontal}{\noindent\rule{\hsize}{0.4pt}}

\setlength{\headheight}{42pt}
\setlength{\headsep}{0.25in}
\setlength{\columnsep}{0.35cm}
\setlength{\columnseprule}{1pt}

\usepackage[T1]{fontenc}
\usepackage{lmodern}
\usepackage{tkz-fct}

% Put class number, class name, and professor 
% name.
% Use only in case of emergency, this
% should be covered by the preamble.
% \renewcommand\classname{}

% Put the assignment name with \S if 
% necessary for the section and the question 
% numbers.
\renewcommand\assignment{Homework 1}

\begin{document}
    % Templates
    \iffalse
    % Use these for equations.
    \begin{equation*}
        \begin{gathered}
            Equations go here.
        \end{gathered}
    \end{equation*}

    % Use this if a line of math is too long.
    \resizebox{\hsize}{!}{$Long equation goes here$}

    % Use these for multiple columns.
    \begin{multicol*}{# of columns}
        % Remove the * if you want the columns to be balanced.
    \end{multicol*}

    % Use this to add a horizontal line.
    \horizontal

    \fi

    % Begin homework here.
    %%%%%%%%%%%%%%%%%%%%%%

    \paragraph*{2.1: 4, 7, 9, 12, 14, 17, 26, 30, 32, 38, 40, 46}

    \begin{itemize}
        \item [4.] $\vec{RP}$
        
        \begin{mdframed}
            \begin{equation*}
                \begin{gathered}
                    \text{Given: } R(-3,7) \text{ and } P(-1, 3)                \\
                    \vec{RP} =  \expval{-1 - (-3), 3 - 7} = \boxed{\text{a.} \expval{2, -4}} =
                                \boxed{\text{b.}\, 2 \ihat - 4 \jhat}
                \end{gathered}
            \end{equation*}
        \end{mdframed}
        
        \item [7.] $2\vec{PQ} - 2\vec{PR}$

        \begin{mdframed}
            \begin{equation*}
                \begin{gathered}
                    \text{Given: } R(-3,7), P(-1, 3), \text{ and } Q(1, 5)  \\
                    2\vec{PQ} - 2\vec{PR} = 
                    2\expval{1 - (-1), 5 - 3} - 2\expval{-3 - (-1), 7 - 3}  \\
                    = 2\expval{2, 2} - 2\expval{-2, 4} = 
                    \expval{4, 4} - \expval{-4, 8}  =                       \\
                    = \boxed{\text{a.} \expval{8, -4}} = \boxed{\text{b.}\, 8\ihat - 4\jhat}
                \end{gathered}
            \end{equation*}
        \end{mdframed}

        \item [9.] The unit vector in the direction of $\vec{PQ}$
        
        \begin{mdframed}
            \begin{equation*}
                \begin{gathered}
                    \text{As found: } \vec{PQ} = \expval{2, 2}                  \\
                    \text{UV of } \vec{PQ} = \frac{\expval{2, 2}}{||\vec{PQ}||} \\
                    \frac{\expval{2, 2}}{\sqrt{2^2 + 2^2}} 
                    = \frac{\expval{2, 2}}{2\sqrt{2}}
                    = \expval{\frac{1}{\sqrt{2}}, \frac{1}{\sqrt{2}}}
                    = \expval{\frac{\sqrt{2}}{2}, \frac{\sqrt{2}}{2}}
                    = \boxed{\frac{\sqrt{2}}{2}\ihat + \frac{\sqrt{2}}{2}\jhat}
                \end{gathered}
            \end{equation*}
        \end{mdframed}

        \item [12.] A vector $\mathbf{v}$ has initial point $(-2, 5)$ and terminal point $(3, -1)$. Find the unit vector in the direction of $\mathbf{v}$. Express the answer in component form.
        
        \begin{mdframed}
            \begin{equation*}
                \begin{gathered}
                    \text{Given: } \vec{v} = \expval{3 - (-2), -1 - 5} = \expval{5, -6}    \\
                    \text{UV of } \vec{v} = \frac{\vec{v}}{||\vec{v}||}                     \\
                    \frac{\expval{5, -6}}{\sqrt{5^2 + (-6)^2}} 
                    = \frac{\expval{5, -6}}{\sqrt{25 + 36}}
                    = \frac{\expval{5, -6}}{\sqrt{61}} 
                    = \boxed{\expval{\frac{5}{\sqrt{61}}, -\frac{6}{\sqrt{61}}}}
                \end{gathered}
            \end{equation*}
        \end{mdframed}

        \item [14.] The vector $\mathbf{v}$ has initial point $P(1, 1)$ and terminal point $Q$ that is on the x-axis and left of the initial point. Find the coordinates of terminal point $Q$ such that the magnitude of the vector $\mathbf{v}$ is 10.
        
        \begin{mdframed}
            \begin{equation*}
                \begin{gathered}
                    \text{Given: } P(1, 1),\; Q(q_{x}, 0), 
                    \text{ and } ||\vec{P, Q}|| = 10        \\
                    \text{Note that: } q_{x} < 1            \\
                    (q_{x} - 1)^2 + (0 - 1)^2 = 10          \\
                    (q_{x} - 1)^2 + 1 = 10                  \\
                    (q_{x} - 1)^2 = 9                       \\
                    q_{x} = \pm \sqrt{9} + 1                \\
                    q_{x} = 1 - 3 = -2                      \\
                    \boxed{Q = (-2, 0)}
                \end{gathered}
            \end{equation*}
        \end{mdframed}

        \pagebreak

        \item [17.] Let a be a standard-position vector with terminal point $(-2, -4)$. Let b be a vector with initial point $(1, 2)$ and terminal point $(-1, 4)$. Find the magnitude of vector $-3a + b - 4\ihat + \jhat$. 
        
        \begin{mdframed}
            \begin{equation*}
                \begin{gathered}
                    \text{Given: } \vec{a} = \expval{-2, -4}, \; 
                    \vec{b} = \expval{-1 - 1, 4 - 2} = \expval{-2, 2}                   \\
                    -3\expval{-2, -4} + \expval{-2, 2} - 4\expval{1, 0} + \expval{0, 1}
                    = \expval{6, 12} + \expval{-2, 2} - \expval{4, 0} + \expval{0, 1}
                    = \expval{0, 15} \\
                    ||\expval{0, 15}|| = \boxed{15}
                \end{gathered}
            \end{equation*}
        \end{mdframed}

        \item [25.] $||\mathbf{v}|| = 7, \mathbf{u} = \expval{3, 4} \leftarrow \text{Done accidentally.}$
        
        \begin{mdframed}
            \begin{align*}
                \vec{v} & = 7 * \text{UV of } \vec{u} = 7 *  \frac{\vec{u}}{||\vec{u}||}\\
                \vec{v} & = 7 * \frac{\expval{3,4}}{\sqrt{3^2 + 4^2}}                   
                          = 7 * \expval{\frac{3}{5}, \frac{4}{5}}                       \\
                \vec{v} & = \boxed{\expval{\frac{21}{5}, \frac{28}{5}}}
            \end{align*}
        \end{mdframed}

        \item [26.] $||\mathbf{v}|| = 3, \mathbf{u} = \expval{-2, 5}$
        
        \begin{mdframed}
            \begin{align*}
                \vec{v} & = 3 * \text{UV of } \vec{u} = 3 *  \frac{\vec{u}}{||\vec{u}||}\\
                \vec{v} & = 3 * \frac{\expval{-2,5}}{\sqrt{(-2)^2 + 5^2}} 
                          = 3 * \expval{-\frac{2}{\sqrt{29}}, \frac{5}{\sqrt{29}}}      \\
                \vec{v} & = \boxed{-\expval{\frac{6}{\sqrt{29}}, \frac{15}{\sqrt{29}}}}
            \end{align*}
        \end{mdframed}

        \item [30.] $||\textbf{u}|| = 6, \theta = 60^{\circ}$
        
        \begin{mdframed}
            \begin{align*}
                \vec{u} & = 6 * \expval{\cos(60^{\circ}), \sin(60^{\circ})}     \\
                \vec{u} & = 6 * \expval{\frac{1}{2}, \frac{\sqrt{3}}{2}}
                          = \boxed{\expval{3, 3\sqrt{3}}}
            \end{align*}
        \end{mdframed}

        \item [32.] $||\textbf{u}|| = 8, \theta = \pi$
        
        \begin{mdframed}
            \begin{align*}
                \vec{u} & = 8 * \expval{\cos(\pi), \sin(\pi)}     \\
                \vec{u} & = 8 * \expval{-1, 0}
                          = \boxed{\expval{-8, 0}}
            \end{align*}
        \end{mdframed}

        \item [38.] Consider vectors $a = \expval{2, -4}, b = \expval{-1, 2}$, and $c = 0$. Determine the scalars $\alpha$ and $\beta$ such that $c = \alpha a + \beta b$.
        
        \begin{mdframed}
            \begin{equation*}
                \begin{gathered}
                    \text{Given: } a = \expval{2, -4}, b = \expval{-1, 2}, c = 0    \\
                    c = \alpha * a + \beta * b                                      \\
                    0 = \alpha * \expval{2, -4} + \beta \expval{-1, 2}              \\
         \expval{0,0} = \expval{2\alpha - 1\beta, -4\alpha + 2\beta}                \\
                    2\alpha - \beta = 0 \rightarrow 2\alpha = \beta                 \\
                    -4\alpha + 2\beta = 0 \rightarrow -4\alpha = -2\alpha           \\
                    \therefore \boxed{\beta = 2\alpha, \; \alpha \in \mathbb{R}}
            \end{gathered}
            \end{equation*}
        \end{mdframed}

        \pagebreak

        \item [40.] Consider the function $f(x) = x^4$ where $x \in \mathbb{R}$.
        
        \begin{mdframed}
            \begin{itemize}
                \item [a.]  Determine the real number $z_{0}$ such that point $Q(2, z_{0})$ is situated on the line tangent $t$ to the graph of $f$ at point $P(1,1)$.
                
                \begin{equation*}
                    \begin{gathered}
                        f'(x) = 4x^3                \\
                        t(x) = 4(x-1) + 1 = 4x - 3   \\
                        Q = (2, t(2)) 
                          = (2, 4(2) - 3)           \\
                        Q = (2, 5)                  \\
                        z_{0} = \boxed{5}
                    \end{gathered}
                \end{equation*}
                    
                \begin{center}
                    \begin{tikzpicture}
                        \begin{axis}[
                            axis lines=center,
                            axis equal,
                            xlabel=$x$,
                            ylabel=$y$,
                            restrict x to domain=-5:5,
                            restrict y to domain=-0:10,
                            legend style={anchor=north}
                        ]
                            \addplot[color=red,samples=100]{x^4};
                            \addlegendentry{\(x^4\)}
                            \addplot[color=blue,samples=100]{4*x^3};
                            \addlegendentry{\(4x^3\)}
                            \addplot[mark=*] coordinates {(1,1)} node[pin=0:{$P(1,1)$}]{};
                            \addplot[color=green]{4*x-3};
                            \addlegendentry{\(4x-3\)}
                            \addplot[mark=*] coordinates {(2, 5)} node[pin=0:{$Q(2, z_{0} = 5)$}]{};
                        \end{axis}
                    \end{tikzpicture}
                \end{center}
                
                \item[b.] Determine the unit vector $\mathbf{u}$ with inital point $P$ and terminal point $Q$.
                    
                    \begin{align*}
                        \vec{u} & = \frac{\expval{2 - 1, 5 - 1}}{||\vec{u}||}   \\
                        \vec{u} & = \frac{\expval{1, 4}}{\sqrt{17}} 
                                  = \boxed{\expval{\frac{1}{\sqrt{17}}, \frac{4}{\sqrt{17}}}}
                    \end{align*}
            \end{itemize}
            


        \end{mdframed}

        \item [46.] A baseball player throws a baseball at an angle of $30^{\circ}$ with the horizontal. If the initial speed of the ball is 100 mph, find the horizontal and vertical components of the initial velocity vector of the baseball. (Round to two decimal places.)
        
        \begin{mdframed}
            \begin{equation*}
                \begin{gathered}
                    \text{Given: } v_{0} = 100 \text{mph}, \; \theta = 30^{\circ}   \\
                    \vec{v} = 100\expval{\cos(30^{\circ}), \sin(30^{\circ})}        \\
                    \vec{v} = \expval{50\sqrt{3}, 50} = \boxed{\expval{86.60, 50.00}}
                \end{gathered}
            \end{equation*}
        \end{mdframed}

    \end{itemize}

    \pagebreak

    \paragraph*{2.2: 61, 64, 66, 68, 72, 74, 76, 78, 84, 89, 92, 100, 103}
    
    \begin{itemize}
        \item [61.] Consider a rectangular box with one of the vertices at the origin, as shown in the following figure. If point $A(2, 3, 5)$ is opposite of the origin, then find:
        
        \begin{mdframed}
            \begin{itemize}
                \item [a.] the coordinates of the other size vertices of the box
                
                $\boxed{
                    (0, 3, 5),
                    (0, 0, 5),
                    (2, 0, 5),
                    (2, 0, 0),
                    (2, 3, 0),
                    (0, 3, 0)
                }$

                \item [b.] and the length of the diagonal of the box determined by the vertices $O$ and $A$.
                
                \begin{equation*}
                    \begin{gathered}
                        ||\vec{OA} || = \sqrt{2^2 + 3^2 + 5^2} = \sqrt{4 + 9 + 5}
                                      = \boxed{\sqrt{18}}
                    \end{gathered}
                \end{equation*}
            \end{itemize}
        \end{mdframed}

        \item [64.] $(z-2)(z-5)=0$
        
        \begin{mdframed}
            \begin{equation*}
                (z-2)(z-5)=0 \rightarrow z^2 - 7z + 10 = 0
            \end{equation*}

            This is a parabola with zeroes at $x=2$ and $x=5$, and points $(2, 0), (3, -2), (4, -2) (5, 0)$.
        \end{mdframed}

        \item [66.] $(x-2)^{2}+(z-1)^{2}=1$
        
        \begin{mdframed}
            This is a circle resting flat on the $xz$ axis with a radius of 1, and points $(3, 0, 1), (2, 0, 2), (1, 0, 1), (2, 0, 0)$.
        \end{mdframed}

        \item [72.] Center $C(-4, 7, 2)$ and radius 6.
        
        \begin{mdframed}
            \begin{equation*}
                (x+4)^{2} + (y-7)^{2} + (z-2)^{2} = 36
            \end{equation*}
        \end{mdframed}

        \item [74.] Diameter $PQ$, where $P(-16, -3, 9)$ and $Q(-2, 3, 5)$
        
        \begin{mdframed}
            \begin{equation*}
                \begin{gathered}
                    \text{Given: Diameter (opposite points) } P(-16, -3, 9), Q(-2, 3, 5)    \\
                    C = \frac{(-16 + (-2), -3 + 3, 9 + 5)}{2} = \frac{-18, 0, 14}{2}        \\
                      = (-9, 0, 7)                                                          \\
                    \therefore (x - (-9))^{2} + (y - 0)^{2} + (z - 7)^{2} = ||\vec{PQ}||    \\
                    ||\vec{PQ}|| = \sqrt{
                        \left(\frac{-2 - (-16)}{2}\right)^{2} +
                        \left(\frac{3 - (-3)}{2}\right)^{2} +
                        \left(\frac{5 - 9}{2}\right)^{2}
                    }   \\
                    = \sqrt{7^2 + 3^2 + 2^2} = \sqrt{49 + 9 + 4} = \sqrt{62}                \\
                    \boxed{(x + 9)^{2} + y^{2} + (z - 7)^{2} = \sqrt{62}}
                \end{gathered}
            \end{equation*}
        \end{mdframed}

        \item [76.] $x^2 + y^2 + z^2 - 6x + 8y - 10z + 25 = 0$
        
        \begin{mdframed}
            \begin{equation*}
                \begin{gathered}
                    x(x-6)+y(y+8)+z(z-10) = 25          \\
                    (x-3)^{2}+(y+4)^{2}+(z-5)^{2} = 25  \\
                    \therefore \boxed{C(3, -4, 5) \text{ radius: } 5}
                \end{gathered}
            \end{equation*}
        \end{mdframed}

        \pagebreak

        \item [78.] $P(0, 10, 5)$ and $Q(1, 1, -3)$
        
        \begin{mdframed}
            \begin{equation*}
                \begin{gathered}
                    \vec{PQ} = \expval{1 - 0, 1 - 10, -3 - 5} = \expval{1, -9, -8}
                \end{gathered}
            \end{equation*}
            \begin{itemize}
                \item [a.] \boxed{\expval{1, -9, -8}}
                \item [b.] \boxed{\ihat, -9\jhat, -8\khat}
            \end{itemize}
        \end{mdframed}

        \item [84.] $\mathbf{a} = \expval{-1, -2, 4}, \mathbf{b} = \expval{-5, 6, 7}$
        
        \begin{mdframed}
            \begin{equation*}
                \begin{gathered}
                    \vec{a} + \vec{b} = \expval{-1, -2, 4} + \expval{-5, 6, 7} = \boxed{\expval{-6, 4, 11}} \\
                    4\vec{a} = 4\expval{-1, -2, 4} = \boxed{\expval{-4, -8, 16}}                            \\
                    -5\vec{a} + 3\vec{b} = -5\expval{-1, -2, 4}+3\expval{-5, 6, 7} = \boxed{\expval{-10, 16, 1}}
                \end{gathered}
            \end{equation*}
        \end{mdframed}

        \item [87.] $\mathbf{u} = 2i + 3j + 4k, \mathbf{v} = -i + 5j + k$ Whoops, did it again.
        
        \begin{mdframed}
            \begin{equation*}
                \begin{gathered}
                    ||\vec{u} - \vec{v}|| = ||3i - 2j + 3k|| = \sqrt{3^2 + (-2)^2 + 3^2} = \boxed{\sqrt{22}} \\
                    ||-2\vec{u}|| = \sqrt{(-4)^2 + (-6)^2 + 8^2} = \boxed{\sqrt{116}}
                \end{gathered}
            \end{equation*}
        \end{mdframed}

        \item[89.] $\mathbf{u} = \expval{2\cos t, -2\sin t, 3}, \mathbf{v} = \expval{0, 0, 3}$, where $t \in \mathbb{R}$.
        
        \begin{mdframed}
            \begin{equation*}
                \begin{gathered}
                    ||\vec{u} - \vec{v}|| = ||\expval{2\cos t - 0, -2\sin t - 0, 3 - 3}|| 
                        = \sqrt{4\cos^{2}t + 4\sin^{2} t + 0^2} = 2\sqrt{\sin^{2}t + \cos^{2}t} = \boxed{2} \\
                    ||-2\vec{u}|| = ||\expval{-4\cos t, 4\sin t, -6}|| 
                        = \sqrt{16\cos^{2}t + 16\sin^{2}t + 36} = \sqrt{16 + 36} = \sqrt{56} = \boxed{2\sqrt{13}}
                \end{gathered}
            \end{equation*}
        \end{mdframed}

        \item [92.] $\mathbf{a} = \expval{4, -3, 6}$
        
        \begin{mdframed}
            \begin{equation*}
                \begin{gathered}
                    \text{UV of } \vec{a} = \frac{\vec{a}}{||\vec{a}||} \\
                    \frac{\vec{a}}{||\vec{a}||} = \frac{\expval{4, -3, 6}}{\sqrt{16 + 9 + 36}}
                    = \boxed{\expval{\frac{4}{\sqrt{61}}, -\frac{3}{\sqrt{61}}, \frac{6}{\sqrt{61}}}}
                \end{gathered}
            \end{equation*}
        \end{mdframed}

        \item [100.] $\mathbf{v} = \expval{2, 4, 1}, \; ||\mathbf{u}|| = 15$, $\mathbf{u}$ and $\mathbf{v}$ have the same direction.
        
        \begin{mdframed}
            \begin{equation*}
                \begin{gathered}
                    \vec{u} = 15\frac{\vec{v}}{||\vec{v}||} \\
                    \vec{u} = 15\frac{\expval{2, 4, 1}}{\sqrt{2^2, 4^2, 1^1}} 
                        = 15\frac{\expval{2, 4, 1}}{\sqrt{21}} 
                        = \boxed{\expval{\frac{30}{\sqrt{21}}, \frac{60}{\sqrt{21}}, \frac{15}{\sqrt{21}}}}
                \end{gathered}
            \end{equation*}
        \end{mdframed}

        \item [103.] Determine a vector of magnitude 5 in the direction of vector $\vec{AB}$, where $A(2, 1, 5)$ and $B(3, 4, -7)$.
        
        \begin{mdframed}
            \begin{equation*}
                \begin{gathered}
                    \vec{AB} = \expval{3 - 2, 4 - 1, -7 - 5} = \expval{1, 3, -12}   \\
                    5\frac{\vec{AB}}{||\vec{AB}||}
                        = 5\frac{\expval{1, 3, -12}}{\sqrt{1^2 + 3^2 + (-12)^2}}
                        = 5\frac{\expval{1, 3, -12}}{\sqrt{154}}
                        = \boxed{\expval{\frac{5}{\sqrt{154}}, \frac{15}{\sqrt{154}}, -\frac{60}{\sqrt{154}}}}
                \end{gathered}
            \end{equation*}
        \end{mdframed}

    \end{itemize}

    \pagebreak

    \paragraph*{2.3: 126, 130, 132, 136, 140, 142, 146, 148, 150, 152, 154, 168, 170, 172}

    \begin{itemize}
        \item [126.] $\mathbf{u} = \expval{4, 5, -6}, \mathbf{v} = \expval{0, -2, -3}$
        
        \begin{mdframed}
            \begin{equation*}
                \begin{gathered}
                    \vec{u} \cdot \vec{v} = (4 * 0) + (5 * -2) + (-6 * -3) = \boxed{8}
                \end{gathered}
            \end{equation*}
        \end{mdframed}

        \item [130.] $a = i - j + k, b = j + 3k, c = -i + 2j - 4\khat$
        
        \begin{mdframed}
            \begin{equation*}
                \begin{gathered}
                    (a \cdot b)c = ((1 * 0) + (-1 * 1) + (1 * 3))(-\ihat + 2\jhat - 4\khat) 
                    = 2(-\ihat + 2\jhat - 4\khat) = \boxed{\expval{-2, 4, -8}}
                \end{gathered}
            \end{equation*}
        \end{mdframed}

        \item [132.] $\mathbf{a} = \expval{2, 1}, \mathbf{b} = \expval{-1, 3}$
        
        \begin{mdframed}
            \begin{itemize}
                \item [a.]
                \begin{equation*}
                    \begin{gathered}
                            \theta = \arccos\left(\frac{a \cdot b}{||a||*||b||}\right)  \\
                                   = \arccos\left(\frac{2}{\sqrt{5} * \sqrt{10}}\right)
                                   = \arccos\left(\frac{1}{5\sqrt{2}}\right) = \boxed{1.28 \text{ radians}}
                    \end{gathered}
                \end{equation*}
                \item [b.] \boxed{1.28 * 2 < \pi; acute}
            \end{itemize}
        \end{mdframed}

        \item [136.] $\mathbf{a} = \expval{0,-1,-3}, \mathbf{b} = \expval{2,3,-1}$
        
        \begin{mdframed}
            \begin{equation*}
                \begin{gathered}
                    \theta = \arccos\left(\frac{a \cdot b}{||a||*||b||}\right)  \\
                    \arccos\left(\frac{(0 * 2) + (-1 * 3) + (-3 * -1)}
                        {\sqrt{10}*\sqrt{14}}\right)
                        = \arccos\left(\frac{0}
                        {\sqrt{10}*\sqrt{14}}\right) = \arccos(0) = \boxed{\frac{\pi}{2}} 
                \end{gathered}
            \end{equation*}
        \end{mdframed}

        \item [140.] 
        \item [142.]
        \item [146.]
        \item [148.]
        \item [150.]
        \item [152.]
        \item [154.]
        \item [168.]
        \item [170.]
        \item [172.]
    \end{itemize}
    
\end{document}
