\documentclass[10pt, letterpaper]{article}

% Files using this must be two subfolders
% deep. Adjust the number of ../ for the
% depth of the file.
% Imports
\usepackage{fancyhdr}
\usepackage{geometry}
\usepackage{icomma}
\usepackage{amsmath}
\usepackage{multicol}
\usepackage{mathptmx}
\usepackage{anyfontsize}
\usepackage{t1enc}
\usepackage{tabto}
\usepackage{listings}
\usepackage{filecontents}
\usepackage{subcaption}
\usepackage{tikz}
\usepackage[parfill]{parskip}
\usepackage{graphicx}
\usepackage[]{mdframed}
\usepackage{amsmath}
\usepackage[makeroom]{cancel}
\usepackage{pgfplots}
\usepackage{pgfplotstable}
\usepackage{xfrac}
\usepackage{amssymb}
\usepackage{mathtools}
\pgfplotsset{compat=1.18}
\usetikzlibrary{patterns}
\usepgfplotslibrary{polar}
\usepgfplotslibrary{fillbetween}

\geometry{margin=2.5cm}

\newcommand{\name}{Kaleb Burris}
\newcommand{\classname}{MATH F253, Elizabeth S. Allman, University of Alaska Fairbanks}
\newcommand{\assignment}{FILL IN ASSIGNMENT NAME}

\pagestyle{fancy}

\fancyhead[L]{
    \name 
    \newline
    \classname
    \newline
    \assignment
}

\newcommand{\horizontal}{\noindent\rule{\hsize}{0.4pt}}

\setlength{\headheight}{42pt}
\setlength{\headsep}{0.25in}
\setlength{\columnsep}{0.35cm}
\setlength{\columnseprule}{1pt}

\usepackage[T1]{fontenc}
\usepackage{lmodern}
\usepackage{tkz-fct}

% Put class number, class name, and professor 
% name.
% Use only in case of emergency, this
% should be covered by the preamble.
% \renewcommand\classname{}

% Put the assignment name with \S if 
% necessary for the section and the question 
% numbers.
\renewcommand\assignment{Homework 3}

\begin{document}
    % Templates
    \iffalse
    % Use these for equations.
    \begin{equation*}
        \begin{gathered}
            Equations go here.
        \end{gathered}
    \end{equation*}

    % Use this if a line of math is too long.
    \resizebox{\hsize}{!}{$Long equation goes here$}

    % Use these for multiple columns.
    \begin{multicol*}{# of columns}
        % Remove the * if you want the columns to be balanced.
    \end{multicol*}

    % Use this to add a horizontal line.
    \horizontal

    \fi

    % Begin homework here.
    %%%%%%%%%%%%%%%%%%%%%%

    \paragraph*{2.5:   268, 271, 274, 280, 286, 288, 299}

    \begin{enumerate}
        \item [268.]\mbox{}
        \begin{mdframed}
            \begin{itemize}
                \item [a.]
                
                    \begin{equation*}
                        \begin{gathered}
                            \text{Given: } P(3,2,2), \quad \vec{n} = \expval{2,3,-1} \\
                            \boxed{2(x-3) + 3(y-2) - (z-2) = 0} \\
                        \end{gathered}
                    \end{equation*}

                \item [b.]
                
                    \begin{equation*}
                        \begin{gathered}
                            \boxed{2x + 3y - z - 10 = 0}
                        \end{gathered}
                    \end{equation*}
            \end{itemize}
        \end{mdframed}

        \item [271.]\mbox{}
        \begin{mdframed}
            \begin{itemize}
                \item [a.]
                \begin{equation*}
                    \begin{gathered}
                        \vec{n} = \expval{4,5,10} = \boxed{4\ihat + 5\ihat + 10\khat}
                    \end{gathered}
                \end{equation*}
                \item [b.]
                \begin{equation*}
                    \begin{gathered}
                        \text{$x$-axis: } 4x - 20 = 0 \rightarrow \boxed{x = 5} \\
                        \text{$y$-axis: } 5y - 20 = 0 \rightarrow \boxed{y = 4} \\
                        \text{$z$-axis: } 10z - 20 = 0 \rightarrow \boxed{z = 2}
                    \end{gathered}
                \end{equation*}      
            \end{itemize}
        \end{mdframed}

        \item [274.]\mbox{}
        \begin{mdframed}
            \begin{itemize}
                \item [a.]
                \begin{equation*}
                    \begin{gathered}
                        \vec{n} = \expval{1,0,1} = \boxed{1\ihat + 1\khat}
                    \end{gathered}
                \end{equation*}
                \item [b.]
                \begin{equation*}
                    \begin{gathered}
                        \text{$x$-axis: } \boxed{x = 0}          \\
                        \text{$y$-axis: } \boxed{y = \mathbb{R}} \\
                        \text{$z$-axis: } \boxed{z = 0}
                    \end{gathered}
                \end{equation*}      
            \end{itemize}
        \end{mdframed}

        \item [280.]\mbox{}
        \begin{mdframed}
            \begin{equation*}
                \begin{gathered}
                    \vec{t} = \expval{1,-1,1}, \vec{n} = \expval{\alpha, 5, 1}:
                    \quad \text{Find $\alpha$ such that } \vec{t} \cdot \vec{n} = 0 \\
                    \vec{t} \cdot \vec{n} = \alpha - 5 + 1 \rightarrow \boxed{\alpha = 4}
                \end{gathered}
            \end{equation*}
        \end{mdframed}

        \item [286.]\mbox{}
        \begin{mdframed}
            
        \end{mdframed}
    \end{enumerate}

    \pagebreak

    \paragraph*{3.1:   2, 3, 6, 9, 13, 14, 16, 23, 29, 37}

    \pagebreak

    \paragraph*{3.2:  43, 47, 53, 54, 63, 64, 67, 68, 70, 71, 72, 77, 81, 86, 95, 96, 101}

\end{document}