\providecommand\pointsize{12pt}

\documentclass[\pointsize, letterpaper]{article}

% Imports
\usepackage{fancyhdr}
\usepackage{geometry}
\usepackage{icomma}
\usepackage{amsmath}
\usepackage{multicol}
\usepackage{mathptmx}
\usepackage{anyfontsize}
\usepackage{t1enc}
\usepackage{tabto}
\usepackage{listings}
\usepackage{filecontents}
\usepackage{subcaption}
\usepackage{tikz}
\usepackage[parfill]{parskip}
\usepackage{graphicx}
\usepackage[]{mdframed}
\usepackage{amsmath}
\usepackage[makeroom]{cancel}
\usepackage{pgfplots}
\usepackage{pgfplotstable}
\usepackage{xfrac}
\usepackage{amssymb}
\usepackage{tkz-fct}
\usetikzlibrary{patterns}
\usepgfplotslibrary{fillbetween}
\pgfplotsset{compat=1.18}

\geometry{margin=2.5cm}

\newcommand{\name}{Kaleb Burris}
\newcommand{\classname}{MATH F252, Dr. J. Gimbel, University of Alaska Fairbanks}

\pagestyle{fancy}

\fancyhead[L]{
    \name 
    \newline
    \classname
}

\newcommand{\horizontal}{\noindent\rule{\hsize}{0.4pt}}

\setlength{\headheight}{42pt}
\setlength{\headsep}{0.25in}
\setlength{\columnsep}{0.35cm}
\setlength{\columnseprule}{1pt}

\usepackage{pgfplots}
\usepackage{pgfplotstable}
\makeatletter
\long\def\ifnodedefined#1#2#3{%
    \@ifundefined{pgf@sh@ns@#1}{#3}{#2}%
}

\pgfplotsset{
    discontinuous/.style={
    scatter,
    scatter/@pre marker code/.code={
        \ifnodedefined{marker}{
            \pgfpointdiff{\pgfpointanchor{marker}{center}}%
             {\pgfpoint{0}{0}}%
             \ifdim\pgf@y>0pt
                \tikzset{options/.style={mark=*, fill=white}}
                \draw [densely dashed,blue] (marker-|0,0) -- (0,0);
                \draw plot [mark=*] coordinates {(marker-|0,0)};
             \else
                \tikzset{options/.style={mark=none}}
             \fi
        }{
            \tikzset{options/.style={mark=none}}        
        }
        \coordinate (marker) at (0,0);
        \begin{scope}[options]
    },
    scatter/@post marker code/.code={\end{scope}}
    }
}
\makeatother

\usepackage[T1]{fontenc}
\usepackage{lmodern}

\begin{document}
    \paragraph*{Examples:}
    \begin{itemize}
        \item $x = t^{2} + 1, \quad y = 2t^{2}, \quad x \geq 1$
        
        Here we can express x as a function of $t$:
        \begin{align*}
            x-1 & = t^{2}                   \\
            t   & = \pm\sqrt{x-1}           \\
            y   & = 2(\pm\sqrt{x-1})^{2}    \\
                & = 2(x-1)                  \\
                & = 2x-2                    \\
            \text{Note that: } & x \geq 1, 
            \text{ and if } t = 0, \, x = 1 \\
            x   & = \cos t                  \\
            y   & = \sin t                  \\
            x^{2}   & = \cos^{2} t          \\
            y^{2}   & = \sin^{2} t          \\
            x^{2} + y^{2} & = 1             \\
        \end{align*}

        \item Suppose:
        \begin{align*}
            x   & = a + bt          \\
            y   & = c + dt          \\
            t   & = \frac{x-a}{b}   \\
            \text{Then:} &          \\
            y   & = c + 
                    d\frac{x-a}{b}  \\
                & = c + 
                    \frac{d}{b}x - 
                    \frac{da}{t}    \\
                & = \frac{d}{b}x + 
                    \left(
                        c - \frac{da}{t}
                    \right)         \\
            x   & = a(t-\sin t)     \\
            y   & = a(1 - \cos t)   \\
        \end{align*}
    \end{itemize}

    \pagebreak

    \subsection*{The Calculus of parameterized curves}

    Remember that:

    \begin{equation}
        \frac{dy}{dx} = \frac{dy/dt}{dx/dt}
    \end{equation}

    Therefore, for the derivative of a circle:

    \begin{equation}
        \begin{gathered}
            x = \cos t  \\
            y = \sin t  \\
            \frac{dy}{dx} = 
            \frac{\cos t}{-\sin t} = 
            -\cot t
        \end{gathered}
    \end{equation}

    At different values of t:

    \begin{alignat*}{3}
        t = \pi/2:  & \dfrac{\cos(\pi/2)}{-\sin(\pi/2)} 
                    && = \dfrac{0}{-1} 
                    && = 0                   \\
        t = 0:      & \dfrac{\cos(0)}{-\sin(0)} 
                    && = \dfrac{1}{0} 
                    && = \text{Undefined}    \\
        t = \pi:    & \dfrac{\cos(\pi)}{-\sin(\pi)} 
                    && = \dfrac{-1}{0} 
                    && = \text{Undefined}    \\
    \end{alignat*}

    

\end{document}