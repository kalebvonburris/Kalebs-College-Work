% Files using this must be two subfolders
% deep. Adjust the number of ../ for the
% depth of the file.
% Imports
\usepackage{fancyhdr}
\usepackage{geometry}
\usepackage{icomma}
\usepackage{amsmath}
\usepackage{multicol}
\usepackage{mathptmx}
\usepackage{anyfontsize}
\usepackage{t1enc}
\usepackage{tabto}
\usepackage{listings}
\usepackage{filecontents}
\usepackage{subcaption}
\usepackage{tikz}
\usepackage[parfill]{parskip}
\usepackage{graphicx}
\usepackage[]{mdframed}
\usepackage{amsmath}
\usepackage[makeroom]{cancel}
\usepackage{pgfplots}
\usepackage{pgfplotstable}
\usepackage{xfrac}
\usepackage{amssymb}
\usepackage{mathtools}
\pgfplotsset{compat=1.18}
\usetikzlibrary{patterns}
\usepgfplotslibrary{polar}
\usepgfplotslibrary{fillbetween}

\geometry{margin=2.5cm}

\newcommand{\name}{Kaleb Burris}
\newcommand{\classname}{MATH F253, Elizabeth S. Allman, University of Alaska Fairbanks}
\newcommand{\assignment}{FILL IN ASSIGNMENT NAME}

\pagestyle{fancy}

\fancyhead[L]{
    \name 
    \newline
    \classname
    \newline
    \assignment
}

\newcommand{\horizontal}{\noindent\rule{\hsize}{0.4pt}}

\setlength{\headheight}{42pt}
\setlength{\headsep}{0.25in}
\setlength{\columnsep}{0.35cm}
\setlength{\columnseprule}{1pt}

\usepackage[T1]{fontenc}
\usepackage{lmodern}

% Put class number, class name, and professor 
% name.
% Use only in case of emergency, this
% should be covered by the preamble.
% \renewcommand\classname{}

% Put the assignment name with \S if 
% necessary for the section and the question 
% numbers.
\renewcommand\assignment{Homework Set I, Due Monday, January 23rd, 2023}

\begin{document}

    % Templates
    \iffalse
    % Use these for equations.
    \begin{equation*}
        \begin{gathered}
            Equations go here.
        \end{gathered}
    \end{equation*}

    % Use this if a line of math is too long.
    \resizebox{\hsize}{!}{$Long equation goes here$}

    % Use these for multiple columns.
    \begin{multicol*}{# of columns}
        % Remove the * if you want the columns to be balanced.
    \end{multicol*}

    % Use this to add a horizontal line.
    \horizontal

    \fi

    % Begin homework here.
    %%%%%%%%%%%%%%%%%%%%%%

    \paragraph*{1.}
    Suppose $f(x) = x^2-x$ what is the average value of $f$ on [-1,1]?
    
    \begin{mdframed}
        \begin{align*}
                \overline{f} & = \frac{1}{b-a} \int_{a}^{b}f(x)\mathrm{d}x \quad \text{where} \quad f(x) = x^2-x   \\
                \Rightarrow\overline{f} & = \frac{1}{1-(-1)} \int_{-1}^{1}(x^2-x)\mathrm{d}x = \frac{1}{2}\left.\left(\frac{x^3}{3}-\frac{x^2}{2}\right)\right|_{-1}^{1}   \\
                & = \left.\left(\frac{2x^3-3x^2}{12}\right)\right|_{-1}^{1} = \left(-\frac{1}{12}\right) - \left(-\frac{5}{12}\right) = \frac{4}{12} = \boxed{\frac{1}{3} \approx 0.\overline{333}}
        \end{align*}
    \end{mdframed}

    \paragraph*{2.}
    Evaluate $2+4+6+\cdots+100$.

    \begin{mdframed}
        \begin{equation*}
            \begin{gathered}
                2+4+6+\cdots+100 = \sum_{n=1}^{50}2n = 2\sum_{n=1}^{50}n    \\
                \text{Using} \sum_{n=1}^{k}n = \frac{k(k+1)}{2}; \quad \Rightarrow \quad \cancel{2}\left(\frac{50(51)}{\cancel{2}}\right) = \boxed{2,550}
            \end{gathered}
        \end{equation*}
    \end{mdframed}

    \paragraph*{3.}
    Evaluate $\displaystyle\int(7+e^x)^5 e^x \mathrm{d}x$

    \begin{mdframed}
        \begin{equation*}
            \begin{gathered}
                \int(7+e^x)^5 e^x \mathrm{d}x                                   \\
                \text{let} \quad u = 7 + e^x \quad \mathrm{d}u = e^x\mathrm{d}x \\
                \Rightarrow\int u^5\mathrm{d}u = \frac{1}{6}u^6 + C = \boxed{\frac{1}{6}(7 + e^x)^6 + C}
            \end{gathered}
        \end{equation*}
    \end{mdframed}

    \paragraph*{4.}
    Evaluate $\displaystyle\int_{-1}^{1}(3+x)^4 \mathrm{d}x$

    \begin{mdframed}
        \begin{equation*}
            \begin{gathered}
                \int_{-1}^{1}(3+x)^4 \mathrm{d}x                                \\
                \text{let} \quad u = 3 + x; \quad \mathrm{d}u = \mathrm{d}x     \\
                u_{lower} = 3 + (-1) = 2; \quad u_{upper} = 3 + 1 = 4                  \\
                \Rightarrow\int_{-1}^{1}u^4 \mathrm{d}u = \left.\frac{1}{5}u^5 \right|_{2}^{4}= \frac{4^5}{5} - \frac{2^5}{5} = \boxed{198.4}
            \end{gathered}
        \end{equation*}
    \end{mdframed}
\end{document}