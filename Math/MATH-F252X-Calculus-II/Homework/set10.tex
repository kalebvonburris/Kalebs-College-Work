% Files using this must be two subfolders
% deep. Adjust the number of ../ for the
% depth of the file.
% Imports
\usepackage{fancyhdr}
\usepackage{geometry}
\usepackage{icomma}
\usepackage{amsmath}
\usepackage{multicol}
\usepackage{mathptmx}
\usepackage{anyfontsize}
\usepackage{t1enc}
\usepackage{tabto}
\usepackage{listings}
\usepackage{filecontents}
\usepackage{subcaption}
\usepackage{tikz}
\usepackage[parfill]{parskip}
\usepackage{graphicx}
\usepackage[]{mdframed}
\usepackage{amsmath}
\usepackage[makeroom]{cancel}
\usepackage{pgfplots}
\usepackage{pgfplotstable}
\usepackage{xfrac}
\usepackage{amssymb}
\usepackage{mathtools}
\pgfplotsset{compat=1.18}
\usetikzlibrary{patterns}
\usepgfplotslibrary{polar}
\usepgfplotslibrary{fillbetween}

\geometry{margin=2.5cm}

\newcommand{\name}{Kaleb Burris}
\newcommand{\classname}{MATH F253, Elizabeth S. Allman, University of Alaska Fairbanks}
\newcommand{\assignment}{FILL IN ASSIGNMENT NAME}

\pagestyle{fancy}

\fancyhead[L]{
    \name 
    \newline
    \classname
    \newline
    \assignment
}

\newcommand{\horizontal}{\noindent\rule{\hsize}{0.4pt}}

\setlength{\headheight}{42pt}
\setlength{\headsep}{0.25in}
\setlength{\columnsep}{0.35cm}
\setlength{\columnseprule}{1pt}

\usepackage[T1]{fontenc}
\usepackage{lmodern}

% Put class number, class name, and professor 
% name.
% Use only in case of emergency, this
% should be covered by the preamble.
% \renewcommand\classname{}

% Put the assignment name with \S if 
% necessary for the section and the question 
% numbers.
\renewcommand\assignment{Homework Set X, Due Wednesday, April 5, 2023}

\begin{document}
    % Templates
    \iffalse
    % Use these for equations.
    \begin{equation*}
        \begin{gathered}
            Equations go here.
        \end{gathered}
    \end{equation*}

    % Use this if a line of math is too long.
    \resizebox{\hsize}{!}{$Long equation goes here$}

    % Use these for multiple columns.
    \begin{multicol*}{# of columns}
        % Remove the * if you want the columns to be balanced.
    \end{multicol*}

    % Use this to add a horizontal line.
    \horizontal

    \fi

    % Begin homework here.
    %%%%%%%%%%%%%%%%%%%%%%

    Find the interval and radius of convergence of:

    \begin{itemize}
        \item [1.] $\sum\frac{kx^{k}}{5^{k}}$
        \\
        \begin{mdframed}
            \begin{equation*}
                \begin{gathered}
                    \sqrt[k]{\left|\frac{kx^{k}}{5^{k}}\right|} =
                \frac{\sqrt[k]{|kx^{k}|}}{\sqrt[k]{5^{k}}} =
                \frac{|x|}{5}                                                   \\
                \frac{|x|}{5} < 1 \Rightarrow |x| < 5 \Rightarrow -5 < x < 5    \\
                \text{Note at } x = 5: \sum\frac{k5^{k}}{5^{k}} = \sum k \rightarrow \infty, \quad
                \therefore \sum\frac{k5^{k}}{5^{k}} \text{ is Divergent by Divergence test}        
                \\
                \text{Note at } x = -5: \sum\frac{-k5^{k}}{5^{k}} = -\sum k \rightarrow -\infty, \quad
                \therefore \sum\frac{-k5^{k}}{5^{k}} \text{ is Divergent by Divergence test}        
                \\
                \text{Radius} = 5 \quad \text{Interval} = \left(-5,5\right)
                \end{gathered}
            \end{equation*}
        \end{mdframed}

        \item [2.] $\sum\frac{(x-9)^{k}}{k6^{k}}$
        \\
        \begin{mdframed}
            \begin{equation*}
                \begin{gathered}
                    \sqrt[k]{\left|\frac{(x-9)^{k}}{k6^{k}}\right|} =
                    \frac{\sqrt[k]{|(x-9)^{k}|}}{\sqrt[k]{k6^{k}}} =
                    \frac{|x-9|}{6}                                         \\
                    \frac{|x-9|}{6} < 1 \Rightarrow |x-9| < 6 \Rightarrow 
                    -6 < x-9 < 6 \Rightarrow 3 < x < 15                     \\
                    \text{Note at } x=3: \sum\frac{(3-9)^{k}}{k6^{k}} =
                    \sum\frac{(-6)^{k}}{k6^{k}} = 
                    \sum\frac{(-1)^{k}6^{k}}{k6^{k}} =
                    \sum\frac{(-1)^{k}}{k} \rightarrow 
                    \text{Converges}                                        \\
                    \text{Note at } x=15: \sum\frac{(15-9)^{k}}{k6^{k}} =
                    \sum\frac{6^{k}}{k6^{k}} = 
                    \sum\frac{1}{k} \rightarrow \text{ Diverges by P-Series}\\
                    \text{Radius} = 6 \quad \text{Interval} = \left[3,15\right)
                \end{gathered}
            \end{equation*}
        \end{mdframed}

        \item [3.] $\sum\frac{(x-1)^{k}}{k^{2}7^{k}}$
        \\
        \begin{mdframed}
            \begin{equation*}
                \begin{gathered}
                    \left|\frac{\frac{(x-1)^{k+1}}{(k+1)^{2}7^{k+1}}}
                            {\frac{(x-1)^{k}}{k^{2}7^{k}}}\right| =
                            \frac{|(x-1)^{k+1}|}{|(x-1)^{k}|}\cdot
                            \frac{(k+1)^{2}}{k^{2}}\cdot
                            \frac{7^{k+1}}{7^{k}} =
                            |(x-1)|\cdot\frac{k^{2}+2k+1}{k^{2}}\cdot 7     \\
                            \rightarrow (x-1) \cdot 1 \cdot 7 = 7|(x-1)|    \\
                            7|(x-1)| < 1 \Rightarrow |x-1| < 7 \Rightarrow
                            -7 < x - 1 < 7 \Rightarrow -6 < x < 8           \\
                            \text{Note at } x = -6: \sum\frac{(-6-1)^{k}}{k^{2}7^{k}} =
                            \sum\frac{(-7)^{k}}{k^{2}7^{k}} = 
                            \sum\frac{(-1)^{k}7^{k}}{k^{2}7^{k}} =
                            \frac{(-1)^{k}}{k^{2}} \rightarrow \text{ Converges by AS}  \\
                            \text{Note at } x = 8: \sum\frac{(8-1)^{k}}{k^{2}7^{k}} =
                            \sum\frac{7^{k}}{k^{2}7^{k}} = 
                            \frac{1}{k^{2}} \rightarrow \text{ Converges by P-Series}   \\
                            \text{Radius} = 7 \quad \text{Interval} = \left[-6,8\right]
                \end{gathered}
            \end{equation*}
        \end{mdframed}

        \item[4.] $\sum\frac{k^{2}(x-3)^{k}}{k!}$
        \\
        \begin{mdframed}
            \begin{equation*}
                \begin{gathered}
                    \left|\frac {\frac{(k+1)^{2}(x-3)^{k+1}}{(k+1)!}}
                                {\frac{k^{2}(x-3)^{k}}{k!}}\right| =
                            \frac{(k+1)^{2}}{k^{2}}\cdot
                            \left|\frac{(x-3)^{k+1}}{(x-3)^{k}}\right|\cdot
                            \frac{k!}{(k+1)!} =
                            \left(\frac{k+1}{k}\right)^{2}\cdot
                            |x-3|\cdot
                            \frac{1}{k+1} \rightarrow
                            1^{2} \cdot |x-3| \cdot 0   \\
                            0 < 1, \quad 
                            \therefore \text{ Radius} = \infty \quad
                            \text{Interval} = \left(-\infty, \infty\right)
                \end{gathered}
            \end{equation*}
        \end{mdframed}
    \end{itemize}
\end{document}