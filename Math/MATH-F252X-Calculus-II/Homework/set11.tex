% Files using this must be two subfolders
% deep. Adjust the number of ../ for the
% depth of the file.
\providecommand\pointsize{10pt}

\documentclass[\pointsize, letterpaper]{article}

% Imports
\usepackage{fancyhdr}
\usepackage{pgfplots}
\usepackage{geometry}
\usepackage{icomma}
\usepackage{amsmath}
\usepackage{multicol}
\usepackage{mathptmx}
\usepackage{anyfontsize}
\usepackage{t1enc}
\usepackage{tabto}
\usepackage{listings}
\usepackage{filecontents}
\usepackage{subcaption}
\usepackage{tikz}
\usepackage[parfill]{parskip}
\usepackage{graphicx}
\usepackage[]{mdframed}
\usepackage{amsmath}
\usepackage[makeroom]{cancel}
\pgfplotsset{compat=1.18}

\geometry{margin=2.5cm}

\newcommand{\name}{Kaleb Burris}
\newcommand{\classname}{MATH F252, Dr. J. Gimbel}
\newcommand{\assignment}{FILL IN ASSIGNMENT NAME}

\pagestyle{fancy}

\fancyhead[L]{
    \name 
    \newline
    \classname
    \newline
    \assignment
}

\newcommand{\horizontal}{\noindent\rule{\hsize}{0.4pt}}

\setlength{\headheight}{42pt}
\setlength{\headsep}{0.25in}
\setlength{\columnsep}{0.35cm}
\setlength{\columnseprule}{1pt}

% Put class number, class name, and professor 
% name.
% Use only in case of emergency, this
% should be covered by the preamble.
% \renewcommand\classname{}

% Put the assignment name with \S if 
% necessary for the section and the question 
% numbers.
\renewcommand\assignment{Homework Set XI, Due Thursday, April 13, 2023}

\begin{document}
    % Templates
    \iffalse
    % Use these for equations.
    \begin{equation*}
        \begin{gathered}
            Equations go here.
        \end{gathered}
    \end{equation*}

    % Use this if a line of math is too long.
    \resizebox{\hsize}{!}{$Long equation goes here$}

    % Use these for multiple columns.
    \begin{multicol*}{# of columns}
        % Remove the * if you want the columns to be balanced.
    \end{multicol*}

    % Use this to add a horizontal line.
    \horizontal

    \fi

    % Begin homework here.
    %%%%%%%%%%%%%%%%%%%%%%

    \begin{itemize}
        \item [174.] Use appropriate substitutions to write down the Maclaurin series for the given binomial: $(1-x)^{1/3}$.
        \\
        \begin{mdframed}
            \begin{equation*}
                \begin{gathered}
                    \text{Note that: } (1-x)^{r} = 
                    \sum_{k=0}^{\infty}{r\choose k}x^{k}      \\
                    (1-x)^{1/3} = \sum_{k=0}^{\infty}{1/3 \choose k}x^{k} =
                    \boxed{\sum_{k=0}^{\infty}\frac{(1/3)_{k}}{k!}x^{k}}
                \end{gathered}
            \end{equation*}
        \end{mdframed}

        \item [178.] Find the Taylor series of each function with the given center: $\sqrt{x+2}$ at $a = 0$.
        \\
        \begin{mdframed}
            \begin{equation*}
                \begin{gathered}
                    \sqrt{x+2} = (x+2)^{1/2}                            \\
                    (x+2)^{1/2} = 
                    (2+0)^{1/2}\left(1+\frac{x-0}{2+0}\right)^{1/2} =
                    \sqrt{2}\cdot\sqrt{1+\frac{x}{2}}                   \\
                    \text{Note that: } (1-x)^{r} = 
                    \sum_{k=0}^{\infty}{r\choose k}x^{k}                \\
                    \sqrt{2}\cdot\sqrt{1+\frac{x}{2}} =
                    \sqrt{2}\cdot\left(1-\left(-\frac{x}{2}\right)\right)^{1/2} =
                    \sqrt{2}\sum_{k=0}^{\infty}
                    {1/2\choose k}\left(-\frac{x}{2}\right)^{k}         \\
                    = \boxed{\sum_{k=0}^{\infty}\left[\sqrt{2}(-1)^{k}
                    \frac{(1/2)_{k}}{k!}\left(\frac{x^k}{2^{k}}\right)\right]}
                \end{gathered}
            \end{equation*}
        \end{mdframed}

        \item [202.] Find the Maclaurin series of the function: $f(x) = xe^{2x}$.
        \\
        \begin{mdframed}
            \begin{equation*}
                \begin{gathered}
                    \text{Note that: } e^{x} = 
                    \sum_{k=0}^{\infty}\frac{x^{k}}{k!}                 \\
                    xe^{2x} = x\sum_{k=0}^{\infty}\frac{(2x)^{k}}{k!} =
                    \boxed{\sum_{k=0}^{\infty}\frac{2^{k}(x)^{k+1}}{k!}}
                \end{gathered}
            \end{equation*}
        \end{mdframed}
        
        \item [208.] Find the Maclaurin series of $f(x)=\cos^{2}x$ using the identity $\cos^{2}x=\frac{1}{2}+\frac{1}{2}\cos(2x)$.
        \\
        \begin{mdframed}
            \begin{equation*}
                \begin{gathered}
                    \cos^{2}x=\frac{1}{2}+\frac{1}{2}\cos(2x)           \\
                    \text{Note that: } \cos(x)
                     = \sum_{k=0}^{\infty}(-1)^{k}\frac{x^{2k}}{(2k)!}  \\
                    \cos^{2}x = \frac{1}{2} + \frac{1}{2} \cdot
                    \sum_{k=0}^{\infty}(-1)^{k}\frac{(2x)^{2k}}{(2k)!} =
                    \frac{1}{2} + \sum_{k=0}^{\infty}
                    \frac{(-1)^{k}}{2} \cdot \frac{(2x)^{2k}}{(2k)!} =
                    \frac{1}{2} + \sum_{k=0}^{\infty}
                    (-1)^{k} \frac{2^{k}(x)^{2k}}{(2k)!}                \\
                    \text{Solving for a series of 1/2: } 
                    \frac{1}{n^{k}} = 1/2                               \Rightarrow
                    1 = 1/2(n^{k})                                      \Rightarrow
                    2 = n^{k}                                           \Rightarrow
                    \log_{p}(2) = k                                     \Rightarrow
                    \sum_{k=1}^{\infty}\frac{1}{2^{\log_{k}(2)}} = 1/2  \\
                    \sum_{k=1}^{\infty}\frac{1}{2^{\log_{k}(2)}} + \sum_{k=0}^{\infty}
                    (-1)^{k} \frac{2^{k}(x)^{2k}}{(2k)!} =              
                    \boxed{\sum_{k=0}^{\infty}\frac{1}{2^{\log_{k+1}(2)}} +
                    (-1)^{k} \frac{2^{k}(x)^{2k}}{(2k)!}}
                \end{gathered}
            \end{equation*}
        \end{mdframed}
    \end{itemize}

\end{document}