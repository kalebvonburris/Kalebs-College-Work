% Files using this must be two subfolders
% deep. Adjust the number of ../ for the
% depth of the file.
\providecommand\pointsize{10pt}

\documentclass[\pointsize, letterpaper]{article}

% Imports
\usepackage{fancyhdr}
\usepackage{pgfplots}
\usepackage{geometry}
\usepackage{icomma}
\usepackage{amsmath}
\usepackage{multicol}
\usepackage{mathptmx}
\usepackage{anyfontsize}
\usepackage{t1enc}
\usepackage{tabto}
\usepackage{listings}
\usepackage{filecontents}
\usepackage{subcaption}
\usepackage{tikz}
\usepackage[parfill]{parskip}
\usepackage{graphicx}
\usepackage[]{mdframed}
\usepackage{amsmath}
\usepackage[makeroom]{cancel}
\pgfplotsset{compat=1.18}

\geometry{margin=2.5cm}

\newcommand{\name}{Kaleb Burris}
\newcommand{\classname}{MATH F252, Dr. J. Gimbel}
\newcommand{\assignment}{FILL IN ASSIGNMENT NAME}

\pagestyle{fancy}

\fancyhead[L]{
    \name 
    \newline
    \classname
    \newline
    \assignment
}

\newcommand{\horizontal}{\noindent\rule{\hsize}{0.4pt}}

\setlength{\headheight}{42pt}
\setlength{\headsep}{0.25in}
\setlength{\columnsep}{0.35cm}
\setlength{\columnseprule}{1pt}

% Put class number, class name, and professor 
% name.
% Use only in case of emergency, this
% should be covered by the preamble.
% \renewcommand\classname{}

% Put the assignment name with \S if 
% necessary for the section and the question 
% numbers.
\renewcommand\assignment{Homework Set III, Due Friday, January 27th, 2023}

\begin{document}
    % Templates
    \iffalse
    % Use these for equations.
    \begin{equation*}
        \begin{gathered}
            Equations go here.
        \end{gathered}
    \end{equation*}

    % Use this if a line of math is too long.
    \resizebox{\hsize}{!}{$Long equation goes here$}

    % Use these for multiple columns.
    \begin{multicol*}{# of columns}
        % Remove the * if you want the columns to be balanced.
    \end{multicol*}

    % Use this to add a horizontal line.
    \horizontal

    \fi

    % Begin homework here.
    %%%%%%%%%%%%%%%%%%%%%%

    \paragraph*{1.}
    Evaluate $\int\limits_{1}^{e}\frac{(1+\ln(x))^2}{x}\mathrm{d}x$

    \begin{mdframed}
        \begin{equation*}
            \begin{gathered}
                \text{let} \quad u = \ln x + 1 \quad \mathrm{d}u = \frac{1}{x}\mathrm{d}x    \\
                u_{lower} = \ln(1) + 1 = 1 \quad u_{upper} = \ln(e) + 1 = 2 \\
                \rightarrow \int_{1}^{2}u^2\mathrm{d}u = \left.\frac{2}{3}u^3\right|_{1}^{2} = \frac{2(2^3) - 2}{3} = \boxed{\frac{14}{3}}
            \end{gathered}
        \end{equation*}
    \end{mdframed}

    \paragraph*{2.}
    Evaluate $\int\frac{\mathrm{d}x}{\sqrt{25-x^2}}$

    \begin{mdframed}
        \begin{equation*}
            \begin{gathered}
                \text{Using} \quad a^2 - u^2 \implies u = a\sin\theta \quad -\frac{\pi}{2} < \theta < \frac{\pi}{2}  \\
                \Rightarrow x = 5\sin\theta \quad \mathrm{d}x = 5\cos\theta\mathrm{d}\theta \quad \theta = \sin^{-1}\left(\frac{x}{5}\right)    \\
                \int\frac{5\cos\theta\mathrm{d}\theta}{\sqrt{25-25\sin^{2}\theta}} = \int\frac{\cancel{5}\cos\theta\mathrm{d}\theta}{\cancel{5}\sqrt{1-\sin^{2}\theta}} = \int\frac{\cos\theta\mathrm{d}\theta}{\sqrt{\cos^{2}\theta}}  \\
                = \int\frac{\cos\theta\mathrm{d}\theta}{\cos\theta} = \int\mathrm{d}\theta = \theta + C = \boxed{\sin^{-1}\left(\frac{x}{5}\right) + C}
            \end{gathered}
        \end{equation*}
    \end{mdframed}

    \paragraph*{3.}
    Evaluate $\int\frac{\mathrm{d}x}{\sqrt{25+4x^2}}$

    \begin{mdframed}
        \begin{equation*}
            \begin{gathered}
                \text{Using} \quad a^2 + u^2 \implies u = a\tan\theta \quad -\frac{\pi}{2} < \theta < \frac{\pi}{2}  \\
                \Rightarrow 2x = 5\tan\theta \quad \mathrm{d}x = \frac{5}{2}\sec^{2}\theta\mathrm{d}\theta \quad \theta = \tan^{-1}\left(\frac{2x}{5}\right)    \\
                \int\frac{\frac{5}{2}\sec^{2}\theta\mathrm{d}\theta}{\sqrt{25+25\tan^{2}\theta}} = \frac{5}{2}\int\frac{\sec^{2}\theta\mathrm{d}\theta}{5\sqrt{1+\tan^{2}\theta}} =  \frac{1}{2}\int\frac{\sec^{2}\theta\mathrm{d}\theta}{\sqrt{\sec^{2}\theta}}    \\
                = \frac{1}{2}\int\sec\theta\mathrm{d}\theta = \frac{1}{2}(\ln|\sec\theta+\tan\theta|) + C   \\
                = \frac{1}{2}\left[\ln\left|\sec\left(\tan^{-1}\left(\frac{2x}{5}\right)\right) + \tan\left(\tan^{-1}\left(\frac{2x}{5}\right)\right)\right|\right] + C = \boxed{\frac{1}{2}\left[\ln\left|\sec\left(\tan^{-1}\left(\frac{2x}{5}\right)\right) + \frac{2x}{5}\right|\right] + C}
            \end{gathered}
        \end{equation*}
    \end{mdframed}

    \paragraph*{4.}
    Evaluate $\int\frac{e^x}{1+e^{2x}}$

    \begin{mdframed}
        \begin{equation*}
            \begin{gathered}
                \text{Using} \quad a^2 + u^2 \implies u = a\tan\theta \quad -\frac{\pi}{2} < \theta < \frac{\pi}{2}  \\
                \Rightarrow e^x = \tan\theta \quad e^x\mathrm{d}x = \sec^{2}\theta\mathrm{d}\theta \quad \theta = \tan^{-1}\left(e^x\right) \\
                \int\frac{\sec^2\theta\mathrm{d}\theta}{1+\tan^2\theta} = \int\frac{\sec^2\theta\mathrm{d}\theta}{\sec^2\theta} = \int\theta\mathrm{d}\theta = \theta + C = \boxed{\tan^{-1}\left(e^x\right) + C}
            \end{gathered}
        \end{equation*}
    \end{mdframed}

    \paragraph*{4.}
    Find the area between the curves: $y = x^2 + 1$ and $y = x + 1$

    \begin{mdframed}
        \begin{multicols*}{2}
            \begin{align*}
                x + 1   & = x^2 + 1   \\
                x - x^2 & = 0       \\
                x(1 - x)& = 0       \\
                      x & = (0, 1)
            \end{align*}
            \begin{equation*}
                \begin{gathered}
                    \int_{0}^{1}(x- x^2)\mathrm{d}x    \\
                    = \left.\left(\frac{x^2}{2}-\frac{x^3}{3}\right)\right|_{0}^{1} = \frac{1}{2}-\frac{1}{3} = \frac{3-1}{6} = \boxed{\frac{1}{6}}
                \end{gathered}
            \end{equation*}
        \end{multicols*}
    \end{mdframed}

\end{document}