\providecommand\pointsize{10pt}

\documentclass[\pointsize, letterpaper]{article}

% Imports
\usepackage{fancyhdr}
\usepackage{geometry}
\usepackage{icomma}
\usepackage{amsmath}
\usepackage{multicol}
\usepackage{mathptmx}
\usepackage{anyfontsize}
\usepackage{t1enc}
\usepackage{tabto}
\usepackage{listings}
\usepackage{filecontents}
\usepackage{subcaption}
\usepackage{tikz}
\usepackage[parfill]{parskip}
\usepackage{graphicx}
\usepackage[]{mdframed}
\usepackage{amsmath}
\usepackage[makeroom]{cancel}
\usepackage{pgfplots}
\usepackage{pgfplotstable}
\usepackage{xfrac}
\usepackage{amssymb}
\usetikzlibrary{patterns}
\usepgfplotslibrary{fillbetween}
\pgfplotsset{compat=1.18}

\geometry{margin=2.5cm}

\newcommand{\name}{Kaleb Burris}
\newcommand{\classname}{MATH F252, Dr. J. Gimbel, University of Alaska Fairbanks}
\newcommand{\assignment}{FILL IN ASSIGNMENT NAME}

\pagestyle{fancy}

\fancyhead[L]{
    \name 
    \newline
    \classname
    \newline
    \assignment
}

\newcommand{\horizontal}{\noindent\rule{\hsize}{0.4pt}}

\setlength{\headheight}{42pt}
\setlength{\headsep}{0.25in}
\setlength{\columnsep}{0.35cm}
\setlength{\columnseprule}{1pt}

\usepackage{pgfplots}
\usepackage{pgfplotstable}
\makeatletter
\long\def\ifnodedefined#1#2#3{%
    \@ifundefined{pgf@sh@ns@#1}{#3}{#2}%
}

\pgfplotsset{
    discontinuous/.style={
    scatter,
    scatter/@pre marker code/.code={
        \ifnodedefined{marker}{
            \pgfpointdiff{\pgfpointanchor{marker}{center}}%
             {\pgfpoint{0}{0}}%
             \ifdim\pgf@y>0pt
                \tikzset{options/.style={mark=*, fill=white}}
                \draw [densely dashed,blue] (marker-|0,0) -- (0,0);
                \draw plot [mark=*] coordinates {(marker-|0,0)};
             \else
                \tikzset{options/.style={mark=none}}
             \fi
        }{
            \tikzset{options/.style={mark=none}}        
        }
        \coordinate (marker) at (0,0);
        \begin{scope}[options]
    },
    scatter/@post marker code/.code={\end{scope}}
    }
}
\makeatother

\usepackage[T1]{fontenc}
\usepackage{lmodern}

% Put class number, class name, and professor 
% name.
% Use only in case of emergency, this
% should be covered by the preamble.
% \renewcommand\classname{}

% Put the assignment name with \S if 
% necessary for the section and the question 
% numbers.
\renewcommand\assignment{Homework Set IX, Due Friday, March 31, 2023}

\begin{document}
    % Templates
    \iffalse
    % Use these for equations.
    \begin{equation*}
        \begin{gathered}
            Equations go here.
        \end{gathered}
    \end{equation*}

    % Use this if a line of math is too long.
    \resizebox{\hsize}{!}{$Long equation goes here$}

    % Use these for multiple columns.
    \begin{multicol*}{# of columns}
        % Remove the * if you want the columns to be balanced.
    \end{multicol*}

    % Use this to add a horizontal line.
    \horizontal

    \fi

    % Begin homework here.
    %%%%%%%%%%%%%%%%%%%%%%

    Find the radius and interval of convergence for both of the following:

    \begin{itemize}
        \item [1] $\sum\frac{x^{2}}{k3^{k}}$
        \\
        \begin{mdframed}
            \begin{equation*}
                \begin{gathered}
                    \sqrt[k]{\left|\frac{x^{2}}{k3^{k}}\right|}
                    = \frac{\sqrt[k]{x^{2}}}{\sqrt[k]{|k3^{k}|}}
                    = \frac{\sqrt[k]{x^{2}}}{3}                     \\
                    \lim_{k\to\infty}\frac{\sqrt[k]{x^{2}}}{3} < 3  \Rightarrow
                    \lim_{k\to\infty}x^{\frac{2}{k}} < 9            \Rightarrow
                    x^{0} < 9 \Rightarrow 1 < 9                     \\
                    \text{Radius} = \infty \quad \text{Interval} = (-\infty,\infty)
                \end{gathered}
            \end{equation*}
        \end{mdframed}

        \item[2.] $\sum\frac{(x-2)^{k}}{5k^{2}4^{4}}$
        \\
        \begin{mdframed}
            \begin{equation*}
                \begin{gathered}
                    \left|\frac {\frac{(x-2)^{k}}{5k^{2}4^{4}}}
                                {\frac{(x-2)^{k+1}}{5(k+1)^{2}4^{4}}}\right|
                    = \left|\frac{(x-2)^{k}}{5k^{2}4^{4}}\frac{5(k+1)^{2}4^{4}}{(x-2)^{k+1}}\right|
                    = \left|\frac{1}{k^{2}}\frac{(k+1)^{2}}{x-2}\right|
                    = \left|\frac{\sqrt{(k+1)^{2}}}{\sqrt{k^{2}(x-2)}}\right|
                    = \frac{k+1}{k\sqrt{x-2}}   \\
                    \frac{k+1}{k\sqrt{x-2}}     \to 
                    \frac{1}{\sqrt{x-2}}        \Rightarrow
                    \frac{1}{\sqrt{x-2}} < 1    \Rightarrow
                    1 < \sqrt{x-2}              \Rightarrow
                    1 < x-2                     \Rightarrow
                    3 < x                       \\
                    \text{Note: at $x$ = 3, } 
                    \sum\frac{(3-2)^{k}}{5k^{2}4^{4}} =
                    \frac{1}{5(4^{4})}\sum\frac{1}{k^{2}}
                    \quad \text{ the series converges by P-Series}
                    \\
                    \text{Radius} = \infty \quad \text{Interval} = [3,\infty)
                \end{gathered}
            \end{equation*}
        \end{mdframed}
    \end{itemize}

\end{document}