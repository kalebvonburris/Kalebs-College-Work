% Files using this must be two subfolders
% deep. Adjust the number of ../ for the
% depth of the file.
\providecommand\pointsize{10pt}

\documentclass[\pointsize, letterpaper]{article}

% Imports
\usepackage{fancyhdr}
\usepackage{pgfplots}
\usepackage{geometry}
\usepackage{icomma}
\usepackage{amsmath}
\usepackage{multicol}
\usepackage{mathptmx}
\usepackage{anyfontsize}
\usepackage{t1enc}
\usepackage{tabto}
\usepackage{listings}
\usepackage{filecontents}
\usepackage{subcaption}
\usepackage{tikz}
\usepackage[parfill]{parskip}
\usepackage{graphicx}
\usepackage[]{mdframed}
\usepackage{amsmath}
\usepackage[makeroom]{cancel}
\pgfplotsset{compat=1.18}

\geometry{margin=2.5cm}

\newcommand{\name}{Kaleb Burris}
\newcommand{\classname}{MATH F252, Dr. J. Gimbel}
\newcommand{\assignment}{FILL IN ASSIGNMENT NAME}

\pagestyle{fancy}

\fancyhead[L]{
    \name 
    \newline
    \classname
    \newline
    \assignment
}

\newcommand{\horizontal}{\noindent\rule{\hsize}{0.4pt}}

\setlength{\headheight}{42pt}
\setlength{\headsep}{0.25in}
\setlength{\columnsep}{0.35cm}
\setlength{\columnseprule}{1pt}
\usepackage{tkz-fct}

% Put class number, class name, and professor 
% name.
% Use only in case of emergency, this
% should be covered by the preamble.
% \renewcommand\classname{}

% Put the assignment name with \S if 
% necessary for the section and the question 
% numbers.
\renewcommand\assignment{Final Review}

\begin{document}
    % Templates
    \iffalse
    % Use these for equations.
    \begin{equation*}
        \begin{gathered}
            Equations go here.
        \end{gathered}
    \end{equation*}

    % Use this if a line of math is too long.
    \resizebox{\hsize}{!}{$Long equation goes here$}

    % Use these for multiple columns.
    \begin{multicol*}{# of columns}
        % Remove the * if you want the columns to be balanced.
    \end{multicol*}

    % Use this to add a horizontal line.
    \horizontal

    \fi

    % Begin homework here.
    %%%%%%%%%%%%%%%%%%%%%%

    \section*{Part I}

    \begin{itemize}
        \item [1.] Evaluate $\int_{1}^{\infty} \frac{1}{1+x^{2}}dx$
        \\
        \begin{mdframed}
            \begin{equation*}
                \begin{gathered}
                    \int_{1}^{\infty} \frac{1}{1+x^{2}}dx =
                    \lim_{a \to \infty} \int_{1}^{a} \frac{1}{1+x^{2}}dx    \\
                    \lim_{a \to \infty} \int_{1}^{a} \frac{1}{1+x^{2}}dx =
                    \lim_{a \to \infty} \Big[\arctan(x)\Big]_{1}^{a} =
                    \lim_{a \to \infty} \arctan(a) - \arctan(1)             \\
                    = \frac{\pi}{2} - \frac{\pi}{4} = \boxed{\frac{\pi}{4}}
                \end{gathered}
            \end{equation*}
        \end{mdframed}
        
        \item [2.] Suppose:
        
        $x=4-\ln(t)$

        $y=1+\ln(7t)$

        $1 \leq t \leq e$

        Compute the arc length.
        \\
        \begin{mdframed}
            \begin{equation*}
                \begin{gathered}
                    \text{Arc length: } \int_{a}^{b}\sqrt{\left(\frac{dx}{dt}\right)^{2} + \left(\frac{dy}{dt}\right)^{2}}dt    \\
                    \frac{dx}{dt} = -\frac{1}{t}                \\
                    \frac{dy}{dt} = \frac{7}{7t} = \frac{dt}{t} \\
                    \int_{1}^{e}\sqrt{\left(-\frac{1}{t}\right)^{2} + \left(\frac{1}{t}\right)^{2}}dt =
                    \int_{1}^{e}\sqrt{\frac{1}{t^{2}} + \frac{1}{t^{2}}}dt =
                    \int_{1}^{e}\sqrt{\frac{2}{t^{2}}}dt =
                    \sqrt{2}\int_{1}^{e}\frac{1}{t}dt           \\
                    = \sqrt{2}\Big[\ln(t)\Big]_{1}^{e} =
                    \sqrt{2}\Big[\ln(e) - \ln(1)\Big] =
                    \boxed{\sqrt{2}}
                \end{gathered}
            \end{equation*}
        \end{mdframed}

        \item [3.] Suppose:
        
        $x=\cos(3+t)$

        $y=\sin(3+t)$

        What is the area of the region bounded by the graph and the positive x-axis and the positive y-axis?
        \\
        \begin{mdframed}
            \begin{multicols}{2}
                \hfill
                \begin{center}
                    \resizebox{0.8\hsize}{!}{
                    \begin{tikzpicture}
                        \begin{axis} [
                            xmin = 0, xmax = 1.5,
                            ymin = 0, ymax = 1.5,
                            trig format plots=rad,
                            axis equal, axis lines = center,
                            xlabel={$x=\cos(3+t)$}, 
                            ylabel={$y=\sin(3+t)$}
                        ]
                            \addplot [domain=-3:pi/2-3, samples=50, red, thick] ({sin(3+x)}, {cos(3+x)});
                        \end{axis}
                    \end{tikzpicture}
                }
                \end{center}
                \hfill
                
                \columnbreak

                \begin{equation*}
                    \begin{gathered}
                        \text{To find where the curve strikes the axes: }       \\
                        x = \cos(3+t) = 0; \quad 3+t = \arccos(0)               \\
                        3 + t = \frac{\pi}{2}, \quad 
                        t = \frac{\pi}{2} - 3 \text{ when } x = 0               \\
                        y = \sin(3+t) = 0; \quad 3 + t = \arcsin(0)             \\
                        3 + t = 0, \quad 
                        t = -3 \text{ when } y = 0                              \\
                        A = \int_{-3}^{\frac{\pi}{2}-3}\sin(3+t)(-\sin(3+t))dt  \\
                        = -\int_{-3}^{\frac{\pi}{2}-3}\sin^{2}(3+t)dt
                    \end{gathered}
                \end{equation*}
            \end{multicols}

            \begin{equation*}
                \begin{gathered}
                    -\int_{-3}^{\frac{\pi}{2}-3}\sin^{2}(3+t)dt =
                    -\int_{-3}^{\frac{\pi}{2}-3}\frac{1-\cos(6+2t)}{2}dt        \\
                    = 
                    -\left[
                        \frac{1}{2}t-\frac{\sin(6+2t)}{4}
                    \right]_{-3}^{\frac{\pi}{2}-3} =
                    -\left[
                        \left(\frac{\pi-12}{4}-\frac{\sin(6+\pi-3)}{4}\right) -
                        \left(\frac{-3}{2}-\frac{\sin(6-6)}{2}\right)
                    \right]                                                     \\
                    =
                    -\left(\frac{\pi-6}{4}-\frac{\sin(3+\pi)}{4}\right) \approx \boxed{0.679}
                \end{gathered}
            \end{equation*}
        \end{mdframed}
        
        \item [4.] Use the root test to tell if the series converges: $\sum\sqrt{\frac{1+n^{2}}{1+3^{n}}}$
        \\
        \begin{mdframed}
            \begin{equation*}
                \begin{gathered}
                    \lim_{n \to \infty}
                    \sqrt[n]
                    {\left|
                        \frac{1+n^{2}}{1+3^{n}}
                    \right|}
                    = \lim_{n \to \infty}
                    \frac{\sqrt[n]{1+n^{2}}}{\sqrt[n]{1+3^{n}}}                 \\
                    \horizontal                                                 \\
                    \text{Note that: } 
                    \underbrace{\lim_{n \to \infty}\sqrt[n]{n^{2}}}
                    _{\mathclap{\sqrt[n]{n^{2}} = \sqrt[n]{n}^{2} \to 1}} \leq
                    \lim_{n \to \infty}\sqrt[n]{1+n^{2}} \leq
                    \underbrace{\lim_{n \to \infty}\sqrt[n]{2n^{2}}}
                    _{\mathclap{\sqrt[n]{2n^{2}} = 
                    \sqrt[n]{2}\cdot\sqrt[n]{n}^{2} \to 1}}  \\
                    1 \leq \lim_{n \to \infty}\sqrt[n]{1+n^{2}} \leq 1, \quad 
                    \therefore \lim_{n \to \infty}\sqrt[n]{1+n^{2}} = 1         \\
                    \horizontal                                                 \\
                    \text{Note that: } 
                    \underbrace{\lim_{n \to \infty}\sqrt[n]{3^{n}}}
                    _{\mathclap{\sqrt[n]{3^{n}} = 3}} \leq
                    \lim_{n \to \infty}\sqrt[n]{1+3^{n}} \leq
                    \underbrace{\lim_{n \to \infty}\sqrt[n]{2\cdot3^{n}}}
                    _{\mathclap{\sqrt[n]{2\cdot3^{n}} = 
                    \sqrt[n]{2}\cdot\sqrt[n]{3}^{n} \to 1\cdot3 = 3}}           \\
                    3 \leq \lim_{n \to \infty}\sqrt[n]{1+3^{n}} \leq 3, \quad 
                    \therefore \lim_{n \to \infty}\sqrt[n]{1+3^{n}} = 3         \\
                    \horizontal                                                 \\
                    \therefore \lim_{n \to \infty}
                    \frac{\sqrt[n]{1+n^{2}}}{\sqrt[n]{1+3^{n}}} = \frac{1}{3}, \quad
                    \frac{1}{3} < 1, \quad \boxed{\text{Converges}}
                \end{gathered}
            \end{equation*}
        \end{mdframed}

        \item [5.] 
    \end{itemize}

    \pagebreak

    \section*{Part II}


\end{document}