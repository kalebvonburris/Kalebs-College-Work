% Imports
\usepackage{fancyhdr}
\usepackage{geometry}
\usepackage{icomma}
\usepackage{amsmath}
\usepackage{multicol}
\usepackage{mathptmx}
\usepackage{anyfontsize}
\usepackage{t1enc}
\usepackage{tabto}
\usepackage{listings}
\usepackage{filecontents}
\usepackage{subcaption}
\usepackage{tikz}
\usepackage[parfill]{parskip}
\usepackage{graphicx}
\usepackage[]{mdframed}
\usepackage{amsmath}
\usepackage[makeroom]{cancel}
\usepackage{pgfplots}
\usepackage{pgfplotstable}
\usepackage{xfrac}
\usepackage{amssymb}
\usepackage{mathtools}
\pgfplotsset{compat=1.18}
\usetikzlibrary{patterns}
\usepgfplotslibrary{polar}
\usepgfplotslibrary{fillbetween}

\geometry{margin=2.5cm}

\newcommand{\name}{Kaleb Burris}
\newcommand{\classname}{MATH F253, Elizabeth S. Allman, University of Alaska Fairbanks}
\newcommand{\assignment}{FILL IN ASSIGNMENT NAME}

\pagestyle{fancy}

\fancyhead[L]{
    \name 
    \newline
    \classname
    \newline
    \assignment
}

\newcommand{\horizontal}{\noindent\rule{\hsize}{0.4pt}}

\setlength{\headheight}{42pt}
\setlength{\headsep}{0.25in}
\setlength{\columnsep}{0.35cm}
\setlength{\columnseprule}{1pt}

\usepackage[T1]{fontenc}
\usepackage{lmodern}

\renewcommand\classname{MATH 252X Calculus II}
\renewcommand\assignment{HWRK \S 1.2; \# 1, 2, 3, 4, 5, 6}

\begin{document}

    \iffalse
    % Use these for equations.
    \begin{equation*}
        \begin{gathered}
            Equations go here.
        \end{gathered}
    \end{equation*}

    % Use this if a line of math is too long.
    \resizebox{\hsize}{!}{$Long equation goes here$}

    % Use these for multiple columns.
    \begin{multicol*}{# of columns}
        % Remove the * if you want the columns to be balanced.
    \end{multicol*}

    % Use this to add a horizontal line.
    \horizontal

    \fi

    $\Rightarrow$ Indicates the beginning of work.

    \begin{multicols*}{2}

        \paragraph{1.}
            Evaluate the following definite integral:
            \begin{equation*}
                \begin{gathered}
                \int_{2}^{40} x^2 \mathrm{d}x
                \\
                \Rightarrow \frac{x^3}{3} \Big|_{2}^{40} = \frac{40^3}{3} - \frac{2^3}{3} = \frac{64,000}{3} - \frac{8}{3}
                \\
                = \boxed{\frac{63,992}{3} \approx 21,330.\overline{66}}
                \end{gathered}
            \end{equation*}
            
            \horizontal

        \paragraph{2.}

            Evaluate the following summation:

        \begin{equation*}
            \begin{gathered}
                \sum_{n=0}^{5} n - 1
                \\
                \resizebox{0.95\hsize}{!}{$\Rightarrow (0 - 1) + (1 - 1) + (2 - 1) + (3 - 1) + (4 - 1) + (5 - 1)$}
                \\
                = 4 + 3 + 2 + 1 + 0 - 1 = \boxed{9}
            \end{gathered}
        \end{equation*}

        \horizontal

        \noindent Use \emph{u} substitution to evaluate the following integrals:

        \paragraph{3.}
            
            \begin{equation*}
                \begin{gathered}
                    \int 3x^2(x^3+1)^6 \mathrm{d}x
                    \\
                    \Rightarrow u = x^3 + 1 \quad \mathrm{d}u = (3x^2)\mathrm{d}x
                    \\
                    \int u^6 \mathrm{d}u = \boxed{\frac{u^7}{7} + C = \frac{(x^3 + 1)^7}{7} + C}
                \end{gathered}
            \end{equation*}

            \horizontal

        \paragraph{4.}
                
            \begin{equation*}
                \begin{gathered}
                    \int e^{2x}\mathrm{d}x
                    \\
                    \Rightarrow u = 2x \quad \frac{1}{2}\mathrm{d}u = \mathrm{d}x
                    \\
                    \frac{1}{2} \int e^u\mathrm{d}u = \boxed{\frac{e^u}{2} + C = \frac{e^{2x}}{2} + C}
                \end{gathered}
            \end{equation*}

            \horizontal
            
        \paragraph{5.}

            \begin{equation*}
                \begin{gathered}
                    \int \frac{6x}{(5 + 3x^2)^4}\mathrm{d}x
                    \\
                    \Rightarrow u = 5 + 3x^2 \quad \mathrm{d}u = 6x \mathrm{d}x
                    \\
                    \resizebox{0.95\hsize}{!}{$= \int \frac{1}{u^4} \mathrm{d}u = \int u^{-4} \mathrm{d}u = \boxed{-\frac{u^{-3}}{3} + C = -\frac{(5 + 3x^2)^{-3}}{3} + C}$}
                \end{gathered}
            \end{equation*}
            
            \horizontal

        \paragraph{6.}

        \begin{equation*}
            \begin{gathered}
                \int \frac{4 \sin(x)}{3 + \cos(x)} \mathrm{d}x
                \\
                \Rightarrow u = 3 + \cos(x) \quad \mathrm{d}u = -\sin(x)\mathrm{d}x
                \\
                -4 \int \frac{1}{u} \mathrm{d}u = \boxed{-4 \ln|u| + C = -4 \ln|3 + \cos(x)| + C}
            \end{gathered}
        \end{equation*}
        
        \horizontal

    \end{multicols*}
\end{document}