% Files using this must be two subfolders
% deep. Adjust the number of ../ for the
% depth of the file.
% Imports
\usepackage{fancyhdr}
\usepackage{geometry}
\usepackage{icomma}
\usepackage{amsmath}
\usepackage{multicol}
\usepackage{mathptmx}
\usepackage{anyfontsize}
\usepackage{t1enc}
\usepackage{tabto}
\usepackage{listings}
\usepackage{filecontents}
\usepackage{subcaption}
\usepackage{tikz}
\usepackage[parfill]{parskip}
\usepackage{graphicx}
\usepackage[]{mdframed}
\usepackage{amsmath}
\usepackage[makeroom]{cancel}
\usepackage{pgfplots}
\usepackage{pgfplotstable}
\usepackage{xfrac}
\usepackage{amssymb}
\usepackage{mathtools}
\pgfplotsset{compat=1.18}
\usetikzlibrary{patterns}
\usepgfplotslibrary{polar}
\usepgfplotslibrary{fillbetween}

\geometry{margin=2.5cm}

\newcommand{\name}{Kaleb Burris}
\newcommand{\classname}{MATH F253, Elizabeth S. Allman, University of Alaska Fairbanks}
\newcommand{\assignment}{FILL IN ASSIGNMENT NAME}

\pagestyle{fancy}

\fancyhead[L]{
    \name 
    \newline
    \classname
    \newline
    \assignment
}

\newcommand{\horizontal}{\noindent\rule{\hsize}{0.4pt}}

\setlength{\headheight}{42pt}
\setlength{\headsep}{0.25in}
\setlength{\columnsep}{0.35cm}
\setlength{\columnseprule}{1pt}

\usepackage[T1]{fontenc}
\usepackage{lmodern}

% Put class number, class name, and professor 
% name.
% Use only in case of emergency, this
% should be covered by the preamble.
% \renewcommand\classname{}

% Put the assignment name with \S if 
% necessary for the section and the question 
% numbers.
\renewcommand\assignment{Worksheet 13, Due February 17, 4:15pm}

\begin{document}
    % Templates
    \iffalse
    % Use these for equations.
    \begin{equation*}
        \begin{gathered}
            Equations go here.
        \end{gathered}
    \end{equation*}

    % Use this if a line of math is too long.
    \resizebox{\hsize}{!}{$Long equation goes here$}

    % Use these for multiple columns.
    \begin{multicol*}{# of columns}
        % Remove the * if you want the columns to be balanced.
    \end{multicol*}

    % Use this to add a horizontal line.
    \horizontal

    \fi

    % Begin homework here.
    %%%%%%%%%%%%%%%%%%%%%%

    Let $X$ be a continuous random variable ith the pdf given by:

    \begin{equation*}
        f(x) = 
        \left\{
            {\def\arraystretch{1.2}
            \begin{array}{lll}
            0   & \quad & \text{if } x \leq 0       \\
            1/4 & \quad & \text{if } 0 < x \leq 1   \\
            0   & \quad & \text{if } 1 < x \leq 2   \\
            3/8 & \quad & \text{if } 2 < x \leq 4   \\
            0   & \quad & \text{if } 4 < x  
        \end{array}}
        \right.
    \end{equation*}

    \paragraph*{1.}
    Sketch the graph of $f(x)$.
    \\
    \begin{mdframed}
        \centering
        \begin{tikzpicture}
            \begin{axis}[
                xmax=5, ymax=1,
                xmin=0, ymin=-0.2    
            ]
                \draw (0, 1/4) -- (1, 1/4); \node[inner sep=0,circle,draw,fill=black,minimum size=5pt] at (1,1/4){};
                \draw (1, 0)   -- (2, 0);  \node[inner sep=0,circle,draw,fill=white,minimum size=5pt] at (2,0){};
                \draw (2, 3/8)   -- (4, 3/8);  \node[inner sep=0,circle,draw,fill=black,minimum size=5pt] at (2, 3/8){};
                \draw (4, 0)   -- (5, 0);  \node[inner sep=0,circle,draw,fill=black,minimum size=5pt] at (4,3/8){};
                \node[inner sep=0,circle,draw,fill=white,minimum size=5pt] at (4,0){};
                \node[inner sep=0,circle,draw,fill=white,minimum size=5pt] at (1,0){};
            \end{axis}
        \end{tikzpicture}
    \end{mdframed}

    \paragraph*{2.}
    Show that $f(x)$ is a pdf.
    \\
    \begin{mdframed}
        Because the sections are rectangular, we can just sum their width * height:

        \begin{equation*}
            (0 \cdot -\infty) + \left(\frac{1}{4} \cdot 1\right) + (0 \cdot 1) + \left(\frac{3}{8} \cdot 2\right) + (0 \cdot \infty) = 0 + \frac{1}{4} + 0 + \frac{3}{4} + 0 = \frac{4}{4} = 1
        \end{equation*}
        Because the sum of the ``integrals'' is 1, $f(x)$ is a pdf.
    \end{mdframed}

    \pagebreak

    \paragraph*{3.}
    Find $F(x)$, the cdf of $X$. Sketch its graph. (The sketch can be fairly crude, but be sure to show what happens at $x = 0, 1, 2, 4$. In particular, the cdf, $F(x)$, is continuous because X is a continuous random variable, so there are no jumps in the cdf. It is continuous as well.)
    \\
    \begin{mdframed}
        \begin{align*}
            F(X)    & = \int_{-\infty}^{x} f(t)dt   \\
                    & = \left\{
                        {\def\arraystretch{1.2}
                        \begin{array}{l}
                            \int_{-\infty}^{0}0dt    \\
                            \int_{0}^{1} 1/4dt      \\
                            \int_{1}^{2}0dt         \\
                            \int_{2}^{4}3/8dt       \\
                            \int_{4}^{\infty}0dt
                        \end{array}}
                    \right.                         \\
                    & = \left\{
                        {\def\arraystretch{1.2}
                        \begin{array}{lll}
                            0           & \quad & \text{if } 0 < x          \\
                            \frac{x}{4} & \quad & \text{if } 0 < x \leq 1   \\
                            \frac{1}{4} & \quad & \text{if } 1 < x \leq 2   \\
                            \frac{3(x-2)}{8} + \frac{1}{4} & \quad & \text{if } 2 < x \leq 4\\
                            1           & \quad & \text{if } 4 < x
                        \end{array}}
                    \right.
        \end{align*}

        \centering
        \begin{tikzpicture}
            \begin{axis}[
                xmax=5, ymax=1.2,
                xmin=0, ymin=0    
            ]
                \addplot[domain=0:1]{x/4};
                \addplot[domain=1:2]{1/4};
                \addplot[domain=2:4]{3*(x - 2)/8 + 1/4};
                \addplot[domain=4:5]{1};
            \end{axis}
        \end{tikzpicture}
    \end{mdframed}

    \pagebreak

    \paragraph*{4.}
    Use the cdf to find the following probabilities:

    \begin{enumerate}[label=(\alph*)]
        \item $P(X \leq 2.4)$
        \\
        \begin{mdframed}
            \begin{align*}
                P(X \leq 2.4)   & = \frac{3(2.4-2)}{8} + \frac{1}{4}    \\
                F(2.4)          & = \frac{3(0.4)}{8} + 0.25             \\
                                & = \frac{1.2}{8} + 0.25 = \boxed{0.65}
            \end{align*}    
        \end{mdframed}
        
        \item $P(0.7 \leq X \leq 1.0)$
        \\
        \begin{mdframed}
            \begin{align*}
                P(0.7 \leq X \leq 1.0)  & = 1 - F(0.7)          \\
                                        & = 1 - \frac{0.7}{4}   \\
                                        & = 1 - 0.175 = \boxed{0.825}
            \end{align*}    
        \end{mdframed}
        
        \item $P(x > 2.5)$
        \\
        \begin{mdframed}
            \begin{align*}
                P(X > 2.5)  & = 1 - F(2.5)                      \\
                            & = 1 - \frac{3(2.5-2)}{8} + 0.25   \\
                            & = 1 - \frac{3(0.5)}{8} + 0.25     \\
                            & = 1 - \frac{1.5}{8} + 0.25        \\
                            & = 1 - 0.1875 + 0.25 = \boxed{4375}
            \end{align*}    
        \end{mdframed}
        
        \item $P(0.3 \leq X \leq 3.5)$
        \\
        \begin{mdframed}
            \begin{align*}
                P(0.3 \leq X \leq 3.5)  & = F(3.5) - F(0.3)                 \\
                                        & = \left(\frac{3(3.5-2)}{8}\right) - \frac{0.3}{4}   \\
                                        & = \left(\frac{3(1.5)}{8}\right) - 0.075   \\
                                        & = \left(\frac{4.5}{8}\right) - 0.075      \\
                                        & = 0.5625 - 0.075 = \boxed{0.4875}
            \end{align*}
        \end{mdframed}
    \end{enumerate}


\end{document}