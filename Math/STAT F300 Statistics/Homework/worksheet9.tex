% Files using this must be two subfolders
% deep. Adjust the number of ../ for the
% depth of the file.
% Imports
\usepackage{fancyhdr}
\usepackage{geometry}
\usepackage{icomma}
\usepackage{amsmath}
\usepackage{multicol}
\usepackage{mathptmx}
\usepackage{anyfontsize}
\usepackage{t1enc}
\usepackage{tabto}
\usepackage{listings}
\usepackage{filecontents}
\usepackage{subcaption}
\usepackage{tikz}
\usepackage[parfill]{parskip}
\usepackage{graphicx}
\usepackage[]{mdframed}
\usepackage{amsmath}
\usepackage[makeroom]{cancel}
\usepackage{pgfplots}
\usepackage{pgfplotstable}
\usepackage{xfrac}
\usepackage{amssymb}
\usepackage{mathtools}
\pgfplotsset{compat=1.18}
\usetikzlibrary{patterns}
\usepgfplotslibrary{polar}
\usepgfplotslibrary{fillbetween}

\geometry{margin=2.5cm}

\newcommand{\name}{Kaleb Burris}
\newcommand{\classname}{MATH F253, Elizabeth S. Allman, University of Alaska Fairbanks}
\newcommand{\assignment}{FILL IN ASSIGNMENT NAME}

\pagestyle{fancy}

\fancyhead[L]{
    \name 
    \newline
    \classname
    \newline
    \assignment
}

\newcommand{\horizontal}{\noindent\rule{\hsize}{0.4pt}}

\setlength{\headheight}{42pt}
\setlength{\headsep}{0.25in}
\setlength{\columnsep}{0.35cm}
\setlength{\columnseprule}{1pt}

\usepackage[T1]{fontenc}
\usepackage{lmodern}

% Put class number, class name, and professor 
% name.
% Use only in case of emergency, this
% should be covered by the preamble.
% \renewcommand\classname{}

% Put the assignment name with \S if 
% necessary for the section and the question 
% numbers.
\renewcommand\assignment{Worksheet 9, Due Wednesday, February 8, 4:15pm}

\begin{document}
    % Templates
    \iffalse
    % Use these for equations.
    \begin{equation*}
        \begin{gathered}
            Equations go here.
        \end{gathered}
    \end{equation*}

    % Use this if a line of math is too long.
    \resizebox{\hsize}{!}{$Long equation goes here$}

    % Use these for multiple columns.
    \begin{multicol*}{# of columns}
        % Remove the * if you want the columns to be balanced.
    \end{multicol*}

    % Use this to add a horizontal line.
    \horizontal

    \fi

    % Begin homework here.
    %%%%%%%%%%%%%%%%%%%%%%

    Consider an experiment in which we toss a pair of fair 4-sided dice, one red, one white. Let $X = |(\text{\# spots on Red}) - (\text{\# spots on White}) |$; that is, $X$ is the absolute value of the difference between the number of spots on the Red and White dice.
    
    \paragraph*{1.} Find the probability mass function (pmf) of $X$.
    Your final answer should be a table that has one column for the 5 possible values of $X$
    (i.e. 5 rows) and another column for the probability of obtaining each of these values.

    \begin{mdframed}
            \begin{equation*}
                X = \left\{
                    \begin{array}{r l}
                        X = 0 & = \{11, 22, 33, 44, 55, 66\}; \# = 6  \\
                        X = 1 & = \{12, 21, 32, 23, 34, 43, 54, 45, 65, 56\}; \# = 10  \\
                        X = 2 & = \{13, 31, 24, 42, 35, 53, 46, 64\}; \# = 8  \\
                        X = 3 & = \{14, 41, 25, 52, 36, 63\}; \# = 6  \\
                        X = 4 & = \{15, 51, 62, 26\}; \# = 4  \\
                        X = 5 & = \{16, 61\}; \# = 2  \\
                    \end{array}
                \right.
            \end{equation*}

            $\#X = X_0 + X_1 \dots X_5 = 36$

            \centering
            \begin{tabular}{c | c }
                $X$ & $p(x) = P(X = x)$    \\
                \hline
                0 & $\frac{6}{36}$  \\
                \hline
                1 & $\frac{10}{36}$ \\
                \hline
                2 & $\frac{8}{36}$  \\
                \hline
                3 & $\frac{6}{36} $ \\
                \hline
                4 & $\frac{4}{36}$  \\
                \hline
                5 & $\frac{2}{36}$  \\
            \end{tabular}
    \end{mdframed}

    \paragraph*{2.}
    Find the cumulative distribution function (cdf) of $X$ and sketch its graph.

    \begin{mdframed}
        \begin{multicols*}{2}
            \begin{equation*}
                F(x) = \left\{
                    \def\arraystretch{1.5}
                    \begin{array}{r l}
                        X = 0 & \text{ if}(\frac{6}{36} < x)    \\
                        X = 1 & \text{ if}(\frac{6}{36} \leq x < \frac{16}{36})  \\
                        X = 2 & \text{ if}(\frac{16}{36} \leq x < \frac{24}{36})  \\
                        X = 3 & \text{ if}(\frac{24}{36} \leq x < \frac{30}{36})  \\
                        X = 4 & \text{ if}(\frac{30}{36} \leq x < \frac{34}{36})  \\
                        X = 5 & \text{ if}(\frac{34}{36} \leq x \leq 1)  \\
                    \end{array}
                \right.
            \end{equation*}

            \columnbreak

            \begin{tikzpicture}[scale=0.9]
                \begin{axis}[
                    xmin=0, xmax=5.5,
                    ymin=0, ymax=1.1,
                    xlabel={$X$},
                    ylabel={$F(x)$}
                ]

                \draw (0,6/36) -- (1,6/36); \node[inner sep=0,circle,draw,fill=white,minimum size=5pt] at (1,6/36){};
                \draw (1,16/36) -- (2,16/36); \node[inner sep=0,circle,draw,fill=black, minimum size=5] at (1,16/36){};\node[inner sep=0,circle,draw,fill=white,minimum size=5pt] at (2,16/36){};
                \draw (2,24/36) -- (3,24/36); \node[inner sep=0,circle,draw,fill=black, minimum size=5pt] at (2,24/36){};\node[inner sep=0,circle,draw,fill=white,minimum size=5pt] at (3,24/36){};
                \draw (3,30/36) -- (4,30/36); \node[inner sep=0,circle,draw,fill=black, minimum size=5pt] at (3,30/36){};\node[inner sep=0,circle,draw,fill=white,minimum size=5pt] at (4,30/36){};           
                \draw (4,1) -- (5,1); \node[inner sep=0,circle,draw,fill=black, minimum size=5pt] at (4,1){};\node[inner sep=0,circle,draw,fill=black,minimum size=5pt] at (5,1){}; 
                    
                \end{axis}
            \end{tikzpicture}
        \end{multicols*}
    \end{mdframed}
\end{document}

