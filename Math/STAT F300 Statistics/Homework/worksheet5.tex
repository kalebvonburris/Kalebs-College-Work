% Files using this must be two subfolders
% deep. Adjust the number of ../ for the
% depth of the file.
% Imports
\usepackage{fancyhdr}
\usepackage{geometry}
\usepackage{icomma}
\usepackage{amsmath}
\usepackage{multicol}
\usepackage{mathptmx}
\usepackage{anyfontsize}
\usepackage{t1enc}
\usepackage{tabto}
\usepackage{listings}
\usepackage{filecontents}
\usepackage{subcaption}
\usepackage{tikz}
\usepackage[parfill]{parskip}
\usepackage{graphicx}
\usepackage[]{mdframed}
\usepackage{amsmath}
\usepackage[makeroom]{cancel}
\usepackage{pgfplots}
\usepackage{pgfplotstable}
\usepackage{xfrac}
\usepackage{amssymb}
\usepackage{mathtools}
\pgfplotsset{compat=1.18}
\usetikzlibrary{patterns}
\usepgfplotslibrary{polar}
\usepgfplotslibrary{fillbetween}

\geometry{margin=2.5cm}

\newcommand{\name}{Kaleb Burris}
\newcommand{\classname}{MATH F253, Elizabeth S. Allman, University of Alaska Fairbanks}
\newcommand{\assignment}{FILL IN ASSIGNMENT NAME}

\pagestyle{fancy}

\fancyhead[L]{
    \name 
    \newline
    \classname
    \newline
    \assignment
}

\newcommand{\horizontal}{\noindent\rule{\hsize}{0.4pt}}

\setlength{\headheight}{42pt}
\setlength{\headsep}{0.25in}
\setlength{\columnsep}{0.35cm}
\setlength{\columnseprule}{1pt}

\usepackage[T1]{fontenc}
\usepackage{lmodern}

% Put the assignment name with \S if 
% necessary for the section and the question 
% numbers.
\renewcommand\assignment{Worksheet 5, due Monday January 30, 4:15pm}

\def\firstcircle{(0:1.75cm) circle (2.5cm)}
\def\secondcircle{(180:1.75cm) circle (2.5cm)}

\begin{document}

    \paragraph*{1.}
    If $A$ and $B$ are events with probabilities $P (A) = 0.7$ and $P (B) = 0.9$, is it possible for $A$ and $B$ to be mutually exclusive? Why or why not?
    
    \begin{mdframed}
        No. $P(A) + P(B) = 1.6$, since probabilities are portions of 1.0, any sum of probabilities $> 1.0$ are either impossible or have to contain some elements of each other.

        If they were mutually exclusive, then $P(A \cap B) = \emptyset = P(\emptyset) = 0.0$.

        \begin{equation*}
            \begin{gathered}
                \text{The probability of} \quad P(A \cup B) = (P(A) + P(B)) - P(A \cap B) \quad \leq 1 \quad    \\
                (0.7 + 0.9) - 0 = 1.6 \quad \cancel{\leq} \quad 1
            \end{gathered}
        \end{equation*}
        
        
    \end{mdframed}

    \paragraph*{2.}
    In a standard 52 card deck of cards,

    \paragraph*{(a)}
    How many possible 5-card hands are there?

    \begin{mdframed}
        \begin{equation*}
            \begin{gathered}
                T1 = \text{pick 5 cards} = n_1  = C(52,5)  \\
                n = (n_1) = \left(\frac{52!}{5!(52-5)!}\right) = \left(\frac{52!}{5!(47)!}\right) = \left(\frac{(52)(51)(50)(49)(48)}{120}\right) = \boxed{2,598,960}
            \end{gathered}
        \end{equation*}
    \end{mdframed}

    \paragraph*{(b)}
    How many hands are there that have 3 of a kind (3 with the same numerical value) and 2 singletons?

    \begin{mdframed}
        \begin{equation*}
            \begin{gathered}
                T1 = \text{pick a \# for the three of a kind}     = n_1  = 13    \\
                T2 = \text{pick the three of a kind}  = n_2  = C(4,3)     \\
                T3 = \text{pick two \#s for singletons}  = n_3  = C(11, 2)   \\
                T4 = \text{pick higher singleton} = n_4 = C(4,1)    \\
                T5 = \text{pick lower singleton} = n_5 = C(4,1) \\
                n = (n_1)(n_2)(n_3)(n_4)(n_5) = (13)(\frac{4!}{3!(1!)})(\frac{11!}{2!(9!)})(4)(4)   \\
                = 13(4)(\frac{(11)(10)}{2})(4)(4) = \boxed{45,760}
            \end{gathered}
        \end{equation*}
    \end{mdframed}

    \paragraph*{(c)}
    What is the probability of getting a hand with 3 of a kind and 2 singletons?

    \begin{mdframed}
        \begin{equation*}
            \begin{gathered}
                \text{\# of possible hands}: \quad 2,598,960    \\
                \text{\# of special hands}: \quad 45,760        \\
                \text{Probability of special hands}: \quad \frac{45,760}{2,598,960} \approx \boxed{0.0176}
            \end{gathered}
        \end{equation*}
    \end{mdframed}

    \pagebreak

    \paragraph*{3.}
    Calculate each of the following.

    \paragraph*{(a)}
    Number of “words” consisting of 4 of the letter `F' and 11 of the letter `S'. For example, FSSSF SFSFS SSSSS is one such “word”. (Ignore the spaces; they're just there to make it easier to count the number of letters.)

    \begin{mdframed}
        We need every permutation of the set $A = \{F,F,F,F\} \cup \{S,S,S,S,S,S,S,S,S,S,S\}$.
        
        \begin{equation*}
            \begin{gathered}
                T1 = \text{Fill 4 positions with `F's} = n_1 = C(15,4)    \\
                T2 = \text{Fill the remaining with `S's} = n_2 = C(11, 11)
            \end{gathered}
        \end{equation*}
    \end{mdframed}

\end{document}