% Files using this must be two subfolders
% deep. Adjust the number of ../ for the
% depth of the file.
% Imports
\usepackage{fancyhdr}
\usepackage{geometry}
\usepackage{icomma}
\usepackage{amsmath}
\usepackage{multicol}
\usepackage{mathptmx}
\usepackage{anyfontsize}
\usepackage{t1enc}
\usepackage{tabto}
\usepackage{listings}
\usepackage{filecontents}
\usepackage{subcaption}
\usepackage{tikz}
\usepackage[parfill]{parskip}
\usepackage{graphicx}
\usepackage[]{mdframed}
\usepackage{amsmath}
\usepackage[makeroom]{cancel}
\usepackage{pgfplots}
\usepackage{pgfplotstable}
\usepackage{xfrac}
\usepackage{amssymb}
\usepackage{mathtools}
\pgfplotsset{compat=1.18}
\usetikzlibrary{patterns}
\usepgfplotslibrary{polar}
\usepgfplotslibrary{fillbetween}

\geometry{margin=2.5cm}

\newcommand{\name}{Kaleb Burris}
\newcommand{\classname}{MATH F253, Elizabeth S. Allman, University of Alaska Fairbanks}
\newcommand{\assignment}{FILL IN ASSIGNMENT NAME}

\pagestyle{fancy}

\fancyhead[L]{
    \name 
    \newline
    \classname
    \newline
    \assignment
}

\newcommand{\horizontal}{\noindent\rule{\hsize}{0.4pt}}

\setlength{\headheight}{42pt}
\setlength{\headsep}{0.25in}
\setlength{\columnsep}{0.35cm}
\setlength{\columnseprule}{1pt}

\usepackage[T1]{fontenc}
\usepackage{lmodern}

% Put class number, class name, and professor 
% name.
% Use only in case of emergency, this
% should be covered by the preamble.
% \renewcommand\classname{}

% Put the assignment name with \S if 
% necessary for the section and the question 
% numbers.
\renewcommand\assignment{Worksheet 15, Due February 27, 4:15pm}

\begin{document}
    % Templates
    \iffalse
    % Use these for equations.
    \begin{equation*}
        \begin{gathered}
            Equations go here.
        \end{gathered}
    \end{equation*}

    % Use this if a line of math is too long.
    \resizebox{\hsize}{!}{$Long equation goes here$}

    % Use these for multiple columns.
    \begin{multicol*}{# of columns}
        % Remove the * if you want the columns to be balanced.
    \end{multicol*}

    % Use this to add a horizontal line.
    \horizontal

    \fi

    % Begin homework here.
    %%%%%%%%%%%%%%%%%%%%%%

    The weights of adult green sea urchin are normally distributed, with a mean of 52.0 grams and standard deviation of 17.2 grams.

    \paragraph*{1.}
    Find the percentage of such sea urchins with weights between 50 g and 60 g.
    \\
    \begin{mdframed}
        \begin{align*}
            X               & \sim N(\mu=52, \sigma=17.2)           \\
            P(50 < X < 60)  & =  P\left(Z \leq \frac{60-\mu}{\sigma}\right) - P\left(Z \geq \frac{50-\mu}{\sigma}\right)    \\
                            & = \Phi\left(\frac{60-52}{17.2}\right) - \left( 1 - \Phi\left(\frac{50-52}{17.2}\right)\right)  \\
                            & = \Phi(0.465) - (1 - \Phi(-0.116)) \leftarrow \text{ round to -0.12} \\
                            & = \frac{0.6772+0.6808}{2} - (1 - 0.45224)                             \\
                            & = 0.679 - 0.5477 = \boxed{0.1313}
        \end{align*}
    \end{mdframed}

    \paragraph*{2.}
    What percentage weight over 40 g?
    \\
    \begin{mdframed}
        \begin{align*}
            P(X > 40)   & = P\left(Z \geq \frac{40-\mu}{\sigma}\right)                     \\
                        & = 1 - \Phi\left(\frac{40-52}{17.2}\right)             \\
                        & = 1 - \Phi(-0.6976) \leftarrow \text{ round to 0.7}   \\
                        & = 1 - 0.24196 = \boxed{0.758}
        \end{align*}
    \end{mdframed}

    \paragraph*{3.}
    Find the 90$^{th}$ percentile for weights and interpret this value.
    \\
    \begin{mdframed}
        \begin{align*}
            x   & = \mu + Z\sigma                   \\
            x   & = 52 + 1.281(17.2)                \\
                & = 52 + 22.033 = \boxed{54.033}
        \end{align*}
        The top 10\% in weight of adult green sea urchins are at least 54.033 g.
    \end{mdframed}

    \pagebreak

    \paragraph*{4.}
    Find the probability that in a sample of 16 adult green sea urchins, at least one will weigh over 75 grams.
    \\
    \begin{mdframed}
        \begin{align*}
            P(X > 75)   & = P(Z > \frac{75-\mu}{\sigma})                        \\
                        & = 1 - P(Z < \frac{75-52}{17.2})                       \\
                        & = 1 - \Phi(1.337) \leftarrow \text{ round to 1.34}    \\
                        & = 1 - 0.9099 = 0.0901                                 \\
            P(\text{At least 1 in 16} > 75)
                        & = 1 - (1-P(X > 75))^{16}                              \\
                        & = 1 - (1-0.0901)^{16}                                 \\
                        & = 1 - 0.7799 = \boxed{0.221}
        \end{align*}
    \end{mdframed}
    
\end{document}