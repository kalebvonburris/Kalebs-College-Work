% Files using this must be two subfolders
% deep. Adjust the number of ../ for the
% depth of the file.
% Imports
\usepackage{fancyhdr}
\usepackage{geometry}
\usepackage{icomma}
\usepackage{amsmath}
\usepackage{multicol}
\usepackage{mathptmx}
\usepackage{anyfontsize}
\usepackage{t1enc}
\usepackage{tabto}
\usepackage{listings}
\usepackage{filecontents}
\usepackage{subcaption}
\usepackage{tikz}
\usepackage[parfill]{parskip}
\usepackage{graphicx}
\usepackage[]{mdframed}
\usepackage{amsmath}
\usepackage[makeroom]{cancel}
\usepackage{pgfplots}
\usepackage{pgfplotstable}
\usepackage{xfrac}
\usepackage{amssymb}
\usepackage{mathtools}
\pgfplotsset{compat=1.18}
\usetikzlibrary{patterns}
\usepgfplotslibrary{polar}
\usepgfplotslibrary{fillbetween}

\geometry{margin=2.5cm}

\newcommand{\name}{Kaleb Burris}
\newcommand{\classname}{MATH F253, Elizabeth S. Allman, University of Alaska Fairbanks}
\newcommand{\assignment}{FILL IN ASSIGNMENT NAME}

\pagestyle{fancy}

\fancyhead[L]{
    \name 
    \newline
    \classname
    \newline
    \assignment
}

\newcommand{\horizontal}{\noindent\rule{\hsize}{0.4pt}}

\setlength{\headheight}{42pt}
\setlength{\headsep}{0.25in}
\setlength{\columnsep}{0.35cm}
\setlength{\columnseprule}{1pt}

\usepackage[T1]{fontenc}
\usepackage{lmodern}

% Put class number, class name, and professor 
% name.
% Use only in case of emergency, this
% should be covered by the preamble.
% \renewcommand\classname{}

% Put the assignment name with \S if 
% necessary for the section and the question 
% numbers.
\renewcommand\assignment{Worksheet 14, Due February 20, 4:15pm}

\begin{document}
    % Templates
    \iffalse
    % Use these for equations.
    \begin{equation*}
        \begin{gathered}
            Equations go here.
        \end{gathered}
    \end{equation*}

    % Use this if a line of math is too long.
    \resizebox{\hsize}{!}{$Long equation goes here$}

    % Use these for multiple columns.
    \begin{multicol*}{# of columns}
        % Remove the * if you want the columns to be balanced.
    \end{multicol*}

    % Use this to add a horizontal line.
    \horizontal

    \fi

    % Begin homework here.
    %%%%%%%%%%%%%%%%%%%%%%

    Let $X$ be a random variable whose value is the fraction of each can of diced tomatoes that is filled when it leaves the canning factory. The cdf for X is given by:
    
    \begin{equation*}
        F(x) = 
        \left\{
            \begin{array}{lll}
                0       & \quad  & \text{if } x < 0             \\
                x^{100} & \quad  & \text{if } 0 \leq x  < 1     \\
                1       & \quad  & \text{if } 1 \leq  x         \\
            \end{array}
        \right.
    \end{equation*}

    \paragraph*{1.}
    How do I know that $F(x)$ is a bona fide cdf? What do I need to check? Also, find the pdf of $X$.
    \\
    \begin{mdframed}
        You know $F(x)$ is a cdf if the slope of $F(x)$ is never negative as a cdf is the sum of the probabilitie before it. Since those probabilities cannot be zero, the cdf must either be staying the same or increasing. It also much approach 1 as $x \rightarrow \infty$ and it must approach 0 as $x \rightarrow -\infty$.

        For the pdf:

        \begin{align*}
            f(x)    & = F'(x)                   \\
                    & = \frac{d}{dx}\left[x^{100}\right]  \leftarrow \text{Skipping other domains as $\frac{d}{dx}$ of a constant is 0}  \\
                    & = 
                    \left\{
                        \begin{array}[pos]{lll}
                            \boxed{100x^{99}}   & \quad & \text{if } 0 < x < 1      \\
                            0                   & \quad & \text{otherwise} 
                        \end{array}
                    \right.
        \end{align*}
    \end{mdframed}

    \pagebreak

    \paragraph*{2.}
    Find $\mathbb{E}(X)$, var($X$) and sd($X$). State, but do not evaluate, the integral whose value is $\mathbb{E}(\log(X))$.
    \\
    \begin{mdframed}
        \begin{align*}
            \mathbb{E}(X)   & = \int_{0}^{1} xf(x)dx            \\
                            & = \int_{0}^{1} x \cdot 100x^{99}  \\
                            & = \int_{0}^{1} 100x^{100}         \\
                            & = \left.\left(\frac{100x^{100}}{101}\right)\right|_{0}^{1}
                              = \frac{100\cdot(1)^{100}}{101} - \frac{100\cdot(0)^{100}}{101}
                              = \frac{100}{101} \approx \boxed{0.99}
        \end{align*}

        \begin{align*}
            var(X) = \sigma^2_X & = \int_{0}^{1}\left[(x - \mu_X)^2\right]f(x)dx    \\
                                & = \int_{0}^{1}(x^2 - \mu_X + \mu_X^2)(100x^{99})dx\\
                                & = \int_{0}^{1}(x^2 - 0.0099)(100x^{99})dx         \\
                                & = \int_{0}^{1}(100x^{101} - 0.99x^{99})dx         \\
                                & = \left.\left(\frac{100x^{102}}{102} - \frac{0.99x^{100}}{100}\right)\right|_{0}^{1}
                                  = \frac{100\cdot(1)^{102}}{102} - \frac{0.99\cdot(1)^{100}}{100}   \\
                                & = \frac{100}{102} - \frac{99}{100} = 0.98039 - 0.0099 \approx \boxed{0.97049}
        \end{align*}

        \begin{align*}
            sd(X)   & = \sqrt{var(X)}   \\
                    & = \sqrt{0.97049}  \\
                    & \approx \boxed{0.985}
        \end{align*}
        
        For $\int_{a}^{b}f(x) = \mathbb{E}(\log(X))$:
        \begin{align*}
            \mathbb{E}(\log(X)) & = \int_{0}^{1}(\log(x)f(x))dx                         \\
                                & = \boxed{\int_{0}^{1}\left[\log\left(x^{f(x)}\right)\right]dx}
        \end{align*}
    \end{mdframed}

    \pagebreak

    \paragraph*{3.}
    There was a disaster. Find $P(X < 0.95)$
    \\
    \begin{mdframed}
        \begin{align*}
            P(X < 0.95) & = F(0.95)                 \\
                        & = 0.95^{100}              \\
                        & \approx \boxed{0.0059}    \\
            1000 \cdot 0.0059 = \boxed{5.9 \text{ cans}}
        \end{align*}
    \end{mdframed}

    \paragraph*{4.}
    What is the median of X? What is the 95th percentile of $X$?
    \\
    \begin{mdframed}
        Median of $X$:
        \begin{align*}
            F(a)& = 0.5
                & = a^{100} = 0.5 = \sqrt[100]{0.5} = \boxed{0.9931}
        \end{align*}

        95$^{\text{th}}$ percentile:
        \begin{align*}
            F(a)    & = 0.95                                  \\
                    & = a^{100} = 0.95 = \sqrt[100]{0.95} \approx \boxed{0.9995}
        \end{align*}
    \end{mdframed}
\end{document}