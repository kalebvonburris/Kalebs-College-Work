% Files using this must be two subfolders
% deep. Adjust the number of ../ for the
% depth of the file.
% Imports
\usepackage{fancyhdr}
\usepackage{geometry}
\usepackage{icomma}
\usepackage{amsmath}
\usepackage{multicol}
\usepackage{mathptmx}
\usepackage{anyfontsize}
\usepackage{t1enc}
\usepackage{tabto}
\usepackage{listings}
\usepackage{filecontents}
\usepackage{subcaption}
\usepackage{tikz}
\usepackage[parfill]{parskip}
\usepackage{graphicx}
\usepackage[]{mdframed}
\usepackage{amsmath}
\usepackage[makeroom]{cancel}
\usepackage{pgfplots}
\usepackage{pgfplotstable}
\usepackage{xfrac}
\usepackage{amssymb}
\usepackage{mathtools}
\pgfplotsset{compat=1.18}
\usetikzlibrary{patterns}
\usepgfplotslibrary{polar}
\usepgfplotslibrary{fillbetween}

\geometry{margin=2.5cm}

\newcommand{\name}{Kaleb Burris}
\newcommand{\classname}{MATH F253, Elizabeth S. Allman, University of Alaska Fairbanks}
\newcommand{\assignment}{FILL IN ASSIGNMENT NAME}

\pagestyle{fancy}

\fancyhead[L]{
    \name 
    \newline
    \classname
    \newline
    \assignment
}

\newcommand{\horizontal}{\noindent\rule{\hsize}{0.4pt}}

\setlength{\headheight}{42pt}
\setlength{\headsep}{0.25in}
\setlength{\columnsep}{0.35cm}
\setlength{\columnseprule}{1pt}

\usepackage[T1]{fontenc}
\usepackage{lmodern}

% Put class number, class name, and professor 
% name.
% Use only in case of emergency, this
% should be covered by the preamble.
% \renewcommand\classname{}

% Put the assignment name with \S if 
% necessary for the section and the question 
% numbers.
\renewcommand\assignment{Worksheet 16, Due March 3, 4:15pm}

\begin{document}
    % Templates
    \iffalse
    % Use these for equations.
    \begin{equation*}
        \begin{gathered}
            Equations go here.
        \end{gathered}
    \end{equation*}

    % Use this if a line of math is too long.
    \resizebox{\hsize}{!}{$Long equation goes here$}

    % Use these for multiple columns.
    \begin{multicol*}{# of columns}
        % Remove the * if you want the columns to be balanced.
    \end{multicol*}

    % Use this to add a horizontal line.
    \horizontal

    \fi

    % Begin homework here.
    %%%%%%%%%%%%%%%%%%%%%%

    \paragraph*{1.}\hbox{}
    \begin{enumerate}[label=(\alph*)]
        \item Find$ p_{Y|X}(y|1)$, and show that this is a pmf.
        \\
        \begin{mdframed}
            \begin{equation*}
                \begin{gathered}
                    \sum_x p_X(1) = 0.13 + 0.05 + 0.02 = 0.20   \\
                    p_{Y|X}(y|1) =  \begin{array}{c | c}
                        Y & p_{Y|X}(y|1)                        \\
                        \hline
                        0 & 0.13 / 0.20 = 0.65                  \\
                        \hline
                        1 & 0.05 / 0.20 = 0.25                  \\
                        \hline
                        2 & 0.02 / 0.20 = 0.10                  \\
                        \hline
                          & 0.65 + 0.25 + 0.10 = 1.00
                    \end{array}                                 \\
                    \sum_{Y}p_{Y|X}(y|1) = 1.00 \quad \therefore p_{Y|X}(y|1) \text { is a pmf}
                \end{gathered}
            \end{equation*}
        \end{mdframed}

        \item Find $\mathbb{E}(Y|1)$. Interpret this value.
        \\
        \begin{mdframed}
            \begin{align*}
                \mathbb{E}(Y|1) = \mathbb{E}(Y|X=1) & = \sum_{Y}y \cdot p_{x|y}(y|1)        \\
                                                    & = 0(0.65) + 1(0.25) + 2(0.10) = 0.25 + 0.2 = \boxed{0.45}
            \end{align*}
            Given that we went on a single hike with our dog on any given, the expected number of trips to Chicago that day is 0.45.
        \end{mdframed}

        \item Are $X$ and $Y$ independent? Explain.
        \\
        \begin{mdframed}
            \begin{equation*}
                \sum_{X} p_X(1) = 0.20; \quad \sum_{Y}(0) = 0.13 + 0.53 + 0.24 = 0.90
            \end{equation*}
            \begin{align*}
                p(X=1, Y=0) & \stackrel{?}{=} \sum_{X} p_X(1) \cdot \sum_{Y}(0)     \\
                0.13        & \stackrel{?}{=} 0.2 \cdot 0.90                        \\
                0.13        & \cancel{=} 0.18 \quad \therefore \text{ X and Y are dependent}
            \end{align*}
        \end{mdframed}
    \end{enumerate}

    \pagebreak

    \paragraph*{2.}
    \begin{enumerate}[label=(\alph*)]
        \item Find $f_{X|Y}(x|y)$ when $y > 5$ or $y < 3$.
        \\
        \begin{mdframed}
            \begin{align*}
                f_{X|Y}(x|y)    & = \frac{p(x,y)}{p(y)} (3 > y, y > 5) 
                                  = \frac{\frac{1}{30}x^{2}}{0} \quad \boxed{\text{always undefined}}
            \end{align*}
        \end{mdframed}

        \item Find $f_{X|Y}(x|y)$ when $3 < y < 5$. Show that this is a pdf.
        \\
        \begin{mdframed}
            \begin{align*}
                f_{X|Y}(x|y)    & = \frac{p(x,y)}{p(y)} (3 < y < 5)     \\
                                & = \begin{cases}
                                    \frac{\frac{1}{30}\left[3x^{2}+2\left(\frac{7}{15}+\frac{1}{15}y\right)\right]}{\frac{7}{15}+\frac{1}{15}y} & 1 < x < 2 \\
                                    0 & \text{otherwise}
                                \end{cases}
            \end{align*}
        \end{mdframed}

        \item Find $\mathbb{E}(X|4)$.
        \\
        \begin{mdframed}
            \begin{align*}
                \mathbb{E}(X|4) = \mathbb{E}(X|Y=4) & = \int_{1}^{2}f(x,4)dx                                
                                                    = \int_{1}^{2}\left[\frac{1}{30}(x^2+2(4))\right]dx     
                                                    = \int_{1}^{2}\left[\frac{1}{30}(x^2+8)\right]dx        \\
                                                    & = \frac{1}{30}\left[\frac{x^3}{3}+x\right]_{1}^{2}    
                                                    = \frac{1}{30}\left[\left(\frac{2^3}{3}+2\right) - \left(\frac{1^3}{3}+1\right)\right]  \\
                                                    & = \frac{1}{30}\left[\left(\frac{8}{3}+2\right) - \left(\frac{1}{3}+1\right)\right]
                                                    = \frac{1}{30}\left[\frac{7}{3}+1\right]
                                                    = \frac{1}{30}\left[\frac{9}{3}\right] = \frac{9}{90}   \\
                                                    & = \boxed{\frac{1}{10} = 0.10}
            \end{align*}
        \end{mdframed}
        
        \item Are $X$ and $Y$ independent? Explain.
        \\
        \begin{mdframed}
            \begin{align*}
                f_{X|Y}(x|y)    & \stackrel{?}{=} f_{X}(x) \cdot f_{Y}(y)   \\
                \frac{1}{30}(3x^{2}+2y) & \stackrel{?}{=} \left(\frac{1}{5}x^{2} + \frac{8}{15}\right)\left(\frac{7}{30} + \frac{1}{15}y\right) \\
                \frac{1}{30}(3x^{2}+2y) & \cancel{=} \left(\frac{7}{150}x^{2}+\frac{1}{15}x^{2}y+\frac{8}{225}y+\frac{8}{225}\right)    \\
                \cancel{=}      & \therefore \text{ $X$ and $Y$ are dependent}
            \end{align*}
        \end{mdframed}
    \end{enumerate}

\end{document}