% Files using this must be two subfolders
% deep. Adjust the number of ../ for the
% depth of the file.
% Imports
\usepackage{fancyhdr}
\usepackage{geometry}
\usepackage{icomma}
\usepackage{amsmath}
\usepackage{multicol}
\usepackage{mathptmx}
\usepackage{anyfontsize}
\usepackage{t1enc}
\usepackage{tabto}
\usepackage{listings}
\usepackage{filecontents}
\usepackage{subcaption}
\usepackage{tikz}
\usepackage[parfill]{parskip}
\usepackage{graphicx}
\usepackage[]{mdframed}
\usepackage{amsmath}
\usepackage[makeroom]{cancel}
\usepackage{pgfplots}
\usepackage{pgfplotstable}
\usepackage{xfrac}
\usepackage{amssymb}
\usepackage{mathtools}
\pgfplotsset{compat=1.18}
\usetikzlibrary{patterns}
\usepgfplotslibrary{polar}
\usepgfplotslibrary{fillbetween}

\geometry{margin=2.5cm}

\newcommand{\name}{Kaleb Burris}
\newcommand{\classname}{MATH F253, Elizabeth S. Allman, University of Alaska Fairbanks}
\newcommand{\assignment}{FILL IN ASSIGNMENT NAME}

\pagestyle{fancy}

\fancyhead[L]{
    \name 
    \newline
    \classname
    \newline
    \assignment
}

\newcommand{\horizontal}{\noindent\rule{\hsize}{0.4pt}}

\setlength{\headheight}{42pt}
\setlength{\headsep}{0.25in}
\setlength{\columnsep}{0.35cm}
\setlength{\columnseprule}{1pt}

\usepackage[T1]{fontenc}
\usepackage{lmodern}

\usepackage[]{mdframed}


% Put class number, class name, and professor 
% name.
\renewcommand\classname{STAT F300 Statistics, Dr. Short}

% Put the assignment name with \S if 
% necessary for the section and the question 
% numbers.
\renewcommand\assignment{Homework 1, Due Friday, January 20, 23:59}

\begin{document}

    % Templates
    \iffalse
    % Use these for equations.
    \begin{equation*}
        \begin{gathered}
            Equations go here.
        \end{gathered}
    \end{equation*}

    % Use this if a line of math is too long.
    \resizebox{\hsize}{!}{$Long equation goes here$}

    % Use these for multiple columns.
    \begin{multicol*}{# of columns}
        % Remove the * if you want the columns to be balanced.
    \end{multicol*}

    % Use this to add a horizontal line.
    \horizontal

    \fi

    % Begin homework here.
    %%%%%%%%%%%%%%%%%%%%%%

    \renewcommand{\pointsize}{12pt}

    \paragraph{1.} Name (print): \underline{Kaleb Burris}
    
    \paragraph{ } Signature: \underline{\emph{Kaleb Burris}}
        
    \paragraph{ } Student ID: \underline{31196184}
        
    \paragraph{ }    What is your primary field or area of study?
        \\ \text{\boxed{\emph{Computer Science.}}}
        
    \paragraph{ }    What interests you?
        \\ \text{\boxed{\emph{New ideas; things that make other things better.}}}
    

    \paragraph{2.} Do you have any special needs or concerns regarding this class that I should know about? If so, please describe briefly.
    \begin{mdframed}
        \emph{No.}
    \end{mdframed}
    
    \paragraph{3.} When was the most recent calculus course you've taken (semester/year)? Have you
    had Calc II? Calc III? (These aren't required for the course; I'm just trying to get an
    idea of how much Calc I refresher I'll need to include.)
    \begin{mdframed}
        \emph{I took Calculus I last semester and I'm taking Calculus II this semester.}
    \end{mdframed}

    \paragraph{} Please describe your computing experience, e.g. software packages you've worked with, languages you've written in, etc. Please include an estimate of how much time you've
    spent at each, for example, 1 hour, 10 hours, 100 hours, 1000 hours, etc.
    \begin{mdframed}
        \emph{I've used a lot of different packages and done a lot of programming, probably totaling over 2000+ hours. I've spent a decent amount of that time on recreational math problems and projects. My coding experience spans Java, C, C++, Python, Javascript, Rust, MathLab, and a little R. I'm currently writing this using} \LaTeX, \emph{which I picked up over this last winter break.}
    \end{mdframed}

    \paragraph{} Do you have a laptop you can bring to class for working problems that require more than a calculator?
    \begin{mdframed}
        \emph{Yes.}
    \end{mdframed}

    \paragraph{} Do you have a desktop or laptop computer you can use for (some) homework problems?
    \begin{mdframed}
        \emph{Yes.}
    \end{mdframed}

    \clearpage

    \paragraph{4.} Evaluate the following. Be sure to sure your work. In particular, if you apply a formula, you need to state the formula, then plug numbers into the formula, then state the final result.

    $\Rightarrow$ indicates the beginning of work. 
    \\
    You can use \verb+\sum\limits_{min}_^{max}+ to get limits for functions on inline text: $\sum\limits_{i=1}^{15}$
    
    \footnotesize{
    \begin{multicols}{2}
        \paragraph{(a)}
        \begin{equation*}
            \begin{gathered}
                \int 4x^3 \mathrm{d}x
                \\
                \Rightarrow \frac{4x^4}{4} + C = \boxed{x^4 + C}
            \end{gathered}
        \end{equation*}
        \horizontal

        \paragraph{(b)}
        \begin{equation*}
            \begin{gathered}
                \int_{1}^{2} 4x^3 \mathrm{d}x
                \\
                \Rightarrow x^4 \Big|_{1}^{2} = 2^4 - 1^4 
                \\
                = 16 - 1 = \boxed{15} 
            \end{gathered}
        \end{equation*}
        \horizontal

        \paragraph{(c)}
        \begin{equation*}
            \begin{gathered}
                \int \exp(-2x) \mathrm{d}x = \int e^{-2x} \mathrm{d}x
                \\
                \Rightarrow \boxed {-\frac{1}{2}e^{-2x} + C} \quad or \quad \boxed{-\frac{1}{2}\exp(-2x) + C}
            \end{gathered}
        \end{equation*}
        \horizontal

        \paragraph{(d)}
        \begin{equation*}
            \begin{gathered}
                \int_{0}^{2\pi}(1 + \sin(x))\mathrm{d}x
                \\
                \Rightarrow (x - \cos(x))\big|_{0}^{2\pi}
                \\
                = (2\pi - \cos(2\pi)) - (0 - \cos(0)) 
                \\
                = (2\pi - 1) - (-1) = \boxed{2\pi}
            \end{gathered}
        \end{equation*}

        \horizontal

        \paragraph{(e)}
        \begin{equation*}
            \begin{gathered}
                \sum_{i=1}^{50}(0.3i + 1)
                \\
                \text{Using} \quad \sum_{i=1}^{n}i = \frac{n(n+1)}{2}
                \\
                \Rightarrow \sum_{i=1}^{50}(0.3i + 1) = \sum_{i=1}^{50} 1 + 0.3\sum_{i=1}^{50}i
                \\
                = 50 + 0.3\left(\frac{50(50 + 1)}{2}\right) = 50 + 382.5
                \\
                = \boxed{432.5}
            \end{gathered}
        \end{equation*}

        \paragraph{(f)}
        \begin{equation*}
            \begin{gathered}
                \sum_{i=7}^{50}(0.3i + 1)
                \\
                \text{Using} \quad \sum_{i=1}^{n}i = \frac{n(n+1)}{2}
                \\
                \Rightarrow \sum_{i=1}^{50}(0.3i + 1) - \sum_{i=1}^{6}(0.3i + 1) = \sum_{i=7}^{50}(0.3i + 1)
                \\
                = \resizebox{0.95\hsize}{!}{$432.5 - \left(\sum\limits_{i=1}^{6}(0.3i + 1) = 432.5 - \sum\limits_{i=1}^{6} 1 + 0.3\sum\limits_{i=1}^{6}i\right)$}
                \\
                = 432.5 - \left(6 + 0.3\left(\frac{6(6 + 1)}{2}\right)\right) = 432.5 - 12.3
                \\
                = \boxed{420.2}
            \end{gathered}
        \end{equation*}
        \horizontal

        \paragraph{(g)}
        \begin{equation*}
            \begin{gathered}
                \sum_{i=1}^{15}(-0.2i^2 + 1)
                \\
                \text{Using} \quad \sum_{i=1}^{n}i^2 = \frac{1}{6}n(n+1)(2n+1)
                \\
                \Rightarrow \sum_{i=1}^{15} 1 - 0.2\sum_{i}^{15}i^2 = 15 - \frac{1}{6}15(15+1)(2(15)+1)
                \\
                = 15 - 248 = \boxed{-233}
            \end{gathered}
        \end{equation*}

        \horizontal

        \paragraph{(h)}
        \begin{equation*}
            \begin{gathered}
                \sum_{i=6}^{15}(0.5)^i
                \\
                \text{Using} \quad \sum_{i=0}^{n-1} ar^i = \frac{a-ar^n}{1-r}, r \neq 1
                \\
                \Rightarrow \sum_{i=6}^{15}(0.5)^i = \sum_{i=0}^{15}(0.5)^i - \sum_{i=0}^{5}(0.5)^i
                \\
                n + 1 \text{ used because of } n-1 \text{ in formula.}
                \\
                = \frac{1-1(0.5)^{16}}{1-0.5} - \frac{1-1(0.5)^6}{1-0.5} 
                \\
                \approx \boxed{0.061}
            \end{gathered}
        \end{equation*}

        \paragraph{(i)}
        \begin{equation*}
            \begin{gathered}
                \sum_{i=3}^{\infty}(0.8)^i
                \\
                \text{Using} \sum_{i=0}^{\infty}ar^i = \frac{a}{1-r}, -1 < r < 1
                \\
                \text{and} \sum_{i=0}^{n}ar^i = \frac{a - ar^n}{1 - r}, r \neq 1
                \\
                \Rightarrow \sum_{i=3}^{\infty}(0.8)^i = \sum_{i=0}^{\infty}(0.8)^i - \sum_{i=0}^{2}(0.8)^i
                \\
                = \frac{1}{1 - 0.8} - \frac{1 - 1(0.8)^2}{1 - 0.8} = 5 - 3.2 = \boxed{1.8}
            \end{gathered}
        \end{equation*}

    \end{multicols}
    }
    
\end{document}