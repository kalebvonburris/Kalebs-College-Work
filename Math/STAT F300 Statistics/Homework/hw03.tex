% Files using this must be two subfolders
% deep. Adjust the number of ../ for the
% depth of the file.
% Imports
\usepackage{fancyhdr}
\usepackage{geometry}
\usepackage{icomma}
\usepackage{amsmath}
\usepackage{multicol}
\usepackage{mathptmx}
\usepackage{anyfontsize}
\usepackage{t1enc}
\usepackage{tabto}
\usepackage{listings}
\usepackage{filecontents}
\usepackage{subcaption}
\usepackage{tikz}
\usepackage[parfill]{parskip}
\usepackage{graphicx}
\usepackage[]{mdframed}
\usepackage{amsmath}
\usepackage[makeroom]{cancel}
\usepackage{pgfplots}
\usepackage{pgfplotstable}
\usepackage{xfrac}
\usepackage{amssymb}
\usepackage{mathtools}
\pgfplotsset{compat=1.18}
\usetikzlibrary{patterns}
\usepgfplotslibrary{polar}
\usepgfplotslibrary{fillbetween}

\geometry{margin=2.5cm}

\newcommand{\name}{Kaleb Burris}
\newcommand{\classname}{MATH F253, Elizabeth S. Allman, University of Alaska Fairbanks}
\newcommand{\assignment}{FILL IN ASSIGNMENT NAME}

\pagestyle{fancy}

\fancyhead[L]{
    \name 
    \newline
    \classname
    \newline
    \assignment
}

\newcommand{\horizontal}{\noindent\rule{\hsize}{0.4pt}}

\setlength{\headheight}{42pt}
\setlength{\headsep}{0.25in}
\setlength{\columnsep}{0.35cm}
\setlength{\columnseprule}{1pt}

\usepackage[T1]{fontenc}
\usepackage{lmodern}

\graphicspath{ {./images/} }

% Put class number, class name, and professor 
% name.
\renewcommand\classname{STAT F300 Statistics, Dr. Short}

% Put the assignment name with \S if 
% necessary for the section and the question 
% numbers.
\renewcommand\assignment{Homework 3, Due Friday, February 3, 23:59}

\begin{document}

    % Templates
    \iffalse
    % Use these for equations.
    \begin{equation*}
        \begin{gathered}
            Equations go here.
        \end{gathered}
    \end{equation*}

    % Use this if a line of math is too long.
    \resizebox{\hsize}{!}{$Long equation goes here$}

    % Use these for multiple columns.
    \begin{multicol*}{# of columns}
        % Remove the * if you want the columns to be balanced.
    \end{multicol*}

    % Use this to add a horizontal line.
    \horizontal

    \fi

    % Begin homework here.
    %%%%%%%%%%%%%%%%%%%%%%

    \paragraph*{1.}
    Let $A$ and $B$ be events, where $A \subset B$. Show that $P (A) \leq P (B)$. (Hint: B is the disjoint union of $A \text{and} B \cap A^C$. Why? How can this be used to prove the desired result?)

    \begin{mdframed}
        Because $A \subset B, P(B) = P(A) + \text{a value}  \geq 0$. The value added must be at least 0 because a probability cannot be negative.
    \end{mdframed}

    \paragraph*{2.}
    If 60\% of all adults regularly consume coffee, 45\% regularly consume carbonated soda, and 80\% regularly consume at least one of these two products,

    \begin{enumerate}[label=(\alph*)]
        \item What is the probability that a randomly selected adult regularly consumes both coffee and soda?

        \begin{mdframed}
            \begin{align*}
                P(A) & = \{\text{People who regularly consume coffee}\} = 0.60 \\
                P(B) & = \{\text{People who regularly consume carbonated soda}\} = 0.45 \\
                P(C) & = \{\text{People who regularly consume one or the other}\} = 0.80 \\
            \end{align*}

            What is $P(A|B)$? What is $P(A \cap B)$?

            \begin{align*}
                P(C) & = P(A \cup B) \Rightarrow 0.80 = P(A \cup B)  \\
                P(A \cap B) & = (P(A) + P(B)) - P(A \cup B) = (0.60 + 0.45) - 0.80 = 1.05 - 0.8 = \boxed{0.25}
            \end{align*}
            
        \end{mdframed}

        \item What is the probability that a randomly selected adult consumes one or the
        other, but not both?

        \begin{mdframed}
            The question is; what is the sum of the probabilities of both $P(A)$ and $P(B)$, but not $P(A \cap B)$?

            \begin{align*}
                (P(A \cup B)) - P(A \cap B) = 0.80 - 0.25 = \boxed{0.55}
            \end{align*}
        \end{mdframed}
    \end{enumerate}

    \pagebreak

    \paragraph*{3.}
    (Devore \S 2.3 \# 39.) A box in a supply room contains 15 compact fluorescent lightbulbs, of which 5 are rated 13-watt, 6 are rated 18-watt, and 4 are rated 23-watt. Suppose that three of these bulbs are randomly selected.

    \begin{enumerate}[label=(\alph*)]
        \item What is the probability that exactly two of the selected bulbs are rated 23-watt?
        
        \begin{mdframed}
            \begin{align*}
                T1: & \text{ Choose 2 32-watt bulbs } = {4 \choose 2} = 6   \\
                T2: & \text{ Find the number of pairs of bulbs } = {15 \choose 2} = \frac{15!}{2!(15-2)!} = \frac{15!}{2!(13!)} = \frac{(15)(14)}{2} = \frac{210}{2} = 105    \\
                T3: & \text{ Find the probability of two 32-watt bulbs } = \frac{6}{105} \approx \boxed{0.057}
            \end{align*}
        \end{mdframed}

        \item What is the probability that all three of the bulbs have the same rating?
        
        \begin{mdframed}
            \begin{align*}
                    T1: & \text{ Choose 3 32-watt bulbs } = n_1 = {4 \choose 3} = 4   \\
                    T2: & \text{Choose 3 18-watt bulbs } = n_2 = {6 \choose 3} = 20   \\
                    T3: & \text{ Choose 3 13-watt bulbs}  = n_3 = {5 \choose 3} = 10   \\
                    T4: & \text{ Find the number of 3 sets of bulbs } = n_4 = {15 \choose 3} = 455 \\
                    T5: & \text{ Find the probability of having a set of 3 same-rating bulbs } = \frac{n_1 + n_2 + n_3}{n_4}    \\
                    & = \frac{4 + 20 + 10}{455} = \frac{34}{455} \approx \boxed{0.0747}
            \end{align*}
        \end{mdframed}

        \item What is the probability that one bulb of each type is selected?
        
        \begin{mdframed}
            \begin{align*}
                T1: & \text{ Choose 1 32-watt bulb } = n_1 = {4 \choose 1} = 4   \\
                T2: & \text{ Choose 1 18-watt bulb } = n_2 = {6 \choose 1} = 4   \\
                T3: & \text{ Choose 1 13-watt bulb } = n_3 = {5 \choose 1} = 5   \\
                T4: & \text{ Find the number of 3 sets of bulbs } = n_4 = {15 \choose 3} = 455 \\
                T5: & \text{ Find the probability of having each type } = \frac{(n_1)(n_2)(n_3)}{n_4} = \frac{(4)(5)(6)}{455}   \\
                & = \frac{120}{455} \approx \boxed{0.264}
            \end{align*}
        \end{mdframed}
    \end{enumerate}

    \pagebreak

    \paragraph*{4.}
    The game of Dottie Poker is played with a deck of cards that has 40 cars; there are four suits (clubs, hearts, spades, diamonds) with values Ace, 2, 3, \dots, 10. Hands consist of 4 cards.

    \begin{enumerate}[label=(\alph*)]
        \item How many possible hands are there?
        
        \begin{mdframed}
            \begin{align*}
                T1: & \text{ Take 4 cards for a hand } = {40 \choose 4} = \frac{40!}{4!(40-4)!} = \frac{40!}{4!(36!)} \\
                & = \frac{(40)(39)(38)(37)}{4!} = \frac{2,193,360}{24} = \boxed{91,390}   \\
            \end{align*}
        \end{mdframed}
    
        \item What is the probability of getting two pairs?
        
        \begin{mdframed}
            \begin{align*}
                T1: & \text{ Choose a number } = n_1 = {10 \choose 1} = 10              \\
                T2: & \text{ Choose a pair } = n_2 = {4 \choose 2} = 6                  \\
                T3: & \text{ Choose a second number } = n_3 = {9 \choose 1} =  9        \\
                T4: & \text{ Choose a second pairing } = n_4 = {4 \choose 2} = 6        \\
                T5: & \text{ Find the number of hands } = n_5 = {40 \choose 4} = 91,390 \\
                T4: & \text{ Find the probability of a twin-pair } = \frac{(n_1)(n_2)(n_3)(n_4)}{n_5} = \frac{(10)(6)(9)(6)}{91390} \\
                &  = \frac{3,240}{91,390} \approx \boxed{0.0355}
            \end{align*}
        \end{mdframed}
    
        \item What is the probability of getting 3 of one denomination and a singleton?
        
        \begin{mdframed}
            \begin{align*}
                T1: & \text{ Choose a number } = n_1 = {10 \choose 1} = 10              \\
                T2: & \text{ Choose a three of a kind } = n_2 = {4 \choose 3} = 4       \\
                T3: & \text{ Choose a singleton} = n_3 = {36 \choose 1} = 36            \\
                T4: & \text{ Find the number of hands } = n_4 = {40 \choose 4} = 91,390 \\
                T4: & \text{ Find the probability } = \frac{(n_1)(n_2)(n_3)}{n_4} = \frac{(10)(4)(36)}{91,390} = \frac{1,440}{91,390} \approx \boxed{0.0158}
            \end{align*}
        \end{mdframed}
    \end{enumerate}

    \pagebreak

    \paragraph*{5.}
    Dividing 8 players into two groups.

    \begin{enumerate}[label=(\alph*)]
        \item In how many ways can 8 players be divided into teams of size 3 and 5? Explain.
        
        \begin{mdframed}
            Because the players on either team are complementary, (once one team is made, the other is naturally made), you need only choose 3 players for one team, thus there are ${8 \choose 3} = \frac{8!}{3!(8-3)!} = \frac{8!}{3!(5)!} = \frac{(8)(7)(6)}{3!} = \frac{336}{24} = \boxed{14}$ possible arrangements of 3 and 5 teams.
        \end{mdframed}

        \item In how many ways can 8 players be divided into teams of size 4 and 4? Explain.
        
        \begin{mdframed}
            The exact same as before, there are ${8 \choose 4} = \frac{8!}{4!(8-4)!} = \frac{8!}{4!(4!)} = \frac{(8)(7)(6)(5)}{4!} = \frac{1,680}{24} = \boxed{70}$ arrangements of 4 and 4 teams.
        \end{mdframed}

    \end{enumerate}

\end{document}