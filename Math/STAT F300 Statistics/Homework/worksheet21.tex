% Files using this must be two subfolders
% deep. Adjust the number of ../ for the
% depth of the file.
% Imports
\usepackage{fancyhdr}
\usepackage{geometry}
\usepackage{icomma}
\usepackage{amsmath}
\usepackage{multicol}
\usepackage{mathptmx}
\usepackage{anyfontsize}
\usepackage{t1enc}
\usepackage{tabto}
\usepackage{listings}
\usepackage{filecontents}
\usepackage{subcaption}
\usepackage{tikz}
\usepackage[parfill]{parskip}
\usepackage{graphicx}
\usepackage[]{mdframed}
\usepackage{amsmath}
\usepackage[makeroom]{cancel}
\usepackage{pgfplots}
\usepackage{pgfplotstable}
\usepackage{xfrac}
\usepackage{amssymb}
\usepackage{mathtools}
\pgfplotsset{compat=1.18}
\usetikzlibrary{patterns}
\usepgfplotslibrary{polar}
\usepgfplotslibrary{fillbetween}

\geometry{margin=2.5cm}

\newcommand{\name}{Kaleb Burris}
\newcommand{\classname}{MATH F253, Elizabeth S. Allman, University of Alaska Fairbanks}
\newcommand{\assignment}{FILL IN ASSIGNMENT NAME}

\pagestyle{fancy}

\fancyhead[L]{
    \name 
    \newline
    \classname
    \newline
    \assignment
}

\newcommand{\horizontal}{\noindent\rule{\hsize}{0.4pt}}

\setlength{\headheight}{42pt}
\setlength{\headsep}{0.25in}
\setlength{\columnsep}{0.35cm}
\setlength{\columnseprule}{1pt}

\usepackage[T1]{fontenc}
\usepackage{lmodern}

% Put class number, class name, and professor 
% name.
% Use only in case of emergency, this
% should be covered by the preamble.
% \renewcommand\classname{}

% Put the assignment name with \S if 
% necessary for the section and the question 
% numbers.
\renewcommand\assignment{Worksheet 21, Due March 20, 4:15pm}

\begin{document}
    % Templates
    \iffalse
    % Use these for equations.
    \begin{equation*}
        \begin{gathered}
            Equations go here.
        \end{gathered}
    \end{equation*}

    % Use this if a line of math is too long.
    \resizebox{\hsize}{!}{$Long equation goes here$}

    % Use these for multiple columns.
    \begin{multicol*}{# of columns}
        % Remove the * if you want the columns to be balanced.
    \end{multicol*}

    % Use this to add a horizontal line.
    \horizontal

    \fi

    % Begin homework here.
    %%%%%%%%%%%%%%%%%%%%%%

    \begin{itemize}
        \item [1.]
        Calculate $\Gamma(5)$, $\Gamma(2)$, $\Gamma(3/2)$ and $\Gamma(7/2)$.
        \\
        \begin{mdframed}
            \begin{align*}
                \Gamma(5)   & = (5-1)! = 4! = \boxed{24}                            \\
                \Gamma(2)   & = (2-1)! = \boxed{1}                                  \\
                \Gamma(3/2) & \Rightarrow \Gamma(1/2) = \sqrt{\pi}, \quad 
                              \Gamma(\alpha) > 1 \text{ then } \Gamma(\alpha)
                              = (\alpha - 1)\Gamma(\alpha - 1)                      \\
                \Gamma(3/2) & = (3/2 - 1) \Gamma(3/2 - 1)                           \\
                            & = \boxed{\frac{1}{2}\sqrt{\pi}}                       \\
                \Gamma(7/2) & = (7/2-1)\Gamma(7/2-1) = (5/2)\Gamma(5/2)             \\
                            & = (5/2)(5/2-1)\Gamma(5/2-1) = (5/2)(3/2)\Gamma(3/2)   \\
                            & = (15/4)(1/2)\sqrt{\pi}                               \\
                            & = \boxed{\frac{15}{8}\sqrt{\pi}}
            \end{align*}
        \end{mdframed}

        \item[2.]
        A sample of asking prices for 2BR/2BA houses in two large U.S. cities were collected. Let $\overline{X_{n}}$ and $\overline{Y_{m}}$ be the sample average asking prices for the two cities. All prices are assumed to be independent, within cities and between cities. Let $\mu_{X}$ = 268 and $\mu_{Y}$ = 260 be the average values for the individual houses in the two cities (units are thousands of dollars), and let $\sigma_{X}$ = 22 and $\sigma_{Y}$ = 20 be the standard deviations (also in thousands of dollars).

        \begin{itemize}
            \item [(a)]
            Find the expected value and the variance of $\overline{X_{n}}$ - $\overline{Y_{m}}$. (Note that the sample sizes, n and m, need not be equal. (Your answer will have both n and m in it.)
            \\
            \begin{mdframed}
                \begin{align*}
                    \mathbb{E}(\overline{X}_{n} - \overline{Y}_{m})
                        & = \mathbb{E}(\overline{X}_{n}) - \mathbb{E}(\overline{Y}_{m}) \\
                        & = 268 - 260 = \boxed{8}                                       \\
                    \text{var}(\overline{X}_{n} - \overline{Y}_{m})
                        & = \text{var}(\overline{X}_{n})/n+(-1)^{2}\text{var}(\overline{Y}_{m})/m\\
                        & = \boxed{\frac{22}{n} + \frac{20}{m}}
                \end{align*}
            \end{mdframed}

            \item [(b)]
            If asking prices are normally distributed and the samples are of sizes n = 10 and m = 12, is there enough information to calculate $P(-0.2 \leq \overline{X}_{n} - \overline{Y}_{m} - 7.0 \leq 0.2)$? Why or why not? If yes, estimate the value.
            \\
            \begin{mdframed}
                \begin{align*}
                    \overline{X}_{10} - \overline{Y}_{12} \sim N\left(8,\frac{22}{10} 
                    + \frac{20}{12}\right) & \approx N(8, 0.1887)                       \\
                    P(-0.2 \leq \overline{X}_{n} - \overline{Y}_{m} - 7.0 \leq 0.2)
                        & = P()
                \end{align*}
            \end{mdframed}

            \pagebreak

            \item [(c)]
            If the distribution of the asking prices is unknown, and the samples are of sizes n = 36 and m = 25, is there enough information to calculate $P(-0.2 \leq \overline{X}_{n} - \overline{Y}_{m} - 7.0 \leq 0.2)$? Why or why not? If yes, estimate the value.
            \\
            \begin{mdframed}
                
            \end{mdframed}
            
            \item [(d)]
            If the distribution of the asking prices is unknown (not necessarily normal), and the samples are of sizes n = 36 and m = 49, is there enough information to calculate $P(-0.2 \leq \overline{X}_{n} - \overline{Y}_{m} - 7.0 \leq 0.2)$? Why or why not? If yes, estimate the value.
            \\
            \begin{mdframed}
                
            \end{mdframed}
        \end{itemize}

        \item[3.]
        Let $X_{1}, \cdots, X_{n}$ be independent Exponential($\theta$) random variables.

        \begin{itemize}
            \item [(a)]
            Show that $\overline{X}$ is biased for $\theta$. 
            \begin{mdframed}
                
            \end{mdframed}

            \item [(b)]
            Show that $\overline{X}$ is unbiased for 1/$\theta$.
            \begin{mdframed}
                
            \end{mdframed}
        \end{itemize}
    \end{itemize}

\end{document}