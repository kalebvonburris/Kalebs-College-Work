% Files using this must be two subfolders
% deep. Adjust the number of ../ for the
% depth of the file.
% Imports
\usepackage{fancyhdr}
\usepackage{geometry}
\usepackage{icomma}
\usepackage{amsmath}
\usepackage{multicol}
\usepackage{mathptmx}
\usepackage{anyfontsize}
\usepackage{t1enc}
\usepackage{tabto}
\usepackage{listings}
\usepackage{filecontents}
\usepackage{subcaption}
\usepackage{tikz}
\usepackage[parfill]{parskip}
\usepackage{graphicx}
\usepackage[]{mdframed}
\usepackage{amsmath}
\usepackage[makeroom]{cancel}
\usepackage{pgfplots}
\usepackage{pgfplotstable}
\usepackage{xfrac}
\usepackage{amssymb}
\usepackage{mathtools}
\pgfplotsset{compat=1.18}
\usetikzlibrary{patterns}
\usepgfplotslibrary{polar}
\usepgfplotslibrary{fillbetween}

\geometry{margin=2.5cm}

\newcommand{\name}{Kaleb Burris}
\newcommand{\classname}{MATH F253, Elizabeth S. Allman, University of Alaska Fairbanks}
\newcommand{\assignment}{FILL IN ASSIGNMENT NAME}

\pagestyle{fancy}

\fancyhead[L]{
    \name 
    \newline
    \classname
    \newline
    \assignment
}

\newcommand{\horizontal}{\noindent\rule{\hsize}{0.4pt}}

\setlength{\headheight}{42pt}
\setlength{\headsep}{0.25in}
\setlength{\columnsep}{0.35cm}
\setlength{\columnseprule}{1pt}

\usepackage[T1]{fontenc}
\usepackage{lmodern}

% Put class number, class name, and professor 
% name.
% Use only in case of emergency, this
% should be covered by the preamble.
% \renewcommand\classname{}

% Put the assignment name with \S if 
% necessary for the section and the question 
% numbers.
\renewcommand\assignment{Worksheet 22, Due March 22, 4:15pm}

\begin{document}
    % Templates
    \iffalse
    % Use these for equations.
    \begin{equation*}
        \begin{gathered}
            Equations go here.
        \end{gathered}
    \end{equation*}

    % Use this if a line of math is too long.
    \resizebox{\hsize}{!}{$Long equation goes here$}

    % Use these for multiple columns.
    \begin{multicol*}{# of columns}
        % Remove the * if you want the columns to be balanced.
    \end{multicol*}

    % Use this to add a horizontal line.
    \horizontal

    \fi

    % Begin homework here.
    %%%%%%%%%%%%%%%%%%%%%%

    \begin{itemize}
        \item [1.]
        Of $n$ randomly selected Star Wars fans, $X$ describe themselves as extroverts; whereas of $m$ randomly selected Star Trek fans, $Y$ are extroverts. Let $p_{1}$ and $p_{2}$ denote the probabilities that a randomly selected Star Wars fan and Star Trek fan, respectively, are extroverts.

        \begin{itemize}
            \item [(a)]
            Show that $T = (X/n - Y /m)$ is an unbiased estimator for $p_{X} - p_{Y}$.
            \begin{mdframed}
                \begin{align*}
                    p_{X}-p_{Y} & = T = (X/n - Y/m)                 \\
                    p_{1}       & = X/n                             \\
                    p_{2}       & = Y/n                             \\
                    \therefore  & \quad  p_{X} - p_{Y} = X/n - Y/m  \\
                    \therefore  & \quad \text{T is an unbiased estimator}
                \end{align*}
            \end{mdframed}

            \item [(b)]
            Find var($T$). Find the s.e. of the estimator in (a); i.e. find sd($T$).
            \begin{mdframed}
                \begin{align*}
                    \text{var}(T)   & = \text{var}(X/n - Y/m)
                                      = \text{var}(X/n) - \text{var}(Y/m)           \\
                                    & = \boxed{np_{X}(1-p_{X}) - mp_{Y}(1-p_{Y})}   \\
                    \text{s.e.}(T) = \text{sd}(T)& = \sqrt{\text{var}(T)}           \\
                                    & = \boxed{\sqrt{np_{X}(1-p_{X}) - mp_{Y}(1-p_{Y})}}
                \end{align*}
            \end{mdframed}

            \item [(c)]
            How would you use the observed values $x$ and $y$ to estimate the standard error of $T$?
            \begin{mdframed}
                $x$ and $y$ are the cases of extroverts, and so I would use them to calculate $p_{X}$ and $p_{Y}$ as $p_{X} = x/n$ and $p_{Y} = y/n$.
            \end{mdframed}

            \item [(d)]
            If n = m = 200, x = 127, and y = 176, use the estimator of (a) to obtain an estimate of $p_{X} - p_{Y}$.
            \begin{mdframed}
                \begin{align*}
                    T   & = X/n - Y/m   \\
                        & = 127/200 - 176/200 = 0.635 - 0.880 = \boxed{-0.245}
                \end{align*}
            \end{mdframed}

            \item [(e)]
            Use the result of part (c) and the data of part (d) to estimate the standard error of the estimator.
            \begin{mdframed}
                \begin{align*}
                    p_{X}           & = X/n = 0.635                                         \\
                    p_{y}           & = Y/m = 0.880                                         \\
                    \text{s.e.}(T)  & = \sqrt{np_{X}(1-p_{X}) - mp_{Y}(1-p_{Y})}            \\
                                    & = \sqrt{200(0.635)(1-0.635) - 200(0.880)(1-0.980)}    \\
                                    & = \sqrt{46.355 - 21.12} = \sqrt{25.235} \approx \boxed{5.0235}
                \end{align*}
            \end{mdframed}

        \end{itemize}
    \end{itemize}

\end{document}