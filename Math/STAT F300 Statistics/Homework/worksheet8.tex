% Files using this must be two subfolders
% deep. Adjust the number of ../ for the
% depth of the file.
% Imports
\usepackage{fancyhdr}
\usepackage{geometry}
\usepackage{icomma}
\usepackage{amsmath}
\usepackage{multicol}
\usepackage{mathptmx}
\usepackage{anyfontsize}
\usepackage{t1enc}
\usepackage{tabto}
\usepackage{listings}
\usepackage{filecontents}
\usepackage{subcaption}
\usepackage{tikz}
\usepackage[parfill]{parskip}
\usepackage{graphicx}
\usepackage[]{mdframed}
\usepackage{amsmath}
\usepackage[makeroom]{cancel}
\usepackage{pgfplots}
\usepackage{pgfplotstable}
\usepackage{xfrac}
\usepackage{amssymb}
\usepackage{mathtools}
\pgfplotsset{compat=1.18}
\usetikzlibrary{patterns}
\usepgfplotslibrary{polar}
\usepgfplotslibrary{fillbetween}

\geometry{margin=2.5cm}

\newcommand{\name}{Kaleb Burris}
\newcommand{\classname}{MATH F253, Elizabeth S. Allman, University of Alaska Fairbanks}
\newcommand{\assignment}{FILL IN ASSIGNMENT NAME}

\pagestyle{fancy}

\fancyhead[L]{
    \name 
    \newline
    \classname
    \newline
    \assignment
}

\newcommand{\horizontal}{\noindent\rule{\hsize}{0.4pt}}

\setlength{\headheight}{42pt}
\setlength{\headsep}{0.25in}
\setlength{\columnsep}{0.35cm}
\setlength{\columnseprule}{1pt}

\usepackage[T1]{fontenc}
\usepackage{lmodern}

% Put the assignment name with \S if 
% necessary for the section and the question 
% numbers.
\renewcommand\assignment{Worksheet 7, due Monday 6 February, 4:15pm}

\def\firstcircle{(0:1.75cm) circle (2.5cm)}
\def\secondcircle{(180:1.75cm) circle (2.5cm)}

\begin{document}

    \paragraph*{1.}
    Consider an experiment in which two fair 6-sided dice are tossed, one red, the other white.

    \begin{align*}
        A & = \text{ the the sum is at least 6} \\
        B & = \text{ the product is even} \\
        C & = \text{ red + 2 $\times$ white = 8} \\
        D & = \text{ red is odd} \\
    \end{align*}

    \begin{enumerate}[label=(\alph*)]
        \item Are events A and B independent? Be sure to show your work.
        
        \begin{mdframed}
            Is $P(A \cup B) = P(A)P(B)$?

            \begin{align*}
                \#P & = 6^2 = 36 \\
                A & = \{15,51,24,42,33\}    \\
                B & = \{12,21,22,14,41,16,61,23,32,44,46,64,66\}    \\
                A \cup B & = \{24,42\}                              \\
                P(A) & = \frac{\#A}{\#P} = \frac{5}{36} = 0.13\overline{8}    \\
                P(B) & = \frac{\#B}{\#P} = \frac{13}{36} = 0.36\overline{1}   \\
                P(A \cap B) & = \frac{\#(A \cap B)}{\#P} = \frac{2}{36} = 0.0\overline{5}   \\
                0.0\overline{5} & \stackrel{?}{=} \frac{0.13\overline{8}}{0.36\overline{1}} \\
                0.0\overline{5} & \cancel{\approx} 0.385  \\
                \cancel{=}, & \boxed{\therefore \text{ $A$ and $B$ are dependant.}}
            \end{align*}
        \end{mdframed}

        \item  Find a pair of mutually exclusive events. Why are they mutually exclusive?
        
        \begin{mdframed}
            $C$ and $D$ are mutually exclusive because the conditions of $C$ require that the red die rolled must be either 2, 4, or 6.
        
            The first die being red:
            \begin{align*}
                C & = \{22, 61\}  \\
                D & = \{1*,3*,5*\} \leftarrow \text{ Using star (*) to represent any digit from 1-6}    \\
                C \cap D & = \emptyset \quad \boxed{\therefore \text{ C and D are mutually exclusive.}}
            \end{align*}
        \end{mdframed}

        \pagebreak

        \item Are events $B^C$ and $C$ independent? Explain (i.e. show calculations and explain)
        
        \begin{mdframed}
            \begin{align*}
                B^C & = \{11,13,31,15,51,33,35,53,55\}  \\
                P(B^C) & = \frac{\#B^C}{\#P} = \frac{9}{36} = 0.25  \\
                B^C \cap C & = \emptyset   \\
                P(B^C \cap C) & \stackrel{?}{=} P(B^C)P(C)    \\
                \emptyset & \cancel{=} (0.25)(0.08\overline{3}), \boxed{\therefore \text{ $B^C$ and $C$ dependant.}}
            \end{align*}
        \end{mdframed}

        \item Find $P(D|C)$ and $P(D)$. Are $C$ and $D$ independent? Use just the values of
        $P(D|C)$ and $P(D)$ to answer this question. Explain briefly.

        \begin{mdframed}
            \begin{align*}                            \\
                D & = \{11,12\dots,16,31,32,\dots,36,51,52,\dots,56\}, \#D = 18     \\
                P(D) & = \frac{\#D}{\#P} = \frac{18}{36} = \boxed{0.5}              \\
                P(D|C) = D \cup C & = \boxed{\emptyset \therefore \text{ because of the solution to 1. (b), they are also independent.}}
            \end{align*}
        \end{mdframed}
    \end{enumerate}

    \paragraph*{2.}
    The dogs of the world can be divided into 4 types, according to whether they prefer chasing frisbees or digging in the dirt, and according to whether they do / don't enjoy long car rides. The table below summaries the probabilities that a randomly chosen dog has each combination of characteristics.

    Let $F = { prefers chasing frisbees }$ and $R = { likes road trips }$

    \begin{enumerate}[label=(\alph*)]
        \item Are $F$ and $R$ independent? Be sure to show your calculations.
        
        \begin{mdframed}
            \begin{align*}
                P(F) & = 0.37   \\
                P(R) & = 0.58   \\
                P(F \cap R) = 0.20  \\
                P(F \cap R) & \stackrel{?}{=} P(F)P(R)  \\
                0.2 & \stackrel{?}{=} (0.37)(0.58) \\
                0.2 & \cancel{=} 0.2146, \boxed{\therefore \text{ $F$ and $R$ are dependant.}}
            \end{align*}
        \end{mdframed}

        \pagebreak

        \item Are $F^C$ and $R^C$ independent? Be sure to sure your calculations.
        
        \begin{mdframed}
            \begin{align*}
                P(F^C) & = 1 - P(F) = 1 - 0.37 = 0.63           \\
                P(R^C) & = 1 - P(R) = 1 - 0.58 = 0.42           \\
                P(F^C \cap R^C) = 0.25                          \\
                P(F^C \cap R^C) & \stackrel{?}{=} P(F^C)P(R^C)  \\
                0.25 & \stackrel{?}{=} (0.63)(0.42)             \\
                0.25 & \cancel{=} 0.2646, \boxed{\therefore \text{$F^C$ and $R^C$ are dependant.}}
            \end{align*}
        \end{mdframed}
    \end{enumerate}


\end{document}