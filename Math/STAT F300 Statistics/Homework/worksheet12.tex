% Files using this must be two subfolders
% deep. Adjust the number of ../ for the
% depth of the file.
% Imports
\usepackage{fancyhdr}
\usepackage{geometry}
\usepackage{icomma}
\usepackage{amsmath}
\usepackage{multicol}
\usepackage{mathptmx}
\usepackage{anyfontsize}
\usepackage{t1enc}
\usepackage{tabto}
\usepackage{listings}
\usepackage{filecontents}
\usepackage{subcaption}
\usepackage{tikz}
\usepackage[parfill]{parskip}
\usepackage{graphicx}
\usepackage[]{mdframed}
\usepackage{amsmath}
\usepackage[makeroom]{cancel}
\usepackage{pgfplots}
\usepackage{pgfplotstable}
\usepackage{xfrac}
\usepackage{amssymb}
\usepackage{mathtools}
\pgfplotsset{compat=1.18}
\usetikzlibrary{patterns}
\usepgfplotslibrary{polar}
\usepgfplotslibrary{fillbetween}

\geometry{margin=2.5cm}

\newcommand{\name}{Kaleb Burris}
\newcommand{\classname}{MATH F253, Elizabeth S. Allman, University of Alaska Fairbanks}
\newcommand{\assignment}{FILL IN ASSIGNMENT NAME}

\pagestyle{fancy}

\fancyhead[L]{
    \name 
    \newline
    \classname
    \newline
    \assignment
}

\newcommand{\horizontal}{\noindent\rule{\hsize}{0.4pt}}

\setlength{\headheight}{42pt}
\setlength{\headsep}{0.25in}
\setlength{\columnsep}{0.35cm}
\setlength{\columnseprule}{1pt}

\usepackage[T1]{fontenc}
\usepackage{lmodern}

% Put class number, class name, and professor 
% name.
% Use only in case of emergency, this
% should be covered by the preamble.
% \renewcommand\classname{}

% Put the assignment name with \S if 
% necessary for the section and the question 
% numbers.
\renewcommand\assignment{Worksheet 1, Due February 15, 4:15pm}

\begin{document}
    % Templates
    \iffalse
    % Use these for equations.
    \begin{equation*}
        \begin{gathered}
            Equations go here.
        \end{gathered}
    \end{equation*}

    % Use this if a line of math is too long.
    \resizebox{\hsize}{!}{$Long equation goes here$}

    % Use these for multiple columns.
    \begin{multicol*}{# of columns}
        % Remove the * if you want the columns to be balanced.
    \end{multicol*}

    % Use this to add a horizontal line.
    \horizontal

    \fi

    % Begin homework here.
    %%%%%%%%%%%%%%%%%%%%%%

    The cost, X (in \$), of medical treatment for an individual suffering from a mythical malady (MM) in a mythical country has the pdf,

    \begin{equation*}
        f(x) = k \cdot \exp(-x/500), \quad 0 \leq x \leq 1000
    \end{equation*}

    for some positive constant, $k$.

    \paragraph*{1.}
    Find the normalizing contant, $k$.
    \\
    \begin{mdframed}
        \begin{align*}
            f(x)    & = k \cdot e^{(-x/500)}                                        \\
            F(x) = 1& = k \cdot \int_{0}^{1000}e^{(-x/500)}\mathrm{d}x              \\
                    & = k \cdot \left.\frac{-e^{(-x/500)}}{500}\right|_{0}^{1000}   \\
                    & = 500k \cdot (-e^{-2} + e^{0})                                \\
                    & = 500k \cdot (-\frac{1}{e^2} + 1)                             \\
                    & = 500k \cdot 0.865                                            \\
            k       & = \frac{1}{432.5} = \boxed{0.00231}
        \end{align*}
    \end{mdframed}

    \paragraph*{2.}
    Find the probability that the cost for treatment for MM is greater than \$800.
    \\
    \begin{mdframed}
        What is F(x) on $800 < x \leq 1000$?

        \begin{align*}
            P(800<x\leq1000)& = 0.00231 \cdot \int_{800}^{1000}e^{(-x/500)}\mathrm{d}x  \\
                            & = 0.00231 \cdot 500(-e^{(-x/500)}\Big|_{800}^{1000})      \\
                            & = 0.00231 \cdot 500(-e^{(-2)} - (- e^{(-\sfrac{8}{5})}))  \\
                            & = 0.00231 \cdot 500(0.202 - 0.135)                        \\
                            & = 0.00231 \cdot 500(0.067) = \boxed{0.0774}
        \end{align*}
    \end{mdframed}

    \pagebreak

    \paragraph*{3.}
    Find the general formula for $P(a \leq X \leq b)$, where $0 \leq a < b  \leq 1000$.
    \\
    \begin{mdframed}
        \begin{align*}
            P(a \leq X \leq b)  & = k \cdot \int_{a}^{b}e^{-x/500}\mathrm{d}x   \\
                                & = k \cdot 500(-e^{-x/500})\big|_{a}^{b}       \\
                                & = k \cdot 500(e^{-a/500} - e^{-b/500})        \\
                                & = \boxed{1.155\cdot (e^{-a/500} - e^{-b/500})}
        \end{align*}
    \end{mdframed}

    \paragraph*{4.}
    Find $P(10 \leq X \leq 100)$ and $P(10 < X < 100)$. Why are the answers the same?
    \\
    \begin{mdframed}
        \begin{align*}
            P(10 \leq X \leq 100)   & = 1.155 \cdot (e^{-10/500} - e^{-100/500})    \\
                                    & = 1.155 \cdot (0.98 - 0.819)                  \\
                                    & = 1.155 \cdot 0.161 = 0.1862
        \end{align*}

        \begin{align*}
            P(10 < X < 100)   & = 1.155 \cdot (e^{-10/500} - e^{-100/500})          \\
                                    & = 1.155 \cdot (0.98 - 0.819)                  \\
                                    & = 1.155 \cdot 0.161 = 0.1862
        \end{align*}

        They're the same because the inclusion/exclusion of any discrete number on a contininuous graph will result in no net change for the area underneath that graph. The integral for $\int_{0}^{0}f(x)$ will always be the same no matter what $f(x)$ is: 0.
    \end{mdframed}

\end{document}