% Files using this must be two subfolders
% deep. Adjust the number of ../ for the
% depth of the file.
% Imports
\usepackage{fancyhdr}
\usepackage{geometry}
\usepackage{icomma}
\usepackage{amsmath}
\usepackage{multicol}
\usepackage{mathptmx}
\usepackage{anyfontsize}
\usepackage{t1enc}
\usepackage{tabto}
\usepackage{listings}
\usepackage{filecontents}
\usepackage{subcaption}
\usepackage{tikz}
\usepackage[parfill]{parskip}
\usepackage{graphicx}
\usepackage[]{mdframed}
\usepackage{amsmath}
\usepackage[makeroom]{cancel}
\usepackage{pgfplots}
\usepackage{pgfplotstable}
\usepackage{xfrac}
\usepackage{amssymb}
\usepackage{mathtools}
\pgfplotsset{compat=1.18}
\usetikzlibrary{patterns}
\usepgfplotslibrary{polar}
\usepgfplotslibrary{fillbetween}

\geometry{margin=2.5cm}

\newcommand{\name}{Kaleb Burris}
\newcommand{\classname}{MATH F253, Elizabeth S. Allman, University of Alaska Fairbanks}
\newcommand{\assignment}{FILL IN ASSIGNMENT NAME}

\pagestyle{fancy}

\fancyhead[L]{
    \name 
    \newline
    \classname
    \newline
    \assignment
}

\newcommand{\horizontal}{\noindent\rule{\hsize}{0.4pt}}

\setlength{\headheight}{42pt}
\setlength{\headsep}{0.25in}
\setlength{\columnsep}{0.35cm}
\setlength{\columnseprule}{1pt}

\usepackage[T1]{fontenc}
\usepackage{lmodern}

\graphicspath{ {./images/} }

% Put the assignment name with \S if 
% necessary for the section and the question 
% numbers.
\renewcommand\assignment{Homework 4, Due Friday, February 10, 23:59}

\begin{document}

    % Templates
    \iffalse
    % Use these for equations.
    \begin{equation*}
        \begin{gathered}
            Equations go here.
        \end{gathered}
    \end{equation*}

    % Use this if a line of math is too long.
    \resizebox{\hsize}{!}{$Long equation goes here$}

    % Use these for multiple columns.
    \begin{multicol*}{# of columns}
        % Remove the * if you want the columns to be balanced.
    \end{multicol*}

    % Use this to add a horizontal line.
    \horizontal

    \fi

    % Begin homework here.
    %%%%%%%%%%%%%%%%%%%%%%

    \paragraph*{1.}
    \begin{mdframed}
        Given 300,000,000 people where 1000 are future terrorists, one is predicted to be a terrorist at a 99\% success rate. A person can be correctly identified as not a future terrorist at a 99.9\% success rate. What is the probability they are?

        \begin{align*}
            \#P & = 300,000,000  \\
            T & = \{\text{Future Terrorist}\}; \quad \#T = 1000; \quad P(T) = \frac{1000}{300,000,000} = \frac{1}{3,000,000} \approx 0.0000003 \\
            F & = \{\text{Flagged as Potential Future Terrorist}\}; \quad \#F = ?; \quad P(F) = \frac{\#F}{300,000,000}  \\
            \#F & = \#T(0.99) + \#P(0.001) = 990 + 300,000 = 300,990 \\
            P(F) & = \frac{\#F}{\#P} = \frac{300,990}{300,000,000} = 0.0010033  \\
            P(F|T) & = \frac{\#T(0.99)}{\#T} = \frac{990}{1000} = 0.99 \\
            P(T|F) & = \frac{P(T)}{P(F)}P(F|T) = \frac{0.0000003}{0.001033}(0.99) = (0.00323)(0.99) = 0.0032 = 0.32\%
        \end{align*}

        \boxed{\text{Among those flagged as future terrorists, 0.32\% actually are future terrorists.}}

        This does make me very uneasy; a system that generates a certainty of marking someone in a way that could ruin a person's life. Out of the population, 300,000 were marked as future terrorists which is extreme.
    \end{mdframed}

    \paragraph*{2.}
    \begin{mdframed}
        \begin{equation*}
            \begin{gathered}
                P(A \cap B) = P(A)P(B)    \\
                P(A^c \cap B) = (1-P(A))P(B) = P(B) - P(A)P(B)  \\
                = P(B) - P(A \cap B) \rightarrow P(A^c \cap B) + P(A \cap B) = P(B)\\
                \therefore P(A^c) \text{ and} P(B) \text{ are independent.}
            \end{gathered}
        \end{equation*}
    \end{mdframed}

    \pagebreak

    \paragraph*{3.}

    \begin{enumerate}[label=(\alph*)]
        \item Given a failure for each seam of 0.2, and each seam has at least 1 of 25 rivets failing, what is the probability of a rivet being defective?

        \begin{mdframed}
            \begin{align*}
                P(R) & = (1 - P(x))^{25}    \\
                P(S) & = 0.2 = (1 - P(R))   \\
                     & = 0.2 = 1 - (1 - P(x))^{25}\\
                     & = -0.8^{\frac{1}{25}} = 1-P(x)    \\
                     & = -0.991 = 1-P(x) = \boxed{0.0089} \\
            \end{align*}
        \end{mdframed}

        \item How small should the probability of a defective rivet be to ensure that only 10\% of all seams need reworking?
        
        \begin{mdframed}
            \begin{align*}
                P(R) & = (1 - P(x))^{25}    \\
                P(S) & = 0.1 = (1 - P(R))   \\
                     & = 0.1 = 1 - (1 - P(x))^{25}\\
                     & = -0.9^{\frac{1}{25}} = 1-P(x)    \\
                     & = -0.996 = 1-P(x) = \boxed{0.0042} 
            \end{align*}
        \end{mdframed}
    \end{enumerate}

    \paragraph*{4.}

\end{document}