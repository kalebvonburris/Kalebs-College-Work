% Files using this must be two subfolders
% deep. Adjust the number of ../ for the
% depth of the file.
\providecommand\pointsize{10pt}

\documentclass[\pointsize, letterpaper]{article}

% Imports
\usepackage{fancyhdr}
\usepackage{pgfplots}
\usepackage{geometry}
\usepackage{icomma}
\usepackage{amsmath}
\usepackage{multicol}
\usepackage{mathptmx}
\usepackage{anyfontsize}
\usepackage{t1enc}
\usepackage{tabto}
\usepackage{listings}
\usepackage{filecontents}
\usepackage{subcaption}
\usepackage{tikz}
\usepackage[parfill]{parskip}
\usepackage{graphicx}
\usepackage[]{mdframed}
\usepackage{amsmath}
\usepackage[makeroom]{cancel}
\pgfplotsset{compat=1.18}

\geometry{margin=2.5cm}

\newcommand{\name}{Kaleb Burris}
\newcommand{\classname}{MATH F252, Dr. J. Gimbel}
\newcommand{\assignment}{FILL IN ASSIGNMENT NAME}

\pagestyle{fancy}

\fancyhead[L]{
    \name 
    \newline
    \classname
    \newline
    \assignment
}

\newcommand{\horizontal}{\noindent\rule{\hsize}{0.4pt}}

\setlength{\headheight}{42pt}
\setlength{\headsep}{0.25in}
\setlength{\columnsep}{0.35cm}
\setlength{\columnseprule}{1pt}

% Put class number, class name, and professor 
% name.
% Use only in case of emergency, this
% should be covered by the preamble.
% \renewcommand\classname{}

% Put the assignment name with \S if 
% necessary for the section and the question 
% numbers.
\renewcommand\assignment{Worksheet 10, Due Friday, February 10, 4:15pm}

\begin{document}
    % Templates
    \iffalse
    % Use these for equations.
    \begin{equation*}
        \begin{gathered}
            Equations go here.
        \end{gathered}
    \end{equation*}

    % Use this if a line of math is too long.
    \resizebox{\hsize}{!}{$Long equation goes here$}

    % Use these for multiple columns.
    \begin{multicol*}{# of columns}
        % Remove the * if you want the columns to be balanced.
    \end{multicol*}

    % Use this to add a horizontal line.
    \horizontal

    \fi

    % Begin homework here.
    %%%%%%%%%%%%%%%%%%%%%%

    Let $Y$ denote the number of moving violations that a randomly selected customer of We Gotcha Insurance Company was cited for in the last three years. The pmf of $Y$ is given by

    \begin{center}
        \begin{tabular}{c | c c c c}
            $y$ & 0 & 1 & 2 & 3 \\
            \hline
            $p(y)$ & 0.65 & 0.23 & 0.09 & 0.03 
        \end{tabular}
    \end{center}

    \paragraph*{1.}
    Calculate and interpret the expected value of $Y$.

    \begin{mdframed}
        \begin{align*}
            \mathbb{E}(Y)   & = \sum_{y=0}^{3} y \cdot p(y)   \\
                            & = 0 + 0.23 + 0.18 + 0.09        \\
                            & = \boxed{0.50}                          \\
        \end{align*}

        The expected value of $Y$ is 0.5, meaning that on average a randomly selected customer was cited for 0.5 moving violations in the last three years.
    \end{mdframed}

    \paragraph*{2.}
    Find $\mathbb{E}(Y^2)$, var($Y$) and sd($Y$); interpret sd($Y$).

    \begin{mdframed}
        $\mathbb{E}(Y^2)$:

        \begin{align*}
            \mathbb{E}(Y^2) & = \sum_{y=0}^{3} y^2 \cdot p(y)   \\
                            & = 0 + 0.23 + 0.36 + 0.27          \\
                            & = \boxed{0.86}
        \end{align*}

        var$(Y) = \sigma^2_Y = \mathbb{E}(Y)^2 -(\mu_y)^2   \quad   \text{where} \quad  \mu_y = \mathbb{E}(Y)$

        \begin{align*}
            \mathbb{E}(Y^2) -(\mu_y)^2  & = 0.86 - (0.5)^2  \\
                                        & = 0.86 - 0.25     \\
                                        & \approx \boxed{0.61}
        \end{align*}

        sd($Y$):
        
        \begin{align*}
            \text{sd}(Y)    & = \sqrt{\text{var}(Y)}    \\
                            & = \sqrt{0.61}             \\
                            & = \boxed{0.78}
        \end{align*}
    
        The expected number of moving violations is 0.5 with a typical variation of about 0.78 tickets above or below the mean of 0.5.
    \end{mdframed}

    \paragraph*{3.}
    We Gotcha imposes a surcharge of $100Y^2$ dollars on customers with $Y$ many moving violations. Calculate the expected amount of the surcharge. Interpret your answer.

    \begin{mdframed}
        \begin{align*}
            \mathbb{E}(100Y^2)  & = 100 \cdot \mathbb{E}(Y^2)   \\
                                & = 100 \cdot 0.86              \\
                                & = \boxed{86}
        \end{align*}

        The expected surcharge is \$86. This means that on average, a random customer for the company pays \$86 in moving violations.
    \end{mdframed}
    
    \paragraph*{4.}
    Find the standard deviation of the surcharge.

    \begin{mdframed}
        Because the tickets and surcharge are related (Every ticket has a surcharge), we can use the standard deviation of $Y$.

        \begin{align*}
            \text{sd}(100Y^2)   & = 100 \cdot \text{sd}(Y)  \\
                                & = 100(0.78)               \\
                                & = \boxed{78}
        \end{align*}
    \end{mdframed}

\end{document}