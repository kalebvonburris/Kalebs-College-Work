% Files using this must be two subfolders
% deep. Adjust the number of ../ for the
% depth of the file.
\providecommand\pointsize{10pt}

\documentclass[\pointsize, letterpaper]{article}

% Imports
\usepackage{fancyhdr}
\usepackage{pgfplots}
\usepackage{geometry}
\usepackage{icomma}
\usepackage{amsmath}
\usepackage{multicol}
\usepackage{mathptmx}
\usepackage{anyfontsize}
\usepackage{t1enc}
\usepackage{tabto}
\usepackage{listings}
\usepackage{filecontents}
\usepackage{subcaption}
\usepackage{tikz}
\usepackage[parfill]{parskip}
\usepackage{graphicx}
\usepackage[]{mdframed}
\usepackage{amsmath}
\usepackage[makeroom]{cancel}
\pgfplotsset{compat=1.18}

\geometry{margin=2.5cm}

\newcommand{\name}{Kaleb Burris}
\newcommand{\classname}{MATH F252, Dr. J. Gimbel}
\newcommand{\assignment}{FILL IN ASSIGNMENT NAME}

\pagestyle{fancy}

\fancyhead[L]{
    \name 
    \newline
    \classname
    \newline
    \assignment
}

\newcommand{\horizontal}{\noindent\rule{\hsize}{0.4pt}}

\setlength{\headheight}{42pt}
\setlength{\headsep}{0.25in}
\setlength{\columnsep}{0.35cm}
\setlength{\columnseprule}{1pt}

% Put the assignment name with \S if 
% necessary for the section and the question 
% numbers.
\renewcommand\assignment{Worksheet 4, due Friday January 27, 4:15pm}

\def\firstcircle{(0:1.75cm) circle (2.5cm)}
\def\secondcircle{(180:1.75cm) circle (2.5cm)}

\begin{document}

    \paragraph*{1.}
    If A and B are events such that $P (A) = 0.10$, $P (B) = 0.60$, and $P(A \cup B) = 0.64$, then:

    \paragraph*{(a)}
    Sketch and label a Venn diagram appropriately. (That is, your diagram should have the overlapping events, with 4 probabilities written in the correct places and these 4 numbers must sum to 1.) Please ask me if you are unsure what I'm asking for.

    \begin{mdframed}
        \resizebox{\hsize}{!}{\begin{tikzpicture}
            \draw \firstcircle node[text=black,scale=0.5] {$P(A) = 0.1$};
            \draw \secondcircle node [text=black, scale=0.5] {$P(B) = 0.6$};
            \path (0,0) node[scale=0.5] {$P(A \cap B) = 0.06$}
                  (0,-3) node[scale=0.5]{$P(A^c \cup B^c) = 0.36$};
        \end{tikzpicture}}

        \begin{align*}
            P(A^c \cap B^c) & = 1 - (P(A \cup B)) = 1 - 0.64 = 0.36 \\
            P(A \cap B) & = (P(A) + P(B)) - P(A \cup B) = (0.1 + 0.6) - 0.64 = 0.06
        \end{align*}
    \end{mdframed}

    \paragraph*{(b)}
    Find $P(A \cap B)$.
    \begin{mdframed}
        \begin{equation*}
            \begin{gathered}
                P(A) = 0.1, P(B) = 0.6, P(A \cup B) = 0.64    \\
                P(A \cap B) = (P(A) + P(B)) - P(A \cup B) = (0.1 + 0.6) - 0.64 = \boxed{0.06}
            \end{gathered}
        \end{equation*}
    \end{mdframed}

    \paragraph*{(c)}
    Find $P(B \cap A^c)$.

    \begin{mdframed}
        \begin{equation*}
            \begin{gathered}
                P(B \cap A^c) = P(B) - P(A \cap B)  \\
                P(B \cap A^c) = 0.1 - 0.06 = \boxed{0.04}
            \end{gathered}
        \end{equation*}
    \end{mdframed}

    \pagebreak

    \paragraph*{2.}
    Consider an experiment in which we roll two 4-sided dice, one red, one green. Let $A$ be the event that the red die is 2; let $B$ be the event that the sum is at most 4, and let $C$ be the event that the product is odd.

    \paragraph*{(a)}
    Find $P(A \cup B)$. Be sure to show your work; for example, $A \cup B = \{ \}$, all the outcomes are equally likely, therefore $P (A \cup B) = \dots$

    \begin{mdframed}
        \begin{equation*}
            P = \left\{
                \begin{aligned}
                    &\{1,1\}, \{1,2\}, \{1,3\}, \{1,4\}, \\
                    &\{2,1\}, \{2,2\}, \{2,3\}, \{2,4\}, \\
                    &\{3,1\}, \{3,2\}, \{3,3\}, \{3,4\}, \\ 
                    &\{4,1\}, \{4,2\}, \{4,3\}, \{4,4\}
                \end{aligned}
                \right\} \quad \# P = 16
        \end{equation*}
        \begin{equation*}
            A = \left\{\{2,1\}, \{2,2\}, \{2,3\}, \{2,4\}\right\} \quad \#A = 4 \quad P(A) = 0.25
        \end{equation*}

        \begin{equation*}
            B = \left\{
                \begin{aligned}
                    &\{1,1\}, \{1,2\}, \{1,3\}, \{2,1\},    \\
                    &\{2,2\}, \{3,1\}
                \end{aligned}
            \right\} \quad \#B = 6 \quad P(B) = 0.375
        \end{equation*}

        \begin{equation*}
            P(A \cup B) = \left\{
                \begin{aligned}
                    &\{1,1\}, \{1,2\}, \{1,3\}, \{2,1\},    \\
                    &\{2,2\}, \{2,3\}, \{2,4\}, \{3,1\}
                \end{aligned}
            \right\} \quad \#(A \cup B) = 8 \quad \boxed{P(A \cup B) = 0.5}
        \end{equation*}
    \end{mdframed}

    \paragraph*{(b)}
    Find $P(B \cap A^c)$

    \begin{mdframed}
        \begin{equation*}
            P(B \cup A^c) = \frac{\#\left\{\{1,1\}, \{1,2\}, \{1,3\}, \{3,1\}\right\}}{16} = \frac{4}{16} = \boxed{0.25}
        \end{equation*}
    \end{mdframed}

    \paragraph*{3.}

    \paragraph*{(a)}
    $\boxed{\text{True}}$ / False: If $A = \emptyset$ then $P (A) = 0$.

    \paragraph*{(b)}
    True / $\boxed{\text{False}}$: If $P (A) = 0$ then $A = \emptyset$; if true, state why; if false, give an example that shows the statement is false. (Hint: The answer is False.)

    \begin{mdframed}
        If $A$ contains elements of an uncountably infinite set, then any single element $n$ will have $P(n) = 0$.
    \end{mdframed}

\end{document}