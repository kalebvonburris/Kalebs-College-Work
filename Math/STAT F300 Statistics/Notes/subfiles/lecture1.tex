\documentclass[../Notes.tex]{subfiles}

\begin{document}
    (\S 1.2) Descriptive statistics; (\S 1.3) Measures of location

    Common naming conventions:
    \begin{itemize}
        \item Population size: $N$
        \item Sample size: $n$
        \item Sample from two different populations: $n, m,$ or $n_1, n_2$
        \item Data: $x_1, x_2, x_3, \dots x_n$
    \end{itemize}

    Stem-and-leaf displays using R:

\begin{lstlisting}[language=R]
> x sample(1:50, size=20, replace=TRUE)
> sort(x)
    [1]: 2 2 2 3 9 14 18 19 20 21 21 22 22 29 30 32 32
    [18]: 33 44 47
> stem(x)
\end{lstlisting}

    The sample function generates numbers in the range provided as the first argument, with a size equal to the second argument. sort(x) sorts the values stored in x, and stem(x) does the following:

    Each ``stem'' refers to the highest digits and each ``leaf'' is the latter digits. This is the stem-and-leaf display for the dataset stored in x:

    \begin{center}
        \resizebox{0.35\hsize}{!}{
            \begin{tabular}{r | l}
                Stem & Leaves       \\
                \hline              
                0 & 2 2 2 3 9       \\
                1 & 4 8 9           \\
                2 & 0 1 1 2 2 9     \\
                3 & 0 2 2 3         \\
                4 & 4 7
            \end{tabular}
        }
    \end{center}

    Endpoint: 8:44.

\end{document}