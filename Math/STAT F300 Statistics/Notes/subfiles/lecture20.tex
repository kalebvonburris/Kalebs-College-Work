\documentclass[../Notes.tex]{subfiles}

\begin{document}
    \subsection*{\S 4.4 Exponential Distributions}

    \paragraph*{The Exponential Distribution.} The random variable X has an exponential distribution with facts:

    \begin{align}
        X               & \sim \text{exp}(\lambda), \quad \lambda > 0                       \\
        f(x, \lambda)   & =
                                \begin{cases}
                                    \lambda\cdot e^{-\lambda x} & \text{if } x > 0          \\
                                    0   & \text{otherwise}
                                \end{cases}                                                 \\
        \mathbb{E}(X)   & = \frac{1}{\lambda}                                               \\
        \text{var}(X)   & = \frac{1}{\lambda^{2}}                                           
    \end{align}

    \subsection*{\S 4.6 Normal Probability Plots}

    ``Normal probability plots'' are used to check an assumption of normality.

    A new type of notation: If $x_{1}, x_{2}, \cdots, x_{n}$ is a list of numbers, we define $x_{(1)}$ to be the smallest, $x_{(2)}$ to be the smallest, and so on. $x_{(n)}$ is the largest.

    \underline{Example:} \{4, 1, 7, 12, 3, 9\} $\rightarrow x_{(1)} = 1, x_{(2)} = 3, \cdots x_{(6)} = 12$

    Normal probability plots have one point per observation: for the $i^{th}$ point,
    \begin{itemize}
        \item the $x$-coordinate is $\mathbb{E}(Z_{(i)})$ where $Z \sim N(0,1)$, and
        \item the $y$-coordinate is $x_{(i)}$, the $i^{th}$ smallest data value.
    \end{itemize}

    We'll use R to create the normal probability plots. Here's how we'll interpret the points:

    If the points in the normal probability plot lie on the (approximately) straight line, then the data are consistent with an underlying normal population.

    Otherwise, the data are inconsistent with an underlying normal population.

    
\end{document}