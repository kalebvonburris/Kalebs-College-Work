\documentclass[10pt, letterpaper]{paper}

% Imports
\usepackage{fancyhdr}
\usepackage{pgfplots}
\usepackage{geometry}
\usepackage{icomma}
\usepackage{amsmath}
\usepackage{multicol}
\usepackage[parfill]{parskip}
\usepackage{subfiles}
\usepackage[makeroom]{cancel}
\usepackage{mathtools}
\usepackage{listings}
\usepackage{amssymb}
\usepgfplotslibrary{statistics}
\pgfplotsset{compat=1.18}
\usepackage{parskip}
\setlength{\parindent}{15pt}

\geometry{margin=2.5cm}

\addtolength{\hoffset}{-0.75in}
\addtolength{\textwidth}{110pt}

\pagestyle{fancy}

\fancyhead[L]{
    Kaleb Burris
    \newline
    STAT F300
}

\fancyhead[R]{
    Cheat-Sheet :)
    \newline
    Good luck!
}

\setlength{\headheight}{15pt}
\setlength{\headsep}{0.25in}

\newcommand{\horizontal}{\noindent\rule{\hsize}{0.4pt}}

\setlength{\columnsep}{20pt}
\setlength{\columnseprule}{1pt}

\begin{document}

    %\setlength{\belowdisplayskip}{0pt} \setlength{\belowdisplayshortskip}{0pt}
    %\setlength{\abovedisplayskip}{0pt} \setlength{\abovedisplayshortskip}{0pt}

    \begin{multicols*}{2}
        \paragraph*{Fundamentals}\mbox{}

        Sample Mean: $\overline{x} = x_1 + x_2 + \cdots + x_n = \frac{1}{n}\sum\limits_{i=1}^{n}x_i$
        
        Sample Median: Sort values in increasing order, then: 
        $$
            \tilde{x} = 
            \left\{
                \begin{array}{ll}
                    \text{Middle value}                 & \text{If n is odd}    \\
                    \text{Average of two middle values} & \text{If n is even}
                \end{array}
            \right.
        $$

        Sample Range: $\text{Range}(x) = \text{Max}(x) - \text{Min}(x)$

        Population mean: $\mu = \frac{1}{N}\sum\limits_{i=1}^{N}x_i$
        
        Population median: $\tilde{\mu} = \text{median of } \{x_1, x_2, \cdots, x_n\}$

        Sample Variance: $S^2 \text{ or } \sigma^{2} = \frac{1}{n-1}\sum\limits_{i=1}^{n}(x_i-\overline{x})^2$

        Standard Deviation: $\text{sd}(x) = \sqrt{\sigma^2} = \sigma$

        Factorial: $!n = n \times (n-1) \times (n-2) \times \cdots \times 1$

        Stem and Leaf Plots:

        $\rightarrow$ Each ``stem'' refers to the highest digits and each ``leaf'' is the lowest digit. This is the stem-and-leaf plot for: 

        \begin{center}
            \{2 2 2 3 9 14 18 19 20 21 21 22 22 29 30 32 32 112\}
        
            \resizebox{0.5\hsize}{!}{
                \begin{tabular}{r|l}
                    Stem & Leaves       \\
                    \hline              
                    0 & 2 2 2 3 9       \\
                    1 & 4 8 9           \\
                    2 & 0 1 1 2 2 9     \\
                    3 & 0 2 2 3         \\
                    4 & 4 7             \\
                    5 &                 \\
                    \vdots              \\
                    11 & 2
                \end{tabular}
            }
        \end{center}

        Dot Plots:

        $\rightarrow$ Each ``dot'' over the x-axis references a single instance of that x-value in the set, this is the dot plot for:

        \begin{center}
            \{7 8 1 3 4 10 1 2 2 1 1 4 4 9 6\}

            \begin{tikzpicture}
                \begin{axis}[
                    title={Dot Plot},
                    xmin=0, xmax=11,
                    ymin=0, ymax=10,
                    yticklabels={,,},
                    xtick={1,2,3,4,5,6,7,8,9,10},
                    width=0.45\textwidth,
                    height=4cm
                ]
                
                \addplot[only marks, scatter src=explicit symbolic
                    ]
                    table[row sep=\\] {
                        x y \\
                        1 1 \\
                        1 2 \\
                        1 3 \\
                        1 4 \\
                        2 1 \\
                        2 2 \\
                        3 1 \\
                        4 1 \\
                        4 2 \\
                        6 1 \\
                        7 1 \\
                        8 1 \\
                        9 1 \\
                        10 1\\
                    };
                    
                \end{axis}
            \end{tikzpicture}
        \end{center}

        \columnbreak

        Box Plots:

        $\rightarrow Q_{1} = \text{First Quartile} = \text{median of the smallest half of values}$

        $\rightarrow Q_{3} = \text{Second Quartile} = \text{median of largest half of values}$

        $\rightarrow IQR = f_{s} = \text{Fourth Spread} = Q_{3} - Q_{1}$

        Here's the recipe for constructing a boxplot (``cat and whisker plot''):

        \begin{enumerate}
            \item Draw a horizontal line that extends from the smallest to largest values in your data set.
            \item Draw a rectangle with vertical lines at $Q_1$, $Q_2$, and $Q_3$. ($Q_2$ = median)
            \item If $x_i Q_1 - 1.5 * IQR$ or $x_i > Q_3 + 1.5 * IQR$, then $x_i$ is considered an outlier. Put a dot at the locations of outliers.
            \item Draw whiskers that extend from the rectangle to the most extreme non-outlying observation.
        \end{enumerate}

        \underline{Example:} $\{1,5,7,18,20,22,50\}$, use dots instead of bars

        \begin{center}
            \begin{tikzpicture}
                \begin{axis}
                [
                    yticklabels={,,}
                ]
                \addplot[
                boxplot prepared={
                    median=19,
                    upper quartile=20,
                    lower quartile=6,
                    upper whisker=22,
                    lower whisker=1
                },
                ] coordinates {};
                \end{axis}
            \end{tikzpicture}
        \end{center}
        
        \horizontal
        
        \paragraph*{Probabilities}\mbox{}

        Events: $A$, an event or set of events

        Subset: $n \subset A$, a set made of elements in $A$

        Probability: $P(A)$, the probability of event $A$ occuring

        $\rightarrow \sum\limits_{x} P(x) = 1$

        Cardinality: $\#A$, the number of elements in set $A$

        Combination: $C(\#A, n) / {\#A \choose n} = \dfrac{\#A!}{n!(\#A-n)!}$

        Permutation: $P(\#A, n) = \dfrac{\#A}{(\#A - n)!}$

        Given: $P(A|B)$, the probability of $A$ given $B$

        $\rightarrow P(A|B) = \dfrac{P(A \cap B)}{P(B)}$

        Independent if: $P(A \cap B) = P(A) \times P(B)$

        Mutually Exclusive if: $P(A \cap B) = \emptyset$

        Bayes' Theorem: $P(A|B) = \dfrac{P(B|A)P(A)}{P(B)}$

        \horizontal

        \paragraph*{Distributions}\mbox{}

        Expected Value: $\mathbb{E}(X)$, the sum of the value $\times$ probability of each element in a distribution

        Probability Mass Function: $p(x) = P(X = x)$
        
        $\rightarrow p(x = X) = 
            \left\{
                \begin{array}{lcl}
                    p(x_{1}) & \quad & \text{ if conditional}   \\
                    p(x_{2}) & \quad & \text{ if conditional}   \\
                    \vdots
                    &&                                          \\
                    p(x_{n}) & \quad & \text{ if conditional}   \\
                \end{array}
            \right.
        $

        $\rightarrow \sum\limits_{x}p(x) = 1$

        Cumulative Distribution Function: $F_{X} = P(X \leq x)$

        $\rightarrow P(a < X \leq b) = F_{X}(b) - F_{X}(a)$

        $\rightarrow 0 \leq \frac{d}{dx}[F'_{X}]$

        $\rightarrow F_{X}(x) = 
            \left\{
                \begin{array}{lcl}
                    a & \quad & 0   \\
                    x & \quad & f(x)\\
                    \vdots
                    &&              \\
                    b & \quad & 1   \\
                \end{array}
            \right.
        $

        $\rightarrow \lim\limits_{x \to \infty}F_{X}(x) = 1$

        $\rightarrow \lim\limits_{x \to -\infty}F_{X}(x) = 0$

        Discrete Random Variables:

        $\rightarrow\mathbb{E}(X)/\mu = \sum\limits_{x}xp(x)$

        $\rightarrow\mathbb{E}(X^{2}) = \sum\limits_{x}x^{2}p(x)$

        $\rightarrow\text{var}(X) = \mathbb{E}(X^{2}) - \mathbb{E}(X)^{2}$

        $\rightarrow\text{sd}(X) = \sqrt{\text{var}(X)}$
        
        Binomials: $X \sim \text{Binomial}(A, p)$
        
        $\rightarrow\mathbb{E}(X) = n \times p$
        
        $\rightarrow\text{var}(X) = n \times p(1-p)$
        
        $\rightarrow\text{sd}(X) = \sqrt{\text{var}(X)}$

        \columnbreak

        Poisson:

        $\rightarrow\mathbb{E}(X) = \lambda = \text{var}(X)$

        $\rightarrow\text{PMF}/P(X=k) = \dfrac{\lambda^{k}e^{-\lambda}}{k!}$

        $\rightarrow P(\text{$k$ events in $t$ interval}) = \dfrac{(rt)^{k}e^{-rt}}{k!}$      

        Continuous Random Variables: 
        
        $\rightarrow\mathbb{E}(X) = \int_{-\infty}^{\infty}\left[xf(x)\right]dx$ 
        
        $\rightarrow\text{var}(X) = \int_{-\infty}^{\infty}\left[(x - \mu_X)^2\right]f(x)dx$
        
        $\rightarrow\text{sd}(X) = \sqrt{\text{var}(X)}$
    \end{multicols*}

\end{document}