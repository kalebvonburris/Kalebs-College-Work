% Files using this must be two subfolders
% deep. Adjust the number of ../ for the
% depth of the file.
% Imports
\usepackage{fancyhdr}
\usepackage{geometry}
\usepackage{icomma}
\usepackage{amsmath}
\usepackage{multicol}
\usepackage{mathptmx}
\usepackage{anyfontsize}
\usepackage{t1enc}
\usepackage{tabto}
\usepackage{listings}
\usepackage{filecontents}
\usepackage{subcaption}
\usepackage{tikz}
\usepackage[parfill]{parskip}
\usepackage{graphicx}
\usepackage[]{mdframed}
\usepackage{amsmath}
\usepackage[makeroom]{cancel}
\usepackage{pgfplots}
\usepackage{pgfplotstable}
\usepackage{xfrac}
\usepackage{amssymb}
\usepackage{mathtools}
\pgfplotsset{compat=1.18}
\usetikzlibrary{patterns}
\usepgfplotslibrary{polar}
\usepgfplotslibrary{fillbetween}

\geometry{margin=2.5cm}

\newcommand{\name}{Kaleb Burris}
\newcommand{\classname}{MATH F253, Elizabeth S. Allman, University of Alaska Fairbanks}
\newcommand{\assignment}{FILL IN ASSIGNMENT NAME}

\pagestyle{fancy}

\fancyhead[L]{
    \name 
    \newline
    \classname
    \newline
    \assignment
}

\newcommand{\horizontal}{\noindent\rule{\hsize}{0.4pt}}

\setlength{\headheight}{42pt}
\setlength{\headsep}{0.25in}
\setlength{\columnsep}{0.35cm}
\setlength{\columnseprule}{1pt}

\usepackage[T1]{fontenc}
\usepackage{lmodern}

% Put class number, class name, and professor 
% name.
% Use only in case of emergency, this
% should be covered by the preamble.
% \renewcommand\classname{}

% Put the assignment name with \S if 
% necessary for the section and the question 
% numbers.
\renewcommand\assignment{HW 5, Due Friday, 2/23/2023 23:59; \S5 Q: \#12, 15, 16 P: \#26, 41, 46, 52, 53  \S4 P: \#25, 33}
\setlength{\belowdisplayskip}{0pt} \setlength{\belowdisplayshortskip}{0pt}
\setlength{\abovedisplayskip}{0pt} \setlength{\abovedisplayshortskip}{0pt}

\begin{document}

    % Templates
    \iffalse
    % Use these for equations.
    \begin{equation*}
        \begin{gathered}
            Equations go here.
        \end{gathered}
    \end{equation*}

    % Use this if a line of math is too long.
    \resizebox{\hsize}{!}{$Long equation goes here$}

    % Use these for multiple columns.
    \begin{multicol*}{# of columns}
        % Remove the * if you want the columns to be balanced.
    \end{multicol*}

    % Use this to add a horizontal line.
    \horizontal

    \fi

    % Begin homework here.
    %%%%%%%%%%%%%%%%%%%%%%

    \section*{Chapter 6}

    \subsection*{Questions}
    
    \paragraph*{12.}\mbox{}
    \\
    \begin{mdframed}
        The normal force is larger than $mg$ because the hand is inputting an addition force, say $f_{h}$, that would generate a normal force of $fg + f_{h}$
    \end{mdframed}

    \paragraph*{15.}\mbox{}
    \\
    \begin{mdframed}
        \begin{enumerate}[label=\alph*.]
            \item Since the mass is doubled, and $F=ma$, the acceleration is halved. According to $v_{f} = v_{0}t + at$, with a $v_{f} = 0$, time to stop must at least be doubled. Therefore, the distance $d$ must also be doubled.
            
            \item If the initial velocity is doubled, the time to slow down to $v = 0$ will also be doubled, thus the block will travel twice the distance.
        \end{enumerate}
    \end{mdframed}

    \paragraph*{16.}\mbox{}
    \\
    \begin{mdframed}
        The friction points south, away from the direction the crate is accelerating. This is because it's acting as a normal force to the acceleration of the crate, and thus must point opposite to the direction it's accelerating in.
    \end{mdframed}

    \subsection*{Questions}
    
    \paragraph*{26.}\mbox{}
    \\
    \begin{mdframed}
        \begin{enumerate}[label=\alph*.]
            \item 
            \item 
            \item $F_{s} = 10(9.8)(0.5) = 49 N$. Thus, a force > $49 N$ is needed. Rearranging, $49 = 10(a)(0.5) \Rightarrow a = $
        \end{enumerate}
    \end{mdframed}
\end{document}