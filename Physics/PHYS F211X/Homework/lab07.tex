% Files using this must be two subfolders
% deep. Adjust the number of ../ for the
% depth of the file.
\providecommand\pointsize{10pt}

\documentclass[\pointsize, letterpaper]{article}

% Imports
\usepackage{fancyhdr}
\usepackage{pgfplots}
\usepackage{geometry}
\usepackage{icomma}
\usepackage{amsmath}
\usepackage{multicol}
\usepackage{mathptmx}
\usepackage{anyfontsize}
\usepackage{t1enc}
\usepackage{tabto}
\usepackage{listings}
\usepackage{filecontents}
\usepackage{subcaption}
\usepackage{tikz}
\usepackage[parfill]{parskip}
\usepackage{graphicx}
\usepackage[]{mdframed}
\usepackage{amsmath}
\usepackage[makeroom]{cancel}
\pgfplotsset{compat=1.18}

\geometry{margin=2.5cm}

\newcommand{\name}{Kaleb Burris}
\newcommand{\classname}{MATH F252, Dr. J. Gimbel}
\newcommand{\assignment}{FILL IN ASSIGNMENT NAME}

\pagestyle{fancy}

\fancyhead[L]{
    \name 
    \newline
    \classname
    \newline
    \assignment
}

\newcommand{\horizontal}{\noindent\rule{\hsize}{0.4pt}}

\setlength{\headheight}{42pt}
\setlength{\headsep}{0.25in}
\setlength{\columnsep}{0.35cm}
\setlength{\columnseprule}{1pt}

\usepackage{enumitem}
\usepackage{graphicx}
\graphicspath{ {./lab06images/} }

% Put class number, class name, and professor 
% name.
% Use only in case of emergency, this
% should be covered by the preamble.
% \renewcommand\classname{}

% Put the assignment name with \S if 
% necessary for the section and the question 
% numbers.
\renewcommand\assignment{Lab 6: Momentum, 3/7/2023, Partners: Maite Valentin-Lugo, Seth Waln}

\begin{document}

    % Templates
    \iffalse
    % Use these for equations.
    \begin{equation*}
        \begin{gathered}
            Equations go here.
        \end{gathered}
    \end{equation*}

    % Use this if a line of math is too long.
    \resizebox{\hsize}{!}{$Long equation goes here$}

    % Use these for multiple columns.
    \begin{multicol*}{# of columns}
        % Remove the * if you want the columns to be balanced.
    \end{multicol*}

    % Use this to add a horizontal line.
    \horizontal

    \fi

    % Begin homework here.
    %%%%%%%%%%%%%%%%%%%%%%

    \begin{enumerate}
        \item [1.]
        \begin{align*}
            p_{i}   & = m \cdot v_{ix}      \\
            p_{f}   & = (M+m) \cdot v_{fx}  \\
            v_{ix}  & = \frac{p_{f}}{m}     \\
                    & = \boxed{\frac{(M+m) \cdot v_{fx}}{m}}
        \end{align*}

        \item [2.]
        \begin{align*}
            E_{i}   & = \frac{1}{2}(M+m)(v_{fx})^{2}    \\ 
        \end{align*}

        \item [3.]
        \begin{align*}
            E_{f}   & = \frac{1}{2}(M+m)(g)(\Delta h)   \\
            E_{i}   & = E_{f}                           \\
            V_{ix}  & = \sqrt{2g\Delta h}               \\
            v_{ix}  & = \frac{(M+m) \cdot v_{fx}}{m}    
                      = \boxed{\frac{(M+m)\sqrt{2g\Delta h}}{m}}
        \end{align*}

        \item [4.]
        \begin{align*}
            y - y_{0}   & = y_{1}                       \\
            y_{l}       & = \frac{1}{2}gt^{2}           \\
            t           & = \boxed{\sqrt{\frac{2y_{1}}{g}}}
        \end{align*}

        \item [5.]
        \begin{align*}
            \Delta x    & = v_{ix} \cdot t                  \\
            \Delta x    & = \frac{(M+m)\sqrt{2g\Delta h}}{m} \cdot 
                            \boxed{\sqrt{\frac{2y_{1}}{g}}} \\
        \end{align*}

        \item [6.]
        \begin{align*}
            m   & = 0.0661 \pm 0.00005 kg                   \\
            M   & = 0.1648 \pm 0.00005 kg                   \\  
        \end{align*}

        \pagebreak

        \item [7.]
        


        \item [8.]
        


        \item [9.]
        


        \item [10.]
        
        \begin{align*}
            \Delta x    & \stackrel{\text{Excel}}{=} 1.9187 m
        \end{align*}

        \item [11.]
        
        \resizebox{\hsize}{!}{\begin{equation*}
            \delta x    & = \sqrt{
                \left(\frac{2}{0.0661}\sqrt{1.9187 \cdot 1.282}(0.00005)\right)^{2} +
                \left(\frac{-2(0.1648)}{0.0661^{2}}\sqrt{1.9187 \cdot 1.282}(0.00005)\right)^{2} +
                \left(\frac{2}{0.0661}\sqrt{1.9187 \cdot 1.282}(0.00005)\right)^{2} +
            }
        \end{equation*}}

    \end{enumerate}

\end{document}