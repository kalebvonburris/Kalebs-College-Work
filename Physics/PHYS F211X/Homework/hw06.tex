% Files using this must be two subfolders
% deep. Adjust the number of ../ for the
% depth of the file.
% Imports
\usepackage{fancyhdr}
\usepackage{geometry}
\usepackage{icomma}
\usepackage{amsmath}
\usepackage{multicol}
\usepackage{mathptmx}
\usepackage{anyfontsize}
\usepackage{t1enc}
\usepackage{tabto}
\usepackage{listings}
\usepackage{filecontents}
\usepackage{subcaption}
\usepackage{tikz}
\usepackage[parfill]{parskip}
\usepackage{graphicx}
\usepackage[]{mdframed}
\usepackage{amsmath}
\usepackage[makeroom]{cancel}
\usepackage{pgfplots}
\usepackage{pgfplotstable}
\usepackage{xfrac}
\usepackage{amssymb}
\usepackage{mathtools}
\pgfplotsset{compat=1.18}
\usetikzlibrary{patterns}
\usepgfplotslibrary{polar}
\usepgfplotslibrary{fillbetween}

\geometry{margin=2.5cm}

\newcommand{\name}{Kaleb Burris}
\newcommand{\classname}{MATH F253, Elizabeth S. Allman, University of Alaska Fairbanks}
\newcommand{\assignment}{FILL IN ASSIGNMENT NAME}

\pagestyle{fancy}

\fancyhead[L]{
    \name 
    \newline
    \classname
    \newline
    \assignment
}

\newcommand{\horizontal}{\noindent\rule{\hsize}{0.4pt}}

\setlength{\headheight}{42pt}
\setlength{\headsep}{0.25in}
\setlength{\columnsep}{0.35cm}
\setlength{\columnseprule}{1pt}

\usepackage[T1]{fontenc}
\usepackage{lmodern}

% Put class number, class name, and professor 
% name.
% Use only in case of emergency, this
% should be covered by the preamble.
% \renewcommand\classname{}

% Put the assignment name with \S if 
% necessary for the section and the question 
% numbers.
\renewcommand\assignment{HW 6, Due Friday, 3/3/2023 23:59; \S7 Q: \#11 P: \#42, 45 \S8 P: \#13, 22 ,26, 30, 51, 56}
\setlength{\belowdisplayskip}{0pt} \setlength{\belowdisplayshortskip}{0pt}
\setlength{\abovedisplayskip}{0pt} \setlength{\abovedisplayshortskip}{0pt}

\begin{document}

    % Templates
    \iffalse
    % Use these for equations.
    \begin{equation*}
        \begin{gathered}
            Equations go here.
        \end{gathered}
    \end{equation*}

    % Use this if a line of math is too long.
    \resizebox{\hsize}{!}{$Long equation goes here$}

    % Use these for multiple columns.
    \begin{multicol*}{# of columns}
        % Remove the * if you want the columns to be balanced.
    \end{multicol*}

    % Use this to add a horizontal line.
    \horizontal

    \fi

    % Begin homework here.
    %%%%%%%%%%%%%%%%%%%%%%

    \section*{Chapter 7}

    \subsection*{Questions}

    \paragraph*{11.}\mbox{}

    Because both masses are the same, the reading on the scale is 5 kg.

    \subsection*{Problems}

    \paragraph*{42.}\mbox{}

    \begin{enumerate}[label=\alph*.]
        \item 

        \begin{align*}
            F_{cw,x}    & = (1500*9.8)\sin(20^{\circ}) = 5027.7 N       \\
            F_{car,x}   & = (2000*9.8)\sin(30^{\circ}) = 9800.0 N       \\
            \text{Find } F_{break}: F_{car, x} - F_{break} & = F_{cw,x} \\
            9800.0 - F_{break} & = 5027.7                               \\
            F_{break}   & = \boxed{4772.3 N}
        \end{align*}

        \item

        \begin{align*}
            \text{Find $v_{f}$ of } v_{f}^{2}   
                                & = v_{0}^{2} + 2\vec{a}\Delta x        \\
            v_{0}               & = 0 m/s                               \\
            \vec{a}_{car,x}     & = 9.8\sin(30) = 4.9 m/s               \\
            \Delta x            & = \frac{200m}{\sin(30^{\circ})}
                                  = \frac{200m}{0.5} = 400 m            \\
            v_{f}^{2}           & = 0^{2} + 2(4.9)(400)     
                                  \Rightarrow \sqrt{3920} = \boxed{62.6m/s}
        \end{align*}
    \end{enumerate}

    \paragraph*{45.}\mbox{}

    \begin{align*}
        F_{y}   & = -(70 + 10 kg)(9.8m/s^{2}) + -(70 + 10 kg)(0.2m/s^{2})   \\
                & = \boxed{-800.0 N}
    \end{align*}

    \section*{Chapter 8}

    \subsection*{Problems}

    \paragraph*{13.}\mbox{}

    The question is: what velocity must $m_{1}$ rotate at such that the radius $r$ does not change?

    \begin{align*}
        
    \end{align*}

    \paragraph*{22.}\mbox{}

    \paragraph*{26.}\mbox{}

    \paragraph*{30.}\mbox{}

    \paragraph*{51.}\mbox{}

    \paragraph*{56.}\mbox{}
    
\end{document}