\providecommand\pointsize{10pt}

\documentclass[\pointsize, letterpaper]{article}

% Imports
\usepackage{fancyhdr}
\usepackage{pgfplots}
\usepackage{geometry}
\usepackage{icomma}
\usepackage{amsmath}
\usepackage{multicol}
\usepackage{mathptmx}
\usepackage{anyfontsize}
\usepackage{t1enc}
\usepackage{tabto}
\usepackage{listings}
\usepackage{subcaption}
\usepackage{tikz}
\usepackage[parfill]{parskip}
\usepackage{graphicx}
\usepackage{mdframed}
\usepackage{amsmath}
\usepackage{enumitem}
\usepackage[makeroom]{cancel}
\pgfplotsset{compat=1.18}
\usepackage{tkz-euclide}
\usepackage{siunitx}
\usepackage{resizegather}
\usetikzlibrary{angles}
\usepackage{enumitem}
\usepackage{graphicx}

\geometry{margin=2.5cm}

\newcommand{\name}{Kaleb Burris}
\newcommand{\classname}{PHYS 211X}
\newcommand{\assignment}{FILL IN THE ASSIGNMENT}

\pagestyle{fancy}

\fancyhead[L]{
    \name 
    \newline
    \classname
    \newline
    \assignment
}

\newcommand{\horizontal}{\noindent\rule{\hsize}{0.4pt}}

\setlength{\headheight}{42pt}
\setlength{\headsep}{0.25in}
\setlength{\columnsep}{0.35cm}
\setlength{\columnseprule}{1pt}

% Put class number, class name, and professor 
% name.
% Use only in case of emergency, this
% should be covered by the preamble.
% \renewcommand\classname{}

% Put the assignment name with \S if 
% necessary for the section and the question 
% numbers.
\renewcommand\assignment{HW 9, Due Friday, 3/31/2023 23:59; \S11 P: \#52, 54, 71, 81 \S12 Q: \#7, 9 P: \#32, 58, 65, 75, 82}
\setlength{\belowdisplayskip}{0pt} \setlength{\belowdisplayshortskip}{0pt}
\setlength{\abovedisplayskip}{0pt} \setlength{\abovedisplayshortskip}{0pt}

\begin{document}

    % Templates
    \iffalse
    % Use these for equations.
    \begin{equation*}
        \begin{gathered}
            Equations go here.
        \end{gathered}
    \end{equation*}

    % Use this if a line of math is too long.
    \resizebox{\hsize}{!}{$Long equation goes here$}

    % Use these for multiple columns.
    \begin{multicol*}{# of columns}
        % Remove the * if you want the columns to be balanced.
    \end{multicol*}

    % Use this to add a horizontal line.
    \horizontal

    \fi

    % Begin homework here.
    %%%%%%%%%%%%%%%%%%%%%%

    \section*{Chapter 11}

    \subsection*{Problems}

    \begin{itemize}
        \item [52.]
        \item [54.]
        \item [71.]
        \item [81.]
    \end{itemize}

    \section*{Chapter 12}

    \subsection*{Questions}

    \begin{itemize}
        \item [7.] 
        Roll both balls. Since the mass is distributed further from the center of mass on the hollow ball, it will roll slower than the solid ball.

        \item [9.]

        \begin{itemize}
            \item [a.]
            The torque is positive.
            \item [b.]
            (iii) hold steady. Since there is no friction or opposing force, the angular velocity won't decrease.
        \end{itemize}

    \end{itemize}
    \subsection*{Problems}

    \begin{itemize}
        \item [32.]
        
        Equilibrium:
        \begin{align*}
            \tau    & = \tau                                                \\
            r_{cat}F_{cat}  & = r_{tuna}F_{tuna}                            \\
            r_{cat}(4.0kg)(9.8m/s^{2}) & = 4.0m(9.8m/s^{2})(2.0kg)          \\
            r_{cat} & = \frac{4.0m(9.8m/s^{2})(2.0kg)}{(4.0kg)(9.8m/s^{2})} \\
            r_{cat} & = \boxed{2.0m}
        \end{align*}

        \item [58.]
        
        \begin{align*}
            F_{ft} + F_{fb} - F_{g} & = 0
        \end{align*}

        \item [65.]
        \item [75.]
        \item [82.]
    \end{itemize}
\end{document}