% Files using this must be two subfolders
% deep. Adjust the number of ../ for the
% depth of the file.
\providecommand\pointsize{10pt}

\documentclass[\pointsize, letterpaper]{article}

% Imports
\usepackage{fancyhdr}
\usepackage{pgfplots}
\usepackage{geometry}
\usepackage{icomma}
\usepackage{amsmath}
\usepackage{multicol}
\usepackage{mathptmx}
\usepackage{anyfontsize}
\usepackage{t1enc}
\usepackage{tabto}
\usepackage{listings}
\usepackage{filecontents}
\usepackage{subcaption}
\usepackage{tikz}
\usepackage[parfill]{parskip}
\usepackage{graphicx}
\usepackage[]{mdframed}
\usepackage{amsmath}
\usepackage[makeroom]{cancel}
\pgfplotsset{compat=1.18}

\geometry{margin=2.5cm}

\newcommand{\name}{Kaleb Burris}
\newcommand{\classname}{MATH F252, Dr. J. Gimbel}
\newcommand{\assignment}{FILL IN ASSIGNMENT NAME}

\pagestyle{fancy}

\fancyhead[L]{
    \name 
    \newline
    \classname
    \newline
    \assignment
}

\newcommand{\horizontal}{\noindent\rule{\hsize}{0.4pt}}

\setlength{\headheight}{42pt}
\setlength{\headsep}{0.25in}
\setlength{\columnsep}{0.35cm}
\setlength{\columnseprule}{1pt}

% Put class number, class name, and professor 
% name.
% Use only in case of emergency, this
% should be covered by the preamble.
% \renewcommand\classname{}

% Put the assignment name with \S if 
% necessary for the section and the question 
% numbers.
\renewcommand\assignment{HW 3, Due Friday, 2/10/2023 23:59; \S3 P: \#13, 24, 28, 35, 41, 42 \S4 Q: \# 10, 11 P: \#7, 11, 20}
\setlength{\belowdisplayskip}{0pt} \setlength{\belowdisplayshortskip}{0pt}
\setlength{\abovedisplayskip}{0pt} \setlength{\abovedisplayshortskip}{0pt}

\begin{document}

    % Templates
    \iffalse
    % Use these for equations.
    \begin{equation*}
        \begin{gathered}
            Equations go here.
        \end{gathered}
    \end{equation*}

    % Use this if a line of math is too long.
    \resizebox{\hsize}{!}{$Long equation goes here$}

    % Use these for multiple columns.
    \begin{multicol*}{# of columns}
        % Remove the * if you want the columns to be balanced.
    \end{multicol*}

    % Use this to add a horizontal line.
    \horizontal

    \fi

    % Begin homework here.
    %%%%%%%%%%%%%%%%%%%%%%

    \begin{multicols*}{2}
        \section*{Chapter 3}
        \subsection*{Problems}

        \paragraph*{13.}
        A wildlife researcher is tracking a flock of geese. The geese fly 4.0 km due west, then turn toward the north by 40$^{\circ}$ and fly another 4.0 km. How far west are they of their initial position? What is the magnitude of their displacement?

        \begin{mdframed}
            \begin{center}
                \begin{tikzpicture}
                    \draw[->] (0,0) coordinate(A) -- (-2,0) coordinate(B) -- ++(140:2) coordinate(C)
                    pic [draw,->,red,thick,angle radius=1cm] {angle = C--B--A};
                    \draw node at (-1,0)[anchor=south]{4 km};
                    \draw node at (-3,1)[anchor=west]{4 km};
                    \draw[|-|] (-3.5,-0.25) -- (0,-0.25);
                    \draw node at (-1.75,-0.25)[anchor=north]{$\Delta x$};
                \end{tikzpicture}
            \end{center}
            
            \begin{equation*}
                \begin{gathered}
                    \Delta x = 4 + 4 \cos(40)
                \end{gathered}
            \end{equation*}
        \end{mdframed}

        \section*{Chapter 4}
        \subsection*{Questions}

        \subsection*{Problems}

    \end{multicols*}
\end{document}