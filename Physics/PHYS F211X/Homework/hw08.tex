% Files using this must be two subfolders
% deep. Adjust the number of ../ for the
% depth of the file.
% Imports
\usepackage{fancyhdr}
\usepackage{geometry}
\usepackage{icomma}
\usepackage{amsmath}
\usepackage{multicol}
\usepackage{mathptmx}
\usepackage{anyfontsize}
\usepackage{t1enc}
\usepackage{tabto}
\usepackage{listings}
\usepackage{filecontents}
\usepackage{subcaption}
\usepackage{tikz}
\usepackage[parfill]{parskip}
\usepackage{graphicx}
\usepackage[]{mdframed}
\usepackage{amsmath}
\usepackage[makeroom]{cancel}
\usepackage{pgfplots}
\usepackage{pgfplotstable}
\usepackage{xfrac}
\usepackage{amssymb}
\usepackage{mathtools}
\pgfplotsset{compat=1.18}
\usetikzlibrary{patterns}
\usepgfplotslibrary{polar}
\usepgfplotslibrary{fillbetween}

\geometry{margin=2.5cm}

\newcommand{\name}{Kaleb Burris}
\newcommand{\classname}{MATH F253, Elizabeth S. Allman, University of Alaska Fairbanks}
\newcommand{\assignment}{FILL IN ASSIGNMENT NAME}

\pagestyle{fancy}

\fancyhead[L]{
    \name 
    \newline
    \classname
    \newline
    \assignment
}

\newcommand{\horizontal}{\noindent\rule{\hsize}{0.4pt}}

\setlength{\headheight}{42pt}
\setlength{\headsep}{0.25in}
\setlength{\columnsep}{0.35cm}
\setlength{\columnseprule}{1pt}

\usepackage[T1]{fontenc}
\usepackage{lmodern}

% Put class number, class name, and professor 
% name.
% Use only in case of emergency, this
% should be covered by the preamble.
% \renewcommand\classname{}

% Put the assignment name with \S if 
% necessary for the section and the question 
% numbers.
\renewcommand\assignment{HW 8, Due Friday, 3/24/2023 23:59; \S10 Q: P: \#24, 41, 45, 47 \S11 Q: \#5, 9, 13 P: \#27, 28, 40, 47}
\setlength{\belowdisplayskip}{0pt} \setlength{\belowdisplayshortskip}{0pt}
\setlength{\abovedisplayskip}{0pt} \setlength{\abovedisplayshortskip}{0pt}

\begin{document}

    % Templates
    \iffalse
    % Use these for equations.
    \begin{equation*}
        \begin{gathered}
            Equations go here.
        \end{gathered}
    \end{equation*}

    % Use this if a line of math is too long.
    \resizebox{\hsize}{!}{$Long equation goes here$}

    % Use these for multiple columns.
    \begin{multicol*}{# of columns}
        % Remove the * if you want the columns to be balanced.
    \end{multicol*}

    % Use this to add a horizontal line.
    \horizontal

    \fi

    % Begin homework here.
    %%%%%%%%%%%%%%%%%%%%%%

    \section*{Chapter 10}

    \subsection*{Problems}

    \begin{itemize}
        \item [24.]
        
        \begin{itemize}
            \item [a.]
            The particle will move to the right.

            \item [b.]
            \begin{align*}
                KE      & = 3.0 J \text{ at } x=4m  \\
                \vec{v} & = \sqrt{2m(KE)}           \\
                \vec{v} & = \sqrt{2(0.020)(3.0)}    \\
                        & = \boxed{0.35m/s \text{ at } x=4 m}
            \end{align*}

            \item [c.]
            There are turning points at $x=1m$ and $x=4m$.

        \end{itemize}

        \item [41.]
        \begin{align*}
            KE_{i} + U_{i}  & = KE_{f} + U_{f}                                      \\
            KE              & = \frac{1}{2}mv^{2}                                   \\
            U               & = mgy                                                 \\
            \frac{1}{2}mv^{2}_{i} + mgy_{i} & = \frac{1}{2}mv^{2}_{f} + mgy_{f}     \\
            v_{i}           & = \sqrt{v_{f}^{2}+2gy_{f}-2gy_{i}}                    \\
                            & = \sqrt{3^{2} + 2(-9.8)(0.20) - 2(-9.8)(0)}           \\
                            & = \sqrt{9 - 3.92} = \sqrt{5.08} = \boxed{2.25m/s}
        \end{align*}

        \item [45.]
        \begin{align*}
            K_{i}   & = 0               \\
            U_{i}   & = mgh             \\
            K_{f}   & = 1/2mgr          \\
            U_{f}   & = 2mgr            \\
            0 + mgh & = 1/2mgr + 2mgr   \\
            mgh     & = 5/2mgr          \\
            h       & = \boxed{\frac{5}{2}r}
        \end{align*}

        \item [47.]
        
        \begin{center}
            \begin{tikzpicture}
                \begin{axis}[
                    xlabel = Compression (cm),
                    ylabel = Height (cm),
                    ymin=0
                ]
                    \addplot[blue] coordinates
                    {(2,32) (3,65) (4,115) (5,189)};
                \end{axis}
            \end{tikzpicture}
        \end{center}

        \begin{align*}
            K_{i} + U_{i}   & = K_{f} + U_{f}   \\
            0 + 1/2kx^{2}   & = 0 + mgh         \\
            m               & = \frac{1/2kx^{2}}
                                     {gh}       \\
                            & = \frac{1/2(950)x^{2}}
                                     {-9.8h}
        \end{align*}

    \end{itemize}

    \pagebreak

    \section*{Chapter 11}

    \subsection*{Questions}

    \begin{itemize}
        \item [5.]
        \item [9.]
        \item [13.]
    \end{itemize}

    \subsection*{Problems}

    \begin{itemize}
        \item [27.]
        \item [28.]
        \item [40.]
        \item [47.]
    \end{itemize}

\end{document}