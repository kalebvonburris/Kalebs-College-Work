% Files using this must be two subfolders
% deep. Adjust the number of ../ for the
% depth of the file.
\providecommand\pointsize{10pt}

\documentclass[\pointsize, letterpaper]{article}

% Imports
\usepackage{fancyhdr}
\usepackage{pgfplots}
\usepackage{geometry}
\usepackage{icomma}
\usepackage{amsmath}
\usepackage{multicol}
\usepackage{mathptmx}
\usepackage{anyfontsize}
\usepackage{t1enc}
\usepackage{tabto}
\usepackage{listings}
\usepackage{filecontents}
\usepackage{subcaption}
\usepackage{tikz}
\usepackage[parfill]{parskip}
\usepackage{graphicx}
\usepackage[]{mdframed}
\usepackage{amsmath}
\usepackage[makeroom]{cancel}
\pgfplotsset{compat=1.18}

\geometry{margin=2.5cm}

\newcommand{\name}{Kaleb Burris}
\newcommand{\classname}{MATH F252, Dr. J. Gimbel}
\newcommand{\assignment}{FILL IN ASSIGNMENT NAME}

\pagestyle{fancy}

\fancyhead[L]{
    \name 
    \newline
    \classname
    \newline
    \assignment
}

\newcommand{\horizontal}{\noindent\rule{\hsize}{0.4pt}}

\setlength{\headheight}{42pt}
\setlength{\headsep}{0.25in}
\setlength{\columnsep}{0.35cm}
\setlength{\columnseprule}{1pt}

% Put class number, class name, and professor 
% name.
% Use only in case of emergency, this
% should be covered by the preamble.
% \renewcommand\classname{}

% Put the assignment name with \S if 
% necessary for the section and the question 
% numbers.
\renewcommand\assignment{HW 7, Due Friday, 3/10/2023 23:59; \S9 Q: \#7, 8 P: \#11, 18, 20, 22, 40 \S10 Q: \#4, 8 P: \#11, 15, 22}
\setlength{\belowdisplayskip}{0pt} \setlength{\belowdisplayshortskip}{0pt}
\setlength{\abovedisplayskip}{0pt} \setlength{\abovedisplayshortskip}{0pt}

\begin{document}

    % Templates
    \iffalse
    % Use these for equations.
    \begin{equation*}
        \begin{gathered}
            Equations go here.
        \end{gathered}
    \end{equation*}

    % Use this if a line of math is too long.
    \resizebox{\hsize}{!}{$Long equation goes here$}

    % Use these for multiple columns.
    \begin{multicol*}{# of columns}
        % Remove the * if you want the columns to be balanced.
    \end{multicol*}

    % Use this to add a horizontal line.
    \horizontal

    \fi

    % Begin homework here.
    %%%%%%%%%%%%%%%%%%%%%%

    \section*{Chapter 9}

    \subsection*{Questions}

    \begin{itemize}
        \item [7.]
        
            While the particle is moving up, the work done by gravity is negative. While it's moving down, the work done is positive. While it's not moving in either direction, it's doing zero work.

        \item [8.]
        
            Since work is $F \cdot d$, and the force of pulling the rope on a plane is reduced, proportionally to an increase in the distance you need to travel, the product of both cases will be the same.

    \end{itemize}

    \subsection*{Problems}

    \begin{itemize}
        \item [11.]
        
            \begin{itemize}
                
                \item [a.]
                
                \begin{align*}
                    \vec{A} \cdot \vec{B}   & = A_{x}B_{x} + A_{y}B_{y}         \\
                                            & = (4)(-2) + (-2)(-3)
                                              = -8 + 6 = \boxed{-2}
                \end{align*}

                \item [b.]
                
                \begin{align*}
                    \vec{A} \cdot \vec{B}   & = (-4)(2) + (2)(4) = -8 + 8 = \boxed{0}
                \end{align*}
            
            \end{itemize}

        \item [18.]
        
            \begin{align*}
                W_{\vec{F}_{G}} & = (2500 N)(-5.00 m) \approx \boxed{-12,500 J}               \\
                W_{\vec{T}_{1}} & = (1830 N)(5.00 m)(\sin(60^{\circ})) \approx \boxed{7924 J} \\
                W_{\vec{T}_{2}} & = (1295 N)(5.00 m)(\sin(60^{\circ})) \approx \boxed{4578 J}
            \end{align*}

        \item [20.]
        
            \begin{itemize}

                \item [0-1:] $4 N \cdot 1 m = 4 J$
                \item [1-2:] $4 N \cdot 0.5 m - 4 N \cdot 0.5 m  = 0 J$
                \item [2-3:] $2 N \cdot 1 m = 2 J$
                
            \end{itemize}

        \item [22.]
        
            \begin{align*}
                K_{i}   & = \frac{1}{2}mv^{2} = \frac{1}{2}0.5(2.0)^{2} = 1 J   \\ 
                W       & = \int_{0}^{2}F_{x}(x)dx = 50 J                       \\
                K_{t}   & = K_{i} + W = 50 + 1 = 51 J                           \\
                v_{x=2} & = \sqrt{\frac{2K_{t}}{m}} = \sqrt{\frac{2(51)}{0.5}} 
                          = \boxed{204 m/s}                                     \\
                v_{x=4} & = \sqrt{\frac{2(1)}{0.5}} = \boxed{2 m/s}
            \end{align*}

        \item [40.]
        
            \begin{itemize}
                \item [a.]

                    \begin{align*}
                        v_{f}   & = \frac{50 m}{7.0 s} = 7.1 m/s                        \\
                        K_{t}   & = \frac{1}{2}(50 kg)(7.1 m/s)^{2} = 177.5 J = W       \\
                        F       & = \frac{W}{d} = \frac{177.5 J}{50 m} = \boxed{3.55 N}
                    \end{align*}

                \item [b.]
                
                    \begin{align*}
                        P_{s=2} = P_{s=4} = P_{s=6} = \frac{W}{t} = \frac{177.5}{7.0} 
                            = \boxed{25.4 W} 
                    \end{align*}

            \end{itemize}

    \end{itemize}

    \pagebreak

    \section*{Chapter 10}

    \subsection*{Questions}

    \begin{itemize}
        \item [4.]
        
            $v_{a}$ and $v_{b}$ will be the same because with an initial velocity on ball b, the velocity of ball a when it returns to $y=0$ will be equal to ball b. Ball c has no additional vertical velocity and thus will have a velocity lower than a and b.

        \item [8.]
        
            \begin{itemize}
                \item [a.]
                
                    $x=6m$

                \item [b.]
                
                    It has a turning point at $x=8m$.
                
                \item [c.]
                
                    Yes, if the $E$ line is moved to the minima at $x=6m$, then a stable equilibrium will exist where the particle has no kinetic energy, thus it will be at rest. 

            \end{itemize}

    \end{itemize}

    \subsection*{Problems}

    \begin{itemize}
        \item [11.]
        
            \begin{equation*}
                \begin{gathered}
                    y_{0} = 10 m, \quad y{f} = 15 m, \quad v_{0} = 15 m/s               \\
                    K_{i} + U_{Gi} = \frac{1}{2}mv_{0}^{2} + mgy_{0} = K_{f} + U_{Gf}   \\
                    v_{f} = \sqrt{v_{0}^{2} + 2g(y_{0} - y_{f})} 
                \end{gathered}
            \end{equation*}

            \begin{align*}
                v_{f}   & = \sqrt{(15 m/s)^{2} + 2(-9.8 m/s^{2})(-5 m)} \\
                        & = \sqrt{225 + 98}                             \\
                        & = \boxed{18.0 m/s}
            \end{align*}
    
        \item [15.]
        
            \begin{align*}
                200 J   & = \frac{1}{2}(1000)(\Delta s)^{2}     \\
                0.4 J   & = (\Delta s)^{2}                      \\
                \Delta s& = \boxed{0.2 m}
            \end{align*}

        \item [22.]
        
            Because we've added another spring, we've doubled the amount of energy being stored. Therefore, the amount of energy released is doubled, and thus the velocity the block leaves the springs is quadrupled to $\boxed{4v_{0}}$ because the velocity derived from kinetic energy is the square root of a factor of the kinetic energy, thus $2^{2} = 4$.

    \end{itemize}

\end{document}