% Files using this must be two subfolders
% deep. Adjust the number of ../ for the
% depth of the file.
\providecommand\pointsize{10pt}

\documentclass[\pointsize, letterpaper]{article}

% Imports
\usepackage{fancyhdr}
\usepackage{pgfplots}
\usepackage{geometry}
\usepackage{icomma}
\usepackage{amsmath}
\usepackage{multicol}
\usepackage{mathptmx}
\usepackage{anyfontsize}
\usepackage{t1enc}
\usepackage{tabto}
\usepackage{listings}
\usepackage{filecontents}
\usepackage{subcaption}
\usepackage{tikz}
\usepackage[parfill]{parskip}
\usepackage{graphicx}
\usepackage[]{mdframed}
\usepackage{amsmath}
\usepackage[makeroom]{cancel}
\pgfplotsset{compat=1.18}

\geometry{margin=2.5cm}

\newcommand{\name}{Kaleb Burris}
\newcommand{\classname}{MATH F252, Dr. J. Gimbel}
\newcommand{\assignment}{FILL IN ASSIGNMENT NAME}

\pagestyle{fancy}

\fancyhead[L]{
    \name 
    \newline
    \classname
    \newline
    \assignment
}

\newcommand{\horizontal}{\noindent\rule{\hsize}{0.4pt}}

\setlength{\headheight}{42pt}
\setlength{\headsep}{0.25in}
\setlength{\columnsep}{0.35cm}
\setlength{\columnseprule}{1pt}

% Put class number, class name, and professor 
% name.
% Use only in case of emergency, this
% should be covered by the preamble.
% \renewcommand\classname{}

% Put the assignment name with \S if 
% necessary for the section and the question 
% numbers.
\renewcommand\assignment{Lab 0: Introduction to Physics, 1/24/2023, Partner: }

\begin{document}

    % Templates
    \iffalse
    % Use these for equations.
    \begin{equation*}
        \begin{gathered}
            Equations go here.
        \end{gathered}
    \end{equation*}

    % Use this if a line of math is too long.
    \resizebox{\hsize}{!}{$Long equation goes here$}

    % Use these for multiple columns.
    \begin{multicol*}{# of columns}
        % Remove the * if you want the columns to be balanced.
    \end{multicol*}

    % Use this to add a horizontal line.
    \horizontal

    \fi

    % Begin homework here.
    %%%%%%%%%%%%%%%%%%%%%%

    \section*{Lab 01: Introduction to Physics}

    \subsection*{Apparatus}

    \begin{itemize}
        \item Position and Velocity Graphs
        \begin{itemize}
            \item foam board
            \item motion detector
            \item lab quest
            \item meter stick
        \end{itemize}
        \item Jumping Students
        \begin{itemize}
            \item Force plates
            \item lab quest
            \item board
        \end{itemize}
    \end{itemize}

    \subsection*{Objective}

    The purpose of this lab is to introduce you to some of the concepts you will learn in physics this semester, without using equations. You’ll take some measurements and make some plots, but mostly you will be observing and thinking about what you see while trying to make sense of it. 

    \subsection*{Theory}

    To understand the world around us we make observations while looking for patterns and generalizations. From this we try to determine cause-and-effect relationships and make predictions. These days “making observations” frequently refers to using a search engine, but do not worry, kinesthetic-tactile learning is still a thing. One of the great benefits of a lab class is simply handling the equipment! 


    \subsection*{Procedure} 
    
    Instructions are bullet-points, questions to be answered are numbered, information such as this is just text. Note: “graph” and “plot” are used interchangeably.

    \pagebreak

    \section*{(Station 1) Position / Time and Velocity / Time Graphs}

    Follow up questions:
    
    \paragraph*{1.} How can you determine velocity information from a position / time graph? Be specific.

    \begin{mdframed}
        
    \end{mdframed}

    \paragraph*{2.} When a position / time graph has a curved (not straight) line, what can you say about your motion?
    
    \begin{mdframed}
        
    \end{mdframed}
    
    \paragraph*{3.} When a velocity / time graph crosses the time axis, what can you say about your motion?

    \begin{mdframed}
        
    \end{mdframed}

    \pagebreak

    \section*{(Station 2) Jumping Students}


\end{document}