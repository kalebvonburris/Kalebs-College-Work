% Files using this must be two subfolders
% deep. Adjust the number of ../ for the
% depth of the file.
\providecommand\pointsize{10pt}

\documentclass[\pointsize, letterpaper]{article}

% Imports
\usepackage{fancyhdr}
\usepackage{pgfplots}
\usepackage{geometry}
\usepackage{icomma}
\usepackage{amsmath}
\usepackage{multicol}
\usepackage{mathptmx}
\usepackage{anyfontsize}
\usepackage{t1enc}
\usepackage{tabto}
\usepackage{listings}
\usepackage{filecontents}
\usepackage{subcaption}
\usepackage{tikz}
\usepackage[parfill]{parskip}
\usepackage{graphicx}
\usepackage[]{mdframed}
\usepackage{amsmath}
\usepackage[makeroom]{cancel}
\pgfplotsset{compat=1.18}

\geometry{margin=2.5cm}

\newcommand{\name}{Kaleb Burris}
\newcommand{\classname}{MATH F252, Dr. J. Gimbel}
\newcommand{\assignment}{FILL IN ASSIGNMENT NAME}

\pagestyle{fancy}

\fancyhead[L]{
    \name 
    \newline
    \classname
    \newline
    \assignment
}

\newcommand{\horizontal}{\noindent\rule{\hsize}{0.4pt}}

\setlength{\headheight}{42pt}
\setlength{\headsep}{0.25in}
\setlength{\columnsep}{0.35cm}
\setlength{\columnseprule}{1pt}

% Put class number, class name, and professor 
% name.
% Use only in case of emergency, this
% should be covered by the preamble.
% \renewcommand\classname{}

% Put the assignment name with \S if 
% necessary for the section and the question 
% numbers.
\renewcommand\assignment{HW 4, Due Friday, 2/17/2023 23:59; \S4 P: \#47, 51, 5(?), 60, 67  \S4 Q: \# 14, 15 P: \#32, 39, 41}
\setlength{\belowdisplayskip}{0pt} \setlength{\belowdisplayshortskip}{0pt}
\setlength{\abovedisplayskip}{0pt} \setlength{\abovedisplayshortskip}{0pt}

\begin{document}

    % Templates
    \iffalse
    % Use these for equations.
    \begin{equation*}
        \begin{gathered}
            Equations go here.
        \end{gathered}
    \end{equation*}

    % Use this if a line of math is too long.
    \resizebox{\hsize}{!}{$Long equation goes here$}

    % Use these for multiple columns.
    \begin{multicol*}{# of columns}
        % Remove the * if you want the columns to be balanced.
    \end{multicol*}

    % Use this to add a horizontal line.
    \horizontal

    \fi

    % Begin homework here.
    %%%%%%%%%%%%%%%%%%%%%%

    \section*{Chapter 4}
    \subsection*{Problems}

    \paragraph*{47.}
    \begin{enumerate}[label=\alph*.]
        \item A projectile is launched with speed $v_0$ and angle $\theta$. Derive an expression for the projectile's maximum height $h$.
        
        \begin{mdframed}
            \begin{equation*}
                \begin{gathered}
                    v_f = v_0(\Delta t) + \frac{1}{2}a(\Delta t^2)          \\
                    v_y = v_0\sin\theta                                     \\
                    \Delta t = -\frac{v_y}{a}                               \\
                    0 = v_0\left(\frac{v_y}{a}\right) + \frac{1}{2}a\left(\frac{}\right)
                \end{gathered}
            \end{equation*}
        \end{mdframed}

        \item A baseball is hit with a speed of 33.6 m/s. Calculate its height and the distance traveled if it is hit at angles of 30.0$^\circ$, 35.0$^\circ$, and 60.0$^\circ$.
    \end{enumerate}

    \paragraph*{51.}
    A tennis player hits a ball 2.0 m above the ground. The ball leaves his racquet with a speed of 20.0 m/s at an angle of 5.0$^\circ$  above the horizontal. The horizontal distance to the net is 7.0 m, and the net is 1.0 m high. Does the ball clear the net? If so, by how much? If not, by how much does it miss?

    \paragraph*{5(?).}
    Fuck if I know

    \paragraph*{60.}
    You are asked to consult for the city's research hospital, where a group of doctors is investigating the bombardment of cancer tumors with high-energy ions. As FIGURE P4.60 shows, ions are fired directly toward the center of the tumor at speeds of 5.0 x 10° m/s. To cover the entire tumor area, the ions are deflected sideways by passing them between two charged metal plates that accelerate the ions perpendicular to the direction of their initial motion. The acceleration region is 5.0 cm long, and the ends of the acceleration plates are 1.5 m from the target. What sideways acceleration is required to deflect an ion 2.0 cm to one side?

    \paragraph*{67.}
    Communications satellites are placed in a circular orbit where they stay directly over a fixed point on the equator as the earth rotates. These are called geosynchronous orbits. The radius of the earth is 6.37 x 10 m, and the altitude of a geosynchronous orbit is 3.58 x 10 m (22,000 miles). What are (a) the speed and (b) the magnitude of the acceleration of a satellite in a geosynchronous orbit?

    \section*{Chapter 5}
    \subsection*{Questions}
    \paragraph*{14.}
    

\end{document}