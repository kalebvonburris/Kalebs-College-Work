\documentclass[12pt]{article}

% Imports
\usepackage{fancyhdr}
\usepackage{lipsum}
\usepackage{pgfplots}
\usepackage{geometry}
\usepackage{icomma}
\usepackage{amsmath}
\usepackage{multicol}
\usepackage[indent]{parskip}
\usepackage{subfiles}
\usepackage[makeroom]{cancel}
\usepackage{mathtools}
\usepackage{MnSymbol}
\usepackage{amsmath}
\usepackage{tkz-euclide}
\usepackage{siunitx}
\usepackage{calculator}
\usepackage{calculus}
\usepackage{mdframed}
\usepackage{titlesec}
\usepackage{hyperref}
\usepackage{natbib}
\usepackage[T1]{fontenc}
\usepackage{lmodern}
\usepackage{setspace}
\pgfplotsset{compat=1.18}

\geometry{margin=3cm}

\pagestyle{fancy}

\fancyhead[L]{
    Kaleb Burris
    \newline
    Potentials of Nuclear Fusion
    \newline
    PHYS 211X Dr.\,Chowdhury
}

\fancyhead[R] {}

\newcommand{\horizontal}{\noindent\rule{\hsize}{0.4pt}}

\setlength{\headheight}{42pt}
\setlength{\headsep}{0.5in}

\setlength{\columnsep}{1.5cm}
\setlength{\columnseprule}{1pt}

%\setlength{\abovedisplayskip}{0pt}
%\setlength{\belowdisplayskip}{3pt}

\titleformat{\section}[hang]{\Large\bfseries\filcenter}{\Roman{section}}{1.5em}{}

\bibliographystyle{plainnat}
\citestyle{conspar}

% Title page information

\title{
    \textbf{Potentials of Nuclear Fusion} 
    \\
    \vspace{0.5cm}
    \large PHYS 211X - General Physics I, Dr.\,Chowdhury
    \\
    \emph{University of Alaska Fairbanks}
}
\date{\today}
\author{
    Kaleb Burris
}

\doublespacing

\begin{document}

    \maketitle

    \thispagestyle{plain}

    \section{Abstract}

    This paper explores the potential of nuclear fusion as a source of energy and its possible impact on modern society. Nuclear fusion is the process of fusing atomic nuclei to produce vast amounts of energy, similar to the process that powers the sun. Fusion has several advantages, including its potential as an environmentally-friendly source of energy, as it generates no greenhouse gases and produces short-lived nuclear waste. Fusion also has implications for global politics, as it does not rely on scarce or politically sensitive resources, which could potentially reduce energy-related conflicts. 
    
    However, fusion also faces several challenges, including its theoretical state of development and the high cost of research and development. Despite recent advancements, fusion is not projected to be viable for several decades, which effectively ruins its potential as a solution to climate change. Nonetheless, fusion holds promise as a potential future source of energy that could revolutionize humanity in the future.

    \pagebreak

    \section{Introduction}

    The ability to produce and manipulate energy is a key indicator of human development, encompassing various forms such as electrical, thermal, and mechanical energy \citep[215]{hdbp}. In this paper, we explore the potential of nuclear fusion and its possible impact on energy production in modern society. Recent advancements in fusion technology \citep{reed_2022} have raised the prospect of fusion becoming a viable source of electrical power.

    Fusion is the process by which vast amounts of energy are produced as the result of fusing atomic nuclei, whose most obvious example is the sun, energy from which heats the Earth to livable conditions from millions of kilometers away. The current modern form of nuclear power is fission, which uses the heat generated by the decay of radioisotopes to generate power \citep{iaea_2022}.
    
    The potential societal impacts of harnessing fusion energy are significant, and in this paper, we aim to explore those potentials. We will discuss the challenges of fusion power, as well as its potential economic, environmental, and political impacts.

    \section{Current Sources of Energy}

    The dominant sources of energy production in the modern world are coal, oil, and natural gas, which together accounted for 80.9\% of the global energy supply in 2019 \citep[Supply]{KWES}. However, these sources are limited in supply, harmful to the environment, and require continuous extraction to sustain production. To be specific, these primary three sources produced 33,383 kilotons of carbon dioxide, which accounts for 99.3\% of the world's CO2 emissions \citep[Emissions]{KWES}.

    As alternatives to the traditional sources, nuclear fission and hydroelectric power generation combined made up only 7.5\% of the global energy supply in 2019, while solar, geothermal, wind, tidal energy, and all remaining sources accounted for a mere 2.2\% \citep[Supply]{KWES}. This highlights the need for alternative sources of power, and fusion power could potentially fill that gap.

    \section{Fusion's Advantages}

    Nuclear fusion is a process that involves the fusing of atomic nuclei, typically isotopes of hydrogen such as tritium and deuterium. Deuterium is a stable isotope of hydrogen that contains an extra neutron ($^{2}$H) and can be readily extracted from water through a process called hydrolysis \citep{iaea_2022}. Tritium, on the other hand, is a radioactive isotope of hydrogen and is relatively rare in nature. However, tritium can be produced during the fusion process itself, as the high-energy neutrons generated during fusion can interact with lithium to produce tritium. The good news is that lithium is abundant in the Earth's crust, making its supply for fusion reactions practically infinite \citep{iaea_2022}.

    One of the significant advantages of fusion power is its environmental impact. Unlike conventional forms of energy production, fusion generates no greenhouse gases, which are major contributors to climate change. Additionally, the nuclear waste produced by fusion reactions is extremely short-lived compared to the waste generated by nuclear fission reactions and is much more easily dealt with than the emissions produced by the three primary energy sources \citep{iaea_2022}. This defines fusion as an environmentally-friendly option for energy production.

    Furthermore, the potential of nuclear fusion as a source of energy has significant implications for global politics. As fusion does not rely on scarce or politically sensitive resources, such as fossil fuels, dangerous nuclear isotopes, or space and geographically demanding renewables such as solar or wind, it could potentially reduce energy-related conflicts \citep[]{KWES}. It would allow nations without those relevant and rare natural resources to generate energy on-demand, and could potentially be deployed extra-planetarily for future space missions.

    As of now, there has been one successful fusion experiment, in which, 2.05 MJ of energy was used to generate 3.15 MJ, a total net of 53.66\% energy gained from a single reaction, demonstrating just how much even our experimental models can produce \citep{reed_2022}.

    \section{Fusion's Disadvantages}

    Unfortunately, fusion is still in a theoretical state. Although we have successfully generated a positive net sum from a fusion experiment, the yield from such an experiment does not account for the initial energy cost of starting it, which was nearly 100 times the total output of the experiment. It also does not mean that fusion will be deployed soon. Estimates by those same researchers put a potential fusion reactor as being viable in ``a few decades'', although alternatives may reach that goal sooner \citep{reed_2022}. 

    This timescale is catastrophic for any hopes that fusion might be the solution to climate change. Projections claim that if we don't halve carbon emissions by 2030, it will be too late to combat the worst effects of climate change \citep{carrington_2022}. 

    However, fusion's biggest problem is in its cost of development. Currently, \$20 billion has been spent on fusion research, with nearly \$200 million in federal funding being spent this year. That also doesn't include the nearly \$3.4 billion spent in 2022 by private investors. All of this capital has been spent on a potential energy source that is not currently yielding any profit or even output. This also doesn't include the cost of the construction of a fusion power plant, its cost of maintenance, which will necessarily be steep as the running fusion processes generate temperatures above 150 million Kelvin, which will wear through any equipment that might come in contact with it instantly \citep{wurtz_2023}. This also does not include the price of the electrolysis process by which the main ingredient, deuterium, is generated.

    \section{Conclusion}

    In conclusion, while nuclear fusion holds great promise as a potential source of clean, abundant, and potentially cheap energy, there are still significant challenges to overcome. Fusion has several advantages, including its potential for producing vast amounts of energy without greenhouse gas emissions, its use of abundant and simple fuel sources, and its potential to reduce energy-related conflicts. However, fusion is still in the experimental stage, and the timeline for achieving practical fusion reactors is lengthy at best. The high cost of development and the urgent need to address climate change pose additional challenges that may delay or invalidate fusion development. 
    
    Despite these obstacles, recent advancements in fusion technology and successful experiments demonstrate progress in the field beyond our understanding only a few decades ago. With continued research and investment, nuclear fusion could become a viable source of clean energy in the close-to-mid future, offering hope and an alternative for a more sustainable and environmentally friendly energy production system on Earth, and potentially beyond. 

    \pagebreak

    \pagestyle{empty}

    \bibliography{SocietalPaperRefs}

    

\end{document}