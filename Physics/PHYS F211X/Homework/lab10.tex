\documentclass[10pt, letterpaper]{article}

% Imports
\usepackage{fancyhdr}
\usepackage{pgfplots}
\usepackage{geometry}
\usepackage{icomma}
\usepackage{amsmath}
\usepackage{multicol}
\usepackage{mathptmx}
\usepackage{anyfontsize}
\usepackage{t1enc}
\usepackage{tabto}
\usepackage{listings}
\usepackage{filecontents}
\usepackage{subcaption}
\usepackage{tikz}
\usepackage[parfill]{parskip}
\usepackage{graphicx}
\usepackage[]{mdframed}
\usepackage{amsmath}
\usepackage[makeroom]{cancel}
\pgfplotsset{compat=1.18}
\usepackage{tkz-euclide}
\usepackage{siunitx}
\sisetup{quotient-mode=fraction} % Output a/b as \frac{a}{b}

\geometry{margin=2.5cm}

\newcommand{\name}{Kaleb Burris}
\newcommand{\classname}{PHYS 211X}
\newcommand{\assignment}{FILL IN THE ASSIGNMENT}

\pagestyle{fancy}

\fancyhead[L]{
    \name 
    \newline
    \classname
    \newline
    \assignment
}

\newcommand{\horizontal}{\noindent\rule{\hsize}{0.4pt}}

\setlength{\headheight}{42pt}
\setlength{\headsep}{0.25in}
\setlength{\columnsep}{0.35cm}
\setlength{\columnseprule}{1pt}


\graphicspath{ {./lab09images/} }

% Put class number, class name, and professor 
% name.
% Use only in case of emergency, this
% should be covered by the preamble.
% \renewcommand\classname{}

% Put the assignment name with \S if 
% necessary for the section and the question 
% numbers.
\renewcommand\assignment{Lab 10, Day 2: An exploration and discussion of Natural Frequency and Resonance, 3/28/2023, Partners: Maite Valentin-Lugo, Seth Waln}

\begin{document}

    % Templates
    \iffalse
    % Use these for equations.
    \begin{equation*}
        \begin{gathered}
            Equations go here.
        \end{gathered}
    \end{equation*}

    % Use this if a line of math is too long.
    \resizebox{\hsize}{!}{$Long equation goes here$}

    % Use these for multiple columns.
    \begin{multicol*}{# of columns}
        % Remove the * if you want the columns to be balanced.
    \end{multicol*}

    % Use this to add a horizontal line.
    \horizontal

    \fi

    % Begin homework here.
    %%%%%%%%%%%%%%%%%%%%%%

    \section*{Station 1: Elastic Band}

    \begin{itemize}
        \item [1.]\mbox{}
        
        \begin{center}
            Driving Frequency: 50Hz
    
            \begin{tabular}{| c | c |}
                \hline
                Tension & Antinodes     \\
                \hline
                &\\
                \hline
                &\\
                \hline
                &\\
                \hline
                &\\
                \hline
                &\\
                \hline
                &\\
                \hline
            \end{tabular}        
        \end{center}

        \item [2.]\mbox{}
        
        \begin{center}
            Tension: 1N
    
            \begin{tabular}{| c | c |}
                \hline
                Driving Frequency & Antinodes     \\
                \hline
                &\\
                \hline
                &\\
                \hline
                &\\
                \hline
                &\\
                \hline
                &\\
                \hline
                &\\
                \hline
            \end{tabular}        
        \end{center}
    \end{itemize}
    

\end{document}