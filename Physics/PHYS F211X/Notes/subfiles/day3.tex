\documentclass[../Notes.tex]{subfiles}

\begin{document}

    \subsection*{Velocity}

        A vector from $A$ to $B$; a speed with direction. The first derivative of position: 
        \begin{equation*}
            \begin{gathered}
                v = \dot{s} = \frac{\mathrm{d}s}{\mathrm{d}t} = \frac{\Delta s}{\Delta t} = \frac{s_2-s_1}{t_2-t_1}  
            \end{gathered}
        \end{equation*}

    \subsection*{Acceleration}

        The first derivative of velocity, second derivative of position: 
        \begin{equation*}
            \begin{gathered}
                a = \ddot{s} = \frac{\mathrm{d}^2s}{\mathrm{d}t^2} = \frac{\Delta v}{\Delta t} = \frac{v_2-v_1}{t_2-t_1}         
            \end{gathered}
        \end{equation*}

    \subsection*{Examples}

        A ball is thrown up with $v = \SI{40}{\meter/\second}$. The velocity is recorded in 1 second increments as: 

        \{40, 30, 20, 10, 0\}. What is the acceleration? 
        \begin{equation*}
            \begin{gathered}
                \text{We set } v_2 = 0 \text{ as it's the last velocity and } v_1 = 40 \text{ as it's the first velocity.}  \\
                N = 5, \Delta t = 1: \quad v_1 = 0, \quad t_2 = \sum_{i=1}^{N-1}\Delta t = 4  \\
                \frac{v_2-v_1}{t_2-t_1} \implies \frac{0-40}{4-0} = -10 \\
                \text{Therefore, the acceleration is:} \quad \boxed{a = \SI{-10}{\meter/\second^2}}
            \end{gathered}
        \end{equation*}

\end{document}