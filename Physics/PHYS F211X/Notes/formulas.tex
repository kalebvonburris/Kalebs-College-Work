\documentclass[12pt, letterpaper]{book}

% Imports
\usepackage{fancyhdr}
\usepackage{pgfplots}
\usepackage{geometry}
\usepackage{icomma}
\usepackage{amsmath}
\usepackage{multicol}
\usepackage[parfill]{parskip}
\usepackage{subfiles}
\usepackage[makeroom]{cancel}
\usepackage{mathtools}
\usepackage{MnSymbol}
\usepackage{amsmath}
\usepackage{tkz-euclide}
\usepackage{siunitx}
\pgfplotsset{compat=1.18}

\geometry{margin=2.5cm}

\pagestyle{fancy}

\fancyhead[L]{
    Kaleb Burris
    \newline
    PHYS F211X
}

\newcommand{\horizontal}{\noindent\rule{\hsize}{0.4pt}}

\setlength{\headheight}{42pt}
\setlength{\headsep}{0.25in}
\setlength{\columnsep}{1.5cm}
\setlength{\columnseprule}{1pt}

\begin{document}

    \begin{titlepage}
        \Huge \textbf{PHYS 211X}

        \huge \textbf{General Physics I}

        \vfill

        \Large Kaleb Burris
    \end{titlepage}

    \section*{Formulas}
    \begin{multicols*}{2}
        Position:
        \begin{align}
            x, y, s, \text{ or } p  & = \vec{v}\Delta t = \int\vec{v}dt                     \\
                                 x  & = \left(\frac{v_{0}+v{f}}{2}\right)\Delta t           \\
                                 x  & = v_{0} t+\left(\frac{1}{2}\right)\vec{a} t^{2}       \\ 
        \end{align}

        Velocity:
        \begin{align}
            \vec{v}         & = \vec{a}\Delta t = \frac{d}{dt}[p] = \int(\vec{a})dt     \\
            \vec{v}         & = v_{0} + \vec{a}t                                        \\
            \vec{v}^{2}_{f} & = v^{2}_{0} + 2\vec{a}\Delta x                            \\
        \end{align}

        Acceleration
        \begin{equation}
            \vec{a} = \frac{d}{dt}[\vec{v}]
        \end{equation}

        Projectile Motion:
        \begin{equation}
            \label{eqn::yfinal}
            y_{f} = y_{0} + v_{0}(\Delta t) + \frac{1}{2}a(\Delta t^{2})
        \end{equation}

        Force:
        \begin{equation}
            \vec{F} = m\vec{a}
        \end{equation}

        Friction:
        \begin{equation}
            f = \mu N
        \end{equation}

        Drag: 
        \begin{equation}
            \vec{F_D}\text{ or }D = \frac{1}{2}\rho C_{D}Av^{2}
        \end{equation}

        Circular Motion:
        \begin{align}
            \vec{v}     & = r                               \\
            a_{cent}    & =\frac{\vec{v}^{2}}{r}            \\
            f           & = \mu n                           \\
            v_{cent}    & = \sqrt{gr}                       \\
            N           & = mr\omega^{2}                    \\
            N           & = 3mg                             \\
            \omega      & = \frac{\Delta \theta}{\Delta t}  \\
        \end{align}

        Total Energy:
        \begin{equation}
            E = K + U_{0}
        \end{equation}

        PE of a spring:
        \begin{equation}
            U_{p} = \frac{1}{2}k(x-L_{0})^{2}
        \end{equation}

        Total Potential Energy:
        \begin{equation}
            U_{tot} = mg + k(y-L_{0})
        \end{equation}

        Work:
        \begin{align}
            W_{int} & = -\frac{F_{x}}{\Delta }   \\
            F_{x}   & = -\frac{dU}{dx}
        \end{align} 
    \end{multicols*}

    \pagebreak

    \subsection*{Key}

    \begin{align*}
        v       & = \text{velocity, meters/second}                                  \\
        y       & = \text{height, meters}                                           \\
        x       & = \text{distance, meters}                                         \\
        t       & = \text{time, seconds}                                            \\
        m       & = \text{mass, kilograms}                                          \\
        a       & = \text{acceleration, meters/second}^{2}                          \\
        \theta  & = \text{angle, degrees}                                           \\
        g       & = \text{gravity: 9.8 meters/second}^{2}                           \\
        \omega  & = \text{angular velocity, radians or degrees/second, counter-clockwise}\\
        F       & = \text{force, Newtons, kilogram $\cdot$ meters/second}^{2}       \\
        \mu     & = \text{coefficient of friction}                                  \\
        N       & = \text{normal force, Newtons}                                    \\
        A       & = \text{area, meters}^{2}                                         \\
        \rho    & = \text{volumetric mass density, kilograms/meters}^{3}            \\
        C_{D}   & = \text{drag coefficient (geometry dependant)}                    \\
        K       & = \text{kinetic energy}                                           \\
        U       & = \text{potential energy}
    \end{align*}

\end{document}