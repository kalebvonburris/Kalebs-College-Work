\documentclass[10pt, letterpaper]{book}

% Imports
\usepackage{fancyhdr}
\usepackage{pgfplots}
\usepackage{geometry}
\usepackage{icomma}
\usepackage{amsmath}
\usepackage{multicol}
\usepackage[parfill]{parskip}
\usepackage{subfiles}
\usepackage[makeroom]{cancel}
\usepackage{mathtools}
\usepackage{MnSymbol}
\usepackage{amsmath}
\usepackage{tkz-euclide}
\usepackage{siunitx}
\usepackage{calculator}
\usepackage{calculus}
\usepackage{mdframed}
\pgfplotsset{compat=1.18}

\geometry{margin=2cm}

\pagestyle{fancy}

\fancyhead[L]{
    Kaleb Burris
    \newline
    PHYS F211X
}

\newcommand{\horizontal}{\noindent\rule{\hsize}{0.4pt}}

\setlength{\headheight}{42pt}
\setlength{\headsep}{0.25in}
\setlength{\columnsep}{1.5cm}
\setlength{\columnseprule}{1pt}

\setlength{\abovedisplayskip}{0pt}
\setlength{\belowdisplayskip}{3pt}

\begin{document}

    \begin{titlepage}
        \Huge \textbf{PHYS 211X}

        \huge \textbf{General Physics I}

        \vfill

        \Large Kaleb Burris
    \end{titlepage}

    \section*{Formulae}

    \begin{multicols*}{2}

        Dot Product:
        \begin{align}
            \vec{A}\cdot\vec{B} & = ||A|| \; ||B||\cos(\theta)
        \end{align}

        Cross Product:
        \begin{align}
            \vec{A}\times\vec{B} & = AB\sin(\theta)
        \end{align}

        Position:
        \begin{align}
            x, y, s, \text{ or } p  & = \vec{v}\Delta t = \int\vec{v}dt                     \\
                                 x  & = \left(\frac{v_{0}+v{f}}{2}\right)\Delta t           \\
                                 x  & = v_{0} t+\left(\frac{1}{2}\right)\vec{a} t^{2}       
        \end{align}

        Velocity, $v$:
        \begin{align}
            \vec{v}         & = \vec{a}\Delta t = \frac{d}{dt}[p] = \int(\vec{a})dt     \\
            \vec{v}         & = v_{0} + \vec{a}t                                        \\
            \vec{v}^{2}_{f} & = v^{2}_{0} + 2\vec{a}\Delta x                            
        \end{align}

        Acceleration, $a$:
        \begin{equation}
            \vec{a} = \frac{d}{dt}[\vec{v}]
        \end{equation}

        Projectile Motion:
        \begin{equation}
            \label{eqn::yfinal}
            y_{f} = y_{0} + v_{0}(\Delta t) + \frac{1}{2}a(\Delta t^{2})
        \end{equation}

        Force, $F$:
        \begin{equation}
            \vec{F} = m\vec{a}
        \end{equation}

        Friction:
        \begin{equation}
            f = \mu N
        \end{equation}

        Drag: 
        \begin{equation}
            \vec{F_D}\text{ or }D = \frac{1}{2}p C_{D}Av^{2}
        \end{equation}

        Circular Motion:
        \begin{align}
            \vec{v}         & = r                               \\
            F_{c}           & =\frac{m\vec{v}^{2}}{r}           \\
            f               & = \mu n                           \\
            v_{c}           & = \sqrt{gr}                       \\
            N               & = mr\omega^{2}                    \\
            N               & = 3mg                             \\
            \omega          & = \frac{\Delta \theta}{\Delta t}  
        \end{align}

        Total Energy:
        \begin{align}
            E   & = K + U   \\
            KE_{i} + U_{i}  & = KE_{f} + U_{f}  \\
            \frac{1}{2}mv^{2}_{i} + mgy_{i} & = \frac{1}{2}mv^{2}_{f} + mgy_{f}
        \end{align}

        PE of a spring:
        \begin{align}
            U     & = 1/2kx^{2}                 \\
            U_{p} & = \frac{1}{2}k(x-L_{0})
        \end{align}

        Potential Energy, $U$:
        \begin{align}
            U_{tot} & = mg + k\Delta y 
        \end{align}

        Work, $W$:
        \begin{align}
            W_{int} & = -\frac{F_{x}}{\Delta }   \\
            F_{x}   & = -\frac{dU}{dx}
        \end{align} 

        Momentum, $p$:
        \begin{align}
            p    & = mv
        \end{align}

        Torque, $\tau$:
        \begin{align}
            \tau & = r \times F     \\
            \tau & = rF\sin(\theta) \\
            N    & = I\alpha        
        \end{align}

        Inertia, $I$:
        \begin{align}
            I   & = \sum_{i}m_{i}r_{i}^{2} = \int r^{2}dm   \\
            I   & = r\omega^{2}
        \end{align}

        Kinetic Energy of Rotation:
        \begin{align}
            KE_{rot}  & = \frac{1}{2}I\omega^{2}
        \end{align}

        Kinetic Energy of Rolling:
        \begin{align}
            KE_{roll} & = \frac{1}{2}I\omega^{2} + \frac{1}{2}mv_{c}^{2}    
        \end{align}

        Angular Momentum:
        \begin{align}
            L = I\omega
        \end{align}

        Newton's Laws of Gravity:
        \begin{align}
            F   & = \frac{Gm_{1}m_{2}}{r^{2}}           \\
            U   & = -\frac{Gm_{1}m_{2}}{r^{2}}          \\
            G   & = 6.67 \times 10^{-11}Nm^{2}/kg^{2}   \\
            a_{m_{1}}   & = \frac{Gm_{2}}{r^{2}}        \\
            v_{e}   & = \sqrt{\frac{GM}{r}}
        \end{align}

        Orbits:
        \begin{align}
            \Delta T^{2}   & = \frac{4\pi^{2}}{GM_{\odot}}r^{3} \\
            \frac{4\pi^{2}}{GM_{\odot}} & = 1                   \\
            \Delta T^{2} & \propto r^{3}                        \\
            \vec{v} & = \frac{2\pi r}{\Delta t} 
            \text{ Valid for circular}                    
        \end{align}

        Density, $\rho$:
        \begin{align}
            \rho    & = \frac{m}{V}                             
        \end{align}

        Pressure, $\phi$:
        \begin{align}
            \phi    & = \frac{F}{A}                             \\
            \phi_{h}& = \phi_{0}e^{-\frac{mgh}{kT}}             \\
            \phi_{h}& = \rho_{l}hg + \phi_{a}
        \end{align}

        Bernoulli's Law:
        \begin{equation}
            \phi_{1} + \frac{1}{2}\rho v_{1}^{2} + \rho gh_{1} = 
            \phi_{2} + \frac{1}{2}\rho v_{2}^{2} + \rho gh_{2}
        \end{equation}

        Continuity:
        \begin{equation}
            A_{1}v_{1} = A_{2}v_{2}
        \end{equation}

    \end{multicols*}

    \pagebreak

    \subsection*{Key}
    \begin{align*}
        v       & = \text{velocity, meters/second}                                  \\
        y       & = \text{height, meters}                                           \\
        x       & = \text{distance, meters}                                         \\
        t       & = \text{time, seconds}                                            \\
        m       & = \text{mass, kilograms}                                          \\
        a       & = \text{acceleration, meters/second}^{2}                          \\
        \theta  & = \text{angle, degrees}                                           \\
        g       & = \text{gravity: 9.8 meters/second}^{2}                           \\
        \omega  & = \text{angular velocity, radians or degrees/second}              \\
        F       & = \text{force, Newtons, kilogram $\cdot$ meters/second}^{2}       \\
        \mu     & = \text{coefficient of friction}                                  \\
        N       & = \text{normal force, Newtons}                                    \\
        A       & = \text{area, meters}^{2}                                         \\
        \rho    & = \text{volumetric mass density, kilograms/meters}^{3}            \\
        C_{D}   & = \text{drag coefficient (geometry dependant)}                    \\
        K       & = \text{kinetic energy}                                           \\
        U       & = \text{potential energy}                                         \\
        \alpha  & = \text{angular acceleration, degrees|radians/second}^{2}
    \end{align*}
\end{document}