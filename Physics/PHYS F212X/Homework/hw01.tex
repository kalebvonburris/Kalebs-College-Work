\documentclass[12pt, letterpaper]{article}

% Imports
\usepackage{fancyhdr}
\usepackage{geometry}
\usepackage{icomma}
\usepackage{amsmath}
\usepackage{multicol}
\usepackage{mathptmx}
\usepackage{anyfontsize}
\usepackage{t1enc}
\usepackage{tabto}
\usepackage{listings}
\usepackage{filecontents}
\usepackage{subcaption}
\usepackage{tikz}
\usepackage[parfill]{parskip}
\usepackage{graphicx}
\usepackage[]{mdframed}
\usepackage{amsmath}
\usepackage[makeroom]{cancel}
\usepackage{pgfplots}
\usepackage{pgfplotstable}
\usepackage{xfrac}
\usepackage{amssymb}
\usepackage{mathtools}
\pgfplotsset{compat=1.18}
\usetikzlibrary{patterns}
\usepgfplotslibrary{polar}
\usepgfplotslibrary{fillbetween}

\geometry{margin=2.5cm}

\newcommand{\name}{Kaleb Burris}
\newcommand{\classname}{MATH F253, Elizabeth S. Allman, University of Alaska Fairbanks}
\newcommand{\assignment}{FILL IN ASSIGNMENT NAME}

\pagestyle{fancy}

\fancyhead[L]{
    \name 
    \newline
    \classname
    \newline
    \assignment
}

\newcommand{\horizontal}{\noindent\rule{\hsize}{0.4pt}}

\setlength{\headheight}{42pt}
\setlength{\headsep}{0.25in}
\setlength{\columnsep}{0.35cm}
\setlength{\columnseprule}{1pt}

\usepackage[T1]{fontenc}
\usepackage{lmodern}

% Put class number, class name, and professor 
% name.
% Use only in case of emergency, this
% should be covered by the preamble.
% \renewcommand\classname{}

% Put the assignment name with \S if 
% necessary for the section and the question 
% numbers.
\renewcommand\assignment{HW 1: Ch. 18; CQ: 8, 12; Problems: 29,37,38,48,50,56,58,66}

\begin{document}
    \section*{Chapter 18}
    \subsection*{Con. Questions}
    \begin{enumerate}
        \item [8.]
        
        \begin{itemize}
            \item [a.] \mbox{}
            \begin{mdframed}
                Given the ideal gas law of $PV = nRT$, as the volume decreases, the pressure will increase but the temperature will remain the same. Given that the vapor is at $-0.01^{\circ}$C and the pressure is so miniscule, the gas will eventually turn into ice and as the pressure increases, it will turn into water as it passes the melting point.
            \end{mdframed}

            \item[b.] \mbox{}
            \begin{mdframed}
                If the temperature starts at $0.02^{\circ}$C, then the vapor will never turn to ice; it will condense into water directly as the pressure passes the condensation line.
            \end{mdframed}
        \end{itemize}
        \item [12.] \mbox{}
        
        \begin{mdframed}
            Their $C$ line is linear, yet isothermal processes occur as a multiplicative function of pressure and volume. The $C$ line should be curved outwards from the inside of the triangle.
        \end{mdframed}
    \end{enumerate}

    \subsection*{Problems}

    \begin{enumerate}
        \item [29.]
        
        \begin{itemize}
            \item [a.] \mbox{}
            \begin{mdframed}
                When placed in a mixture of water and ice, the pressure will change from 1 atm to $\frac{273.15}{373.15}$ atm, or 0.73 atm.
            \end{mdframed}
            
            \item[b.] \mbox{}
            \begin{mdframed}
                Dry ice is 194.7K, thus the pressure would decrease to $\frac{194.7}{373.15}$ atm or 0.52 atm.
            \end{mdframed}
        \end{itemize}

        \item [37.]
        
        \begin{itemize}
            \item [a.] \mbox{}
            \begin{mdframed}
                This process is isothermic.
            \end{mdframed}

            \item [b.] \mbox{}
            \begin{mdframed}
                \begin{equation*}
                    \begin{gathered}
                        T = 101,350(0.0004)(0.020)(8.3145) = 5.056^{\circ}K     \\
                        5.056 - 273.15 = \boxed{-268.09^{\circ}K}
                    \end{gathered}
                \end{equation*}
            \end{mdframed}

            \item [c.] \mbox{}
            \begin{mdframed}
                \begin{equation*}
                    \begin{gathered}
                        V = \frac{0.020(8.3145)(5.056)}{304,050} = \frac{0.1106}{3} = \boxed{0.0000028m^{3}}
                    \end{gathered}
                \end{equation*}
            \end{mdframed}
        \end{itemize}

        \pagebreak
        
        \item [38.]
        \begin{itemize}
            \item [a.] \mbox{}
            \begin{mdframed}
                \begin{equation*}
                    \begin{gathered}
                        PV = nRT    \\
                        P = 101,350 Pa \quad V = 0.0001 m^{3} \quad n = 0.0050 \quad R = 8.3145   \\
                        T = \frac{10.135}{0.0050 \cdot 8.3145} = \boxed{243.79^{\circ}K}
                    \end{gathered}
                \end{equation*}
            \end{mdframed}

            \item [b.] \mbox{}
            \begin{mdframed}
                \begin{equation*}
                    \begin{gathered}
                        T = 2926^{\circ}K \quad V = 0.0003 m^{3} \quad n = 0.0050 \quad R = 8.3145   \\
                        P = \frac{0.0050(8.3145)(2926)}{0.0003} = \boxed{405,470.45\; Pa}
                    \end{gathered}
                \end{equation*}
            \end{mdframed}

            \item [c.] \mbox{}
            \begin{mdframed}
                \begin{equation*}
                    \begin{gathered}
                        P = 202,700 \quad T = 2438^{\circ}K \quad n = 0.0050 \quad R = 8.3145   \\
                         \frac{0.0050(8.3145)(2438)}{202,700} = \boxed{0.000025\; m^{3}}
                    \end{gathered}
                \end{equation*}
            \end{mdframed}
        \end{itemize}

        \item [48.] \mbox{}
        
        \begin{mdframed}
            \begin{equation*}
                \begin{gathered}
                    PV = nRT \\
                    101,350V = \frac{10,000}{23}(8.3145)(273.15)    \\
                    V = \frac{434.78(2,271.11)}{101,350} = 9.74m^{3}\\
                    \sqrt[3]{9.74} = \boxed{2.135m}
                \end{gathered}
            \end{equation*}
        \end{mdframed}

        \item [50.] I stared at this for 30 minutes. I have no idea where I'm even supposed to start. 1 atmosphere? 3 meters? A rubber room? They make me crazy.
        \item [56.] 
        \item [58.]
        \item [66.]
    \end{enumerate}
\end{document}