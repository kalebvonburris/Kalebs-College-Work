\documentclass[12pt, letterpaper]{article}

% Imports
\usepackage{fancyhdr}
\usepackage{geometry}
\usepackage{icomma}
\usepackage{amsmath}
\usepackage{multicol}
\usepackage{mathptmx}
\usepackage{anyfontsize}
\usepackage{t1enc}
\usepackage{tabto}
\usepackage{listings}
\usepackage{filecontents}
\usepackage{subcaption}
\usepackage{tikz}
\usepackage[parfill]{parskip}
\usepackage{graphicx}
\usepackage[]{mdframed}
\usepackage{amsmath}
\usepackage[makeroom]{cancel}
\usepackage{pgfplots}
\usepackage{pgfplotstable}
\usepackage{xfrac}
\usepackage{amssymb}
\usepackage{mathtools}
\pgfplotsset{compat=1.18}
\usetikzlibrary{patterns}
\usepgfplotslibrary{polar}
\usepgfplotslibrary{fillbetween}

\geometry{margin=2.5cm}

\newcommand{\name}{Kaleb Burris}
\newcommand{\classname}{MATH F253, Elizabeth S. Allman, University of Alaska Fairbanks}
\newcommand{\assignment}{FILL IN ASSIGNMENT NAME}

\pagestyle{fancy}

\fancyhead[L]{
    \name 
    \newline
    \classname
    \newline
    \assignment
}

\newcommand{\horizontal}{\noindent\rule{\hsize}{0.4pt}}

\setlength{\headheight}{42pt}
\setlength{\headsep}{0.25in}
\setlength{\columnsep}{0.35cm}
\setlength{\columnseprule}{1pt}

\usepackage[T1]{fontenc}
\usepackage{lmodern}

% Put class number, class name, and professor 
% name.
% Use only in case of emergency, this
% should be covered by the preamble.
% \renewcommand\classname{}

% Put the assignment name with \S if 
% necessary for the section and the question 
% numbers.
\renewcommand\assignment{HW 2: Ch. 19; CQ: 8,12; Problems: 11,19,24,37,38,50,57,62}

\begin{document}
    \section*{Chapter 18}
    \subsection*{Con. Questions}
    \begin{enumerate}
        \item [8.]
        
        \begin{itemize}
            \item [a.] \mbox{}
            \begin{mdframed}
                Given the ideal gas law of $PV = nRT$, as the volume decreases, the pressure will increase but the temperature will remain the same. Given that the vapor is at $-0.01^{\circ}$C and the pressure is so miniscule, the gas will eventually turn into ice and as the pressure increases, it will turn into water as it passes the melting point.
            \end{mdframed}

            \item[b.] \mbox{}
            \begin{mdframed}
                If the temperature starts at $0.02^{\circ}$C, then the vapor will never turn to ice; it will condense into water directly as the pressure passes the condensation line.
            \end{mdframed}
        \end{itemize}
        \item [12.] \mbox{}
        
        \begin{mdframed}
            Their $C$ line is linear, yet isothermal processes occur as a multiplicative function of pressure and volume. The $C$ line should be curved outwards from the inside of the triangle.
        \end{mdframed}
    \end{enumerate}

    \subsection*{Problems}

    \begin{enumerate}
        \item [11.]
        \item [19.]
        \item [24.]
        \item [37.]
        \item [38.]
        \item [50.]
        \item [57.]
        \item [62.]
    \end{enumerate}
\end{document}